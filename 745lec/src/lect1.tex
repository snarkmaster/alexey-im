\documentclass[11pt]{article}
\usepackage{cancel}
\usepackage{amsmath}
\usepackage{amssymb}
\usepackage{epsfig}
\usepackage{pstricks}

\newcommand{\handout}[5]{
  \noindent
  \begin{center}
  \framebox{
    \vbox{
      \hbox to 5.78in { {\bf 18.745 Introduction to Lie Algebras } \hfill #2 }
      \vspace{4mm}
      \hbox to 5.78in { {\Large \hfill #5  \hfill} }
      \vspace{2mm}
      \hbox to 5.78in { {\em #3 \hfill #4} }
    }
  }
  \end{center}
  \vspace*{4mm}
}

\newcommand{\lecture}[4]{\handout{#1}{#2}{#3}{Scribe: #4}{Lecture #1}}


\DeclareSymbolFont{AMSb}{U}{msb}{m}{n}
\DeclareMathSymbol{\N}{\mathbin}{AMSb}{"4E}
\DeclareMathSymbol{\Z}{\mathbin}{AMSb}{"5A}
\DeclareMathSymbol{\R}{\mathbin}{AMSb}{"52}
\DeclareMathSymbol{\Q}{\mathbin}{AMSb}{"51}
\DeclareMathSymbol{\I}{\mathbin}{AMSb}{"49}
\DeclareMathSymbol{\C}{\mathbin}{AMSb}{"43}
\DeclareMathSymbol{\F}{\mathbin}{AMSb}{"46}

\newcommand{\sll}{\mbox{sl}}
\newcommand{\gl}{\mbox{gl}}
\newcommand{\GL}{\mbox{GL}}
\newcommand{\tr}{\mbox{tr\ }}
\newcommand{\Mat}{\mbox{Mat}}
\newcommand{\Lie}{\mbox{Lie}}
\newcommand{\Der}{\mbox{Der\ }}
\newcommand{\End}{\mbox{End\ }}
\newcommand{\ad}{\mbox{ad\ }}
\newcommand{\im}{\mbox{im\ }}
\newcommand{\Ker}{\mbox{ker\ }}

\newcommand{\g}{\mathfrak{g}}
\newcommand{\h}{\mathfrak{h}}
\newcommand{\m}{\mathfrak{m}}
\newcommand{\He}{\mathfrak{H}}

\newcommand{\sk}{\vspace*{1em}}

\newtheorem{defn}{Definition}
\newtheorem{remark}{Remark}
\newtheorem{example}{Example}
\newtheorem{proof}{Proof}

% 1-inch margins, from fullpage.sty by H.Partl, Version 2, Dec. 15, 1988.
\topmargin 0pt
\advance \topmargin by -\headheight
\advance \topmargin by -\headsep
\textheight 8.9in
\oddsidemargin 0pt
\evensidemargin \oddsidemargin
\marginparwidth 0.5in
\textwidth 6.5in

\parindent 0in
\parskip 1.5ex
%\renewcommand{\baselinestretch}{1.25}

\begin{document}

\lecture{1 --- September 7, 2004}{Fall 2004}{Prof.\ Victor Ka\v{c}}{Patrick Lam}

\begin{defn}
(a) An \emph{algebra} is a vector space over a field $\F$, endowed
with a multiplication $ab$, which is bilinear:

\begin{eqnarray*}
a(\lambda b+\mu c) &=& \lambda ab + \mu ac \\
(\lambda b + \mu c) a &=& \lambda ba + \mu ca
\end{eqnarray*}
An altebra is associative if $(ab)c = a(bc)$.

(b) A \emph{Lie algebra} is an algebra $\g$ with product $[a, b]$, called
the \emph{bracket} of $a$ and $b$, subject to two axioms:
\begin{itemize}
\item skew commutativity: $[a,a] = 0$
\item Jacobi identity: $[a, [b, c]] + [b, [c, a]] + [c, [a,b]] = 0$.
\end{itemize}

\begin{center}
% PSTricks TeX macro
% Title: l1-diag.dia
% Creator: Dia v0.94
% CreationDate: Mon Sep 20 13:20:15 2004
% For: plam
% \usepackage{pstricks}
% The following commands are not supported in PSTricks at present
% We define them conditionally, so when they are implemented,
% this pstricks file will use them.
\ifx\setlinejoinmode\undefined
  \newcommand{\setlinejoinmode}[1]{}
\fi
\ifx\setlinecaps\undefined
  \newcommand{\setlinecaps}[1]{}
\fi
% This way define your own fonts mapping (for example with ifthen)
\ifx\setfont\undefined
  \newcommand{\setfont}[2]{}
\fi
\pspicture(5.550000,-7.350000)(9.088962,-3.350000)
\scalebox{1.000000 -1.000000}{
\newrgbcolor{dialinecolor}{0.000000 0.000000 0.000000}
\psset{linecolor=dialinecolor}
\newrgbcolor{diafillcolor}{1.000000 1.000000 1.000000}
\psset{fillcolor=diafillcolor}
\newrgbcolor{dialinecolor}{1.000000 1.000000 1.000000}
\psset{linecolor=dialinecolor}
\psellipse*(7.000000,5.450000)(1.000000,1.000000)
\psset{linewidth=0.050000}
\psset{linestyle=solid}
\psset{linestyle=solid}
\newrgbcolor{dialinecolor}{0.000000 0.000000 0.000000}
\psset{linecolor=dialinecolor}
\psellipse(7.000000,5.450000)(1.000000,1.000000)
\setfont{Helvetica}{0.800000}
\newrgbcolor{dialinecolor}{0.000000 0.000000 0.000000}
\psset{linecolor=dialinecolor}
\rput[l](6.750000,4.000000){\scalebox{1 -1}{a}}
\setfont{Helvetica}{0.800000}
\newrgbcolor{dialinecolor}{0.000000 0.000000 0.000000}
\psset{linecolor=dialinecolor}
\rput[l](8.175000,6.975000){\scalebox{1 -1}{b}}
\setfont{Helvetica}{0.800000}
\newrgbcolor{dialinecolor}{0.000000 0.000000 0.000000}
\psset{linecolor=dialinecolor}
\rput[l](5.550000,7.000000){\scalebox{1 -1}{c}}
\psset{linewidth=0.050000}
\psset{linestyle=solid}
\psset{linestyle=solid}
\setlinecaps{0}
\newrgbcolor{dialinecolor}{0.000000 0.000000 0.000000}
\psset{linecolor=dialinecolor}
\psline(6.950000,4.253750)(6.950000,4.778750)
\psset{linewidth=0.050000}
\psset{linestyle=solid}
\psset{linestyle=solid}
\setlinecaps{0}
\newrgbcolor{dialinecolor}{0.000000 0.000000 0.000000}
\psset{linecolor=dialinecolor}
\psline(7.600000,5.903750)(8.000000,6.303750)
\psset{linewidth=0.050000}
\psset{linestyle=solid}
\psset{linestyle=solid}
\setlinecaps{0}
\newrgbcolor{dialinecolor}{0.000000 0.000000 0.000000}
\psset{linecolor=dialinecolor}
\psline(6.400000,6.003750)(6.050000,6.353750)
\psset{linewidth=0.060000}
\psset{linestyle=solid}
\psset{linestyle=solid}
\setlinecaps{0}
\newrgbcolor{dialinecolor}{0.000000 0.000000 0.000000}
\psset{linecolor=dialinecolor}
\psclip{\pswedge[linestyle=none,fillstyle=none](5.429375,6.587915){4.884727}{316.263492}{336.748666}}
\psellipse(5.429375,6.587915)(3.454024,3.454024)
\endpsclip
\psset{linewidth=0.060000}
\psset{linestyle=solid}
\setlinejoinmode{0}
\setlinecaps{0}
\newrgbcolor{dialinecolor}{0.000000 0.000000 0.000000}
\psset{linecolor=dialinecolor}
\psline(8.216726,4.906880)(8.626435,5.287193)(8.684860,4.731238)
}\endpspicture\end{center}
\end{defn}

\paragraph{Remark.} In a Lie algebra, one has $[b, a] = -[a,b]$.

\paragraph{Proof.} $0 = [a+b, a+b] = [a, b] + [b, a] + \cancel{[a, a]} + \cancel{[b, b]}$ \hfill $\Box$  

\paragraph{Examples.}
\begin{enumerate}
\item $\g$ a vector space with bracket $[a,b] = 0$.  This is called an 
\emph{abelian Lie algebra}.
\item $\R^3$ with vector multiplication $\times$ (cross product).
\item If $A$ is an associative algebra, then $[a,b] = ab - ba$ satisfies the
two identities.  This Lie algebra is denoted by $A_-$.

\sk\noindent
{\bf Exercise 1.1}.  Check the Jacobi identity on $[a, b] = ab-ba$.  Moreover, this is
true if $A$ is only quasi-associative, \emph{i.e.} $(ab)c-a(bc)$ is symmetric in $a, b$ $(= (ba)c-b(ac))$.

\paragraph{Solution.} First, we show that associativity implies quasi-associativity.
Let $(ab)c = a(bc)$.  Then $(ab)c - a(bc) = 0 = (ba)c - b(ac)$.
Hence we only need to show that $[a,b]$ satisfies the Jacobi
identity if $A$ is quasi-associative.

Here are some consequences of quasi-associativity.
\begin{eqnarray*}
(ab)c-a(bc)-(ba)c+b(ac) &=& 0 \\
(cb)a-c(ba)-(bc)a+b(ca) &=& 0 \\
(ac)b-a(cb)-(ca)b+c(ab) &=& 0 \\
(ba)c-b(ac)-(ab)c+a(bc) &=& 0 \\
\end{eqnarray*}

We expand the Jacobi identity, group terms, and apply
quasi-associativity:
\begin{eqnarray*}
&& [a, [b, c]] + [b, [c, a]] + [c, [a, b]] \\
&=& a(bc-cb) - (bc-cb)a + b(ca-ac) - (ca-ac)b + c(ab-ba) - (ab-ba)c\\
&=& \quad [(cb)a-c(ba)-(bc)a+b(ca)] \\
&& +\; [(ac)b-a(cb)-(ca)b+c(ab)] \\
&& +\; [(ba)c-b(ac)-(ab)c+a(bc)]\\
&=& 0 \mbox{ (since all these terms $= 0$ by quasi-associativity)}
\end{eqnarray*}
$\Box$

A special case is $A = \End V$, then $A_- = \gl_V$ is called the 
\emph{general linear Lie algebra}.  In particular, $A = \Mat_n \F$, then
$A_- = \gl_n(\F)$.

\item Any subalgebra of a Lie algebra is a Lie algebra.

Notation: for subsets $M, N$ of $\g$ we denote $[M,N]$ the span of all
commutators $[m, n]$, where $m \in M$ and $n \in N$.  For example,
subspace $\h$ is a subalgebra of $\g$ if $[\h, \h'] \in \h$.

\paragraph{Example.} $\sll_n(\F) = \{ a \in \gl_n(\F) \mid \tr a = 0 \}$

\sk\noindent
{\bf Exercise 1.2}. Show that $\mbox{tr} [a,b] = 0$ when $a, b \in
\gl_n(\F)$.  Also show that if $f:\gl_n(\F) \rightarrow \F$ is a
linear function such that $f([a, b]) = 0, a, b \in \gl_n(\F)$, then
$f(k) = c \cdot \tr k$.

\paragraph{Solution.} We have
\[ \tr [a, b] = \tr ab - \tr ba = \sum_i \sum_j a_{ji} b_{ij} - \sum_i \sum_j b_{ji} a_{ij} = 0 \]

Now, any matrix $e_{ij}, i \neq j$ can be expressed as
a commutator $[a,b]$ where $a, b \in \gl_n(\F)$, because $e_{ij}e_{jj} - e_{jj}e_{ij}
= e_{ij}$.  Hence $f(e_{ij}) = 0$ for all such matrices.  

But $f(e_{ii}) = f(e_{jj}) = c$ for all $i, j$ because
$f(e_{ii})-f(e_{jj}) = f(e_{ii}) - f(e_{jj}) = 
f(e_{ij}e_{ji} - e_{ji}e_{ij}) = 0$.

Any $x \in \gl_n(\F)$ can be split into $x' + x''$ where
$x'$ is the sum of $e_{ij}, i \neq j$, 
and $x''$ is of the form $\sum_i \lambda_i e_{ii}$.  By linearity,
\[f(x) = 0 + f(x'') = c \sum_i \lambda_i = c \cdot \mbox{tr}.\]
$\Box$

\begin{defn}
An \emph{ideal} $\m \subset \g$ is a subspace such that $[\m, \g] \subset \m$.
\end{defn}

\paragraph{Example.} $\sll_n(\F)$ is an ideal of $\gl_n(\F)$ by Exercise 1.2.

\item Factor algebras: If $\g$ is a Lie algebra and $\m$ is an ideal,
then $\g / \m$ is a Lie algebra with bracket $[a+\m, b+\m] = [a, b]+\m$.

\item Direct sum of two (Lie) algebras $\g_1 \oplus \g_2$: $[(a, b), (a_1, b_1)] = ([a, a_1], [b, b_1])$ where $a, a_1 \in \g_1, b, b_1 \in \g_2$.
\end{enumerate}


More examples of subalgebras of $\gl_V$: Let $B$ be a bilinear ($\F$-valued)
form on a vector space $V$ over $\F$, define $o_{V,B} = \{ a \in \gl_V \mid
B(a(u), v) + B(u, a(v)) = 0 \}$.

\sk\noindent
{\bf Exercise 1.3}. Let $B$ be a bilinear $\F$-valued form on a vector space
$V$ over $\F$.  Define
\[ o_{V,B} = \{a \in \gl_V \ \mid \ B(a(u), v) + B(u, a(v)) = 0 \]
Check that this is a subalgebra of the Lie algebra $\gl_V$.

\paragraph{Solution.} To show that $o_{V,B}$ is a subalgebra (it is clearly a subspace), 
we need only show that
$o_{V,B}$ is closed under the bracket.  Let $x, y \in o_{V,B}$.

Consider
\[ B(x(y(u)), v) = -B(y(u), x(v)) = B(u, y(x(v))) \]

But since $B$ is bilinear and $yx = [y,x]-xy$, $B(u, y(x(v))) = B(u,
[x, y] - xy) = -B(u, x(y(v))) + B(u, [x, y])$ where $B(u, [x, y])$ is
0 from above, implying that $B(x(y(u)), v) = -B(u, x(y(v)))$, giving
closure:
\[ B(x(y(u)), v) + B(u, x(y(v))) = 0 \]
$\Box$

Important special cases: $\dim V < \infty$, $B$ is non-degenerate
(\emph{i.e.} det of the matrix of $B$ in some basis is non-zero).
\begin{itemize}
\item case 1: $B$ is symmetric.  $B(a,b) = B(b,a)$, then $o_{V,B}$ 
is called the orthogonal Lie algebra, notation $so_{V,B}$.
\item case 2: $B$ is skew-symmetric.  $B(a,b) = -B(b,a)$, then $o_{V,B}$ 
is called the symplectic Lie algebra, notation $sp_{V,B}$.
\end{itemize}

\sk\noindent
{\bf Exercise 1.4}. Suppose $\dim V = n$, choose a basis of $V$, let
$so_{V, B}$ and $sp_{V, B} \subset \gl_n$.  Let $B$
be the matrix of the bilinear form.  Show
\begin{eqnarray*}
so_{V,B} &=& \{ a \in \gl_n(\F) \ \mid \ a^T B + Ba = 0 \}\\
sp_{V,B} &=& \{ a \in \gl_n(\F) \ \mid \ a^T B + Ba = 0 \}
\end{eqnarray*}

\paragraph{Solution.} 
Recall that $\gl_n$ is associative, and that $so_{V,B}$ is the
set of $a \in A$ such that $B(u, a(v)) + B(a(u), v) = 0$ and $B$ symmetric;
similarly, $sp_{V,B}$ is the corresponding set when $B$ is skew-symmetric.
For $so_{V,B}$, we expand the definition of the bilinear form to get
\[ u^T B a(v) + (a(u))^T B v = 0 \] 
and expressing $a$ as matrix multiplication,
\[ u^T (Ba) v + (au)^T Bv = 
u^T (Ba) v + u^T a^TB v =
0 \] 
which is equivalent to the matrix condition
\[ a^T B + Ba = 0 \]
when $B$ is a symmetric matrix; similarly, for a skew-symmetric $B$,
we expand to get
\[ u^T B a(v) + (a(u))^T B v = 0 \] 
and expressing $a$ as matrix multiplication,
\[ u^T (Ba) v + (au)^T Bv = 
u^T (Ba) v + u^T a^TB v =
0 \] 
which is equivalent to the matrix condition
\[ a^T B + Ba = 0. \]
$\Box$

\begin{defn}
The \emph{derived algebra} of $\g$ is $[\g, \g]$.  Obviously, this
is an ideal, and hence a subalgebra.
\end{defn}

We now classify Lie algebras in dimensions 1 and 2.

\paragraph{dim 1.} $\g = \F a, [a, a] = 0$.  Only the abelian one.

\paragraph{dim 2.} $\g = \F a + \F b, [\g, \g] = \F[a,b]$.  
case 1.  $[a,b] = 0$.  Then $\g$ abelian.  case 2. $[a,b] = c \neq 0$.
So $\g' = \F c$.  Take $d \not\in \g'$ such that $d \neq 0$.  Then 
$[d, c] = 
\alpha c$, since $\F c$ is an ideal, and $\alpha \neq 0$, since $\g$ nonabelian.
Replacing $d$ by $\frac{1}{\alpha}d$, we get $[d, c] = c$.  Thus we have
a unique non-abelian Lie algebra, $[a, b] = b$.

\sk\noindent
The two most important ways to construct Lie algebras:
\begin{enumerate}
\item as a subalgebra (of $\gl_n$);
\item by structure constants: choose a basis $e_1, \ldots, e_n$ of $\g$;
then $[e_i, e_j] = \sum_{k=1}^n c_{ij}^k e_k$.  The scalars $c_{ij}^k$
are called \emph{structure constants}.  Of course, $c_{ii}^k = 0$ and
$c_{ij}^k = -c_{ij}^k$ by skew-commutativity and a quadratic equation 
which is the Jacobi identity.
\end{enumerate}

\begin{remark}
The non-abelian 2-dimensional Lie algebra is 
\[ \left\{ \left( \begin{array}{cc} \alpha & \beta \\ 0 & 0 \end{array} \right) \right\} \subset \gl_2(\F) \]
If $a = \left( \begin{array}{cc} 1 & 0 \\ 0 & 0 \end{array} \right)$ and
$b = \left( \begin{array}{cc} 0 & 1 \\ 0 & 0 \end{array} \right)$, then
$[a,b] = b$.
\end{remark}

\begin{example}
The Heisenberg Lie algebra $\He_n$ has basis $p_i, q_i (i = 1, \ldots, n), c$.
($\dim \He_n = 2n+1$) where $[p_i, q_j] = \delta_{ij} c, [c, p_i] = 0, 
[c, q_i] = 0, [p_i, p_j] = 0, [q_i, q_j] = 0$.  Jacobi trivially holds.
Realization by operators: $p_i = \frac{\partial}{\partial x},
q_i = x_i, c = 1$ on $\C[x_1, \ldots, x_n]$.
\end{example}

The first important meaning of the Jacobi identity: Rewrite it as follows:
\begin{equation}
\label{jac1}
[a, [b, c]] = [[a, b], c] + [b, [a, c]]. 
\end{equation}

\begin{defn}
For any algebra $A$, an endomorphism $D$ is called a \emph{derivation}
if the Leibniz rule holds, \emph{i.e.} 
\[D(ab) = (Da)b + a(Db).\]
\end{defn}

\begin{defn}
Given an element $a \in \g$, define the operator $\ad a$ (adjoint) on $\g$
by: 
\[ (\ad a) b = [a, b]. \]
\end{defn}
Equation \ref{jac1} means that $\ad a$ is a derivation of the Lie algebra $\g$.
It is called an \emph{inner derivation}.

Notation: Given an algebra $A$, denote by $\Der A (\subset \End A)$
the space of derivations of $A$.

\sk\noindent
{\bf Exercise 1.5}. a) $\Der A$ is closed under the bracket in $\End A$ i.e.
bracket of two derivations is a derivation, or $\Der A$ is a
subalgebra of $\gl_A$. (b) If $A = \g$ is a Lie
algebra, then $[D, \ad a] = \ad(D(a))$ for any derivation
$F \in \Der \g$ and $a \in \g$.  Hence
inner derivations form an ideal of the Lie algebra $\Der \g$.

\paragraph{Solution.} 
a) Let $D_1, D_2$ be derivations.  Consider (by parts) $[D_1, D_2]$:
\begin{eqnarray*}
D_1D_2(ab) &=& D_1((D_2a)b + a(D_2b)) \\
       &=& D_1((D_2a)b) + D_1(a(D_2b)) \\
       &=& (D_1(D_2a))b+(D_2a)(D_1b)+(D_1a)(D_2b) + a(D_1D_2b) \\
D_2D_1(ab) &=& (D_2(D_1a))b+(D_1a)(D_2b)+(D_2a)(D_1b) + a(D_2D_1b) \\
D_1D_2(ab) - D_2D_1(ab) &=& (D_1D_2a-D_2D_1a)b + a(D_1D_2b-D_2D_1b) \\
        &=& ((D_1D_2-D_2D_1)a)b + a((D_1D_2-D_2D_1)b) 
\end{eqnarray*}
showing that the bracket is a derivation.

\sk\noindent
b) Consider 
\begin{eqnarray*}
[D, \ad a] &=& (D(\ad a))b - ((\ad a)D)b \\
 &=& D[a,b] - [a, D(b)]\\
 &=& D(ab) - D(ba) - aD(b) + D(b)a\\
 &=& D(a)b + \cancel{aD(b)} - \cancel{D(b)a} - bD(a) - \cancel{aD(b)} + \cancel{D(b)a} \\
 &=& [D(a), b]\\
 &=& \ad D(a)
\end{eqnarray*}
which shows that inner derivations form an ideal of the Lie algebra
$\Der \g$. $\Box$

\end{document}



