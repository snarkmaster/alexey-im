\documentclass[11pt]{article}
\usepackage{cancel}
\usepackage{amsmath}
\usepackage{amsthm}
\usepackage{amssymb}
\usepackage{epsfig}
\usepackage{pstricks}

\newcommand{\handout}[5]{
  \noindent
  \begin{center}
  \framebox{
    \vbox{
      \hbox to 5.78in { {\bf 18.745 Introduction to Lie Algebras } \hfill #2 }
      \vspace{4mm}
      \hbox to 5.78in { {\Large \hfill #5  \hfill} }
      \vspace{2mm}
      \hbox to 5.78in { {\em #3 \hfill #4} }
    }
  }
  \end{center}
  \vspace*{4mm}
}

\newcommand{\lecture}[4]{\handout{#1}{#2}{#3}{Scribe: #4}{Lecture #1}}


\newcommand{\F}{\mathbb{F}}
\newcommand{\Z}{\mathbb{Z}}
\newcommand{\Q}{\mathbb{Q}}

\newcommand{\sll}{\mbox{sl}}
\newcommand{\gl}{\mbox{gl}}
\newcommand{\so}{\mbox{so}}
\newcommand{\spl}{\mbox{sp}}
\newcommand{\tr}{\mbox{tr}}
\newcommand{\diag}{\mbox{diag}}
\newcommand{\ad}{\mbox{ad}}
\newcommand{\charr}{\mbox{char}}
\newcommand{\centerr}{\mbox{center}}

\newcommand{\g}{\mathfrak{g}}
\newcommand{\gb}{\overline{\g}}
\newcommand{\h}{\mathfrak{h}}
\newcommand{\hb}{\overline{\h}}
\newcommand{\m}{\mathfrak{m}}
\newcommand{\p}{\mathfrak{p}}
\newcommand{\D}{\mathfrak{D}}
\newcommand{\He}{\mathcal{H}}

\newcommand{\sk}{\vspace*{1em}}

\newtheorem{defn}{Definition}
\newtheorem{remark}{Remark}
\newtheorem{example}{Example}
\newtheorem{thm}{Theorem}
\newtheorem{lem}{Lemma}
\newtheorem{cor}{Corollary}

% 1-inch margins, from fullpage.sty by H.Partl, Version 2, Dec. 15, 1988.
\topmargin 0pt
\advance \topmargin by -\headheight
\advance \topmargin by -\headsep
\textheight 8.9in
\oddsidemargin 0pt
\evensidemargin \oddsidemargin
\marginparwidth 0.5in
\textwidth 6.5in

\parindent 0in
\parskip 1.5ex
%\renewcommand{\baselinestretch}{1.25}

\begin{document}

\lecture{10 --- October 12, 2004}{Fall 2004}{Prof.\ Victor Ka\v{c}}{Steven 
Sivek}

Recall that a \emph{trace form} on a Lie algebra $\g$, given a 
representation $\pi$ of $\g$ in a finite dimensional vector space $V$, is 
defined as $(a,b)_V = \tr_V(\pi(a)\pi(b))$.

\begin{defn}
The \emph{Killing form} on a finite-dimensional Lie algebra $\g$ is the 
trace form $K(a,b) = \tr_{\g}((\ad\ a)(\ad\ b))$ in the adjoint 
representation of $\g$.
\end{defn}

\paragraph{Exercise 10.1.} Show that the trace form for the defining 
representation of $\gl_n(\F)$, $\sll_n(\F)$, $\so_n(\F)$, $\spl_n(\F)$ 
is nondegenerate.

\paragraph{Solution.}
Let $e_{xy}$ denote the $n$x$n$ matrix with the element in row $x$, column 
$y$ equal to 1 and all other elements 0.  We note that $\tr(e_{ij}e_{kl}) 
= \tr(\delta_{jk}e_{il}) = \delta_{jk}\delta_{il}$.  To show that this 
form is nondegenerate for a Lie algebra $\g \subset \gl_n(\F)$, it 
suffices to show for any $a \in \g$ that $\tr(ab) \not= 0$ for some $b \in 
\g$, so this is how we will proceed.  (We will assume that $\charr(\F) = 
0$ for simplicity, since e.g. the trace form for $\sll_n(\F)$ is 
degenerate when $\charr(F)\ |\ n$.)

$\g = \gl_n(\F)$: Take nonzero $a = (a_{ij}) \in \g$, and pick $x,y$ such 
that $a_{xy} \not= 0$.  Then $\tr(a e_{yx}) = a_{xy} \not= 0$.

$\g = \sll_n(\F)$:  Choose nonzero $a = (a_{ij}) \in \g$.  If $a$ is not 
a diagonal matrix, then we may pick $x \not= y$ such that $a_{xy} \not= 
0$, and as before we have $\tr(a e_{yx}) \not= 0$.  Otherwise, let $a = 
\diag(b_1, \dots, b_n)$.  Then $a \not= \alpha I_n$ for any 
$\alpha$, since otherwise we would have either $a=0$ or $\tr(a) = n\alpha$ 
with $n,\alpha \not= 0$.  It follows that for some $k < n$ we must have 
$b_k \not= b_{k+1}$; we calculate that $\tr(a (e_{kk} - e_{k+1}{k+1})) = 
b_k-b_{k+1} \not= 0$.

$\g = \so_n(\F)$: Pick a basis for $\F^n$ such that $\so_n(\F)$ is the 
algebra of skew-symmetric matrices, i.e. $\so_n(\F) = \{ a \in \gl_n(\F)\ 
|\ a^T = -a \}$.  Then $\g$ has as a basis $\mathcal{B} = \{ 
e_{ij}-e_{ji}\ |\ i<j \}$.  We may easily calculate that for any $b_1, b_2 
\in \mathcal{B}$, we have $\tr(b_1b_2) = -2$ if $b_1 = b_2$ and 
$\tr(b_1b_2) = 0$ otherwise.  Then take nonzero $a = (a_{ij}) \in \g$, and 
pick $x < y$ such that $a_{xy} \not= 0$; we calculate that $\tr(a 
(e_{xy}-e_{yx})) = -2a_{xy} \not= 0$.

$\g = \spl_n(\F)$: Note that $n$ mus be even, and write $n = 2m$.  If we 
pick as a representative skew-symmetric form the matrix {\tiny $\left( 
\begin{array}{cc} 0 & I_m \\ -I_m & 0 \end{array} \right)$}, then $\g$ 
consists of matrices expressible in block form as $a = {\tiny \left( 
\begin{array}{cc} A & B \\ C & -A^T \end{array} \right) }$, where each 
block is $m$x$m$, and $B$ and $C$ are symmetric.  Take $a \in \spl_n(\F)$ 
of this form; if $B = C = 0$, then $\tr(a \cdot {\tiny \left( 
\begin{array}{cc} D & 0 \\ 0 & -D^T \end{array} \right)} ) = 2\tr(AD)$, 
and since $A \not= 0$ and the trace form on $\gl_m(\F)$ is nondegenerate, 
we can find $D \in \gl_m(\F)$ such that $2\tr(AD) \not= 0$. Otherwise, 
assume without loss of generality that $B \not= 0$.  Then $\tr(a \cdot 
{\tiny \left( \begin{array}{cc} 0 & 0 \\ E & 0 \end{array} \right) }) = 
\tr(BE)$, so we wish to find a symmetric matrix $E$ such that $\tr(BE) 
\not= 0$.  Take the basis $\mathcal{B} = \{e_{ii}\}_{i=1}^m \cup \{e_{ij} 
+ e_{ji}\ |\ i<j\}$ of the $m$x$m$ symmetric matrices; then we can compute 
that for $b_1, b_2 \in \mathcal{B}$, $\tr(b_1b_2) \not= 0$ if and only if 
$b_1 = b_2$.  Write $B = \sum_{b_i \in \mathcal{B}} c_ib_i$ for some 
constants $c_i \in \F$, and fix $k$ such that $c_k \not= 0$; then $\tr(B 
\cdot b_k) = c_k\tr(b_k^2) \not= 0$, as desired.

\paragraph{Exercise 10.2.}  Show that the Killing form on $\sll_n(\F)$, 
$\charr(\F) \not= 2$, is nondegenerate if and only if $\charr(\F) \not|\ 
n$.

\paragraph{Solution.}
Assume that $K$ is nondegenerate but that $\charr(\F)\ |\ n$.  Then 
$\tr(I_n) = n = 0$ in $\F$, so $I_n \in \sll_n(\F)$.  But $\ad\ I_n = 0$,
so for all $g \in \sll_n(\F)$ we have $K(I_n, g) = \tr_\g (\ad\ I_n)(\ad\ 
g) = \tr_\g 0 = 0$, contradicting the nondegeneracy of $K$. Thus if $K$ is 
nondegenerate, we must have $\charr(\F)\ \not|\ n$.

Now suppose that $\charr(\F) \not|\ n$.  Let $\g = \sll_n(\F)$ have basis 
$\mathcal{B} = \{ e_{xy}\ |\ x \not= y \} \cup \{ e_{ii}-e_{nn}\ |\ i<n 
\}$; then for any $a,b \in g$, $K(a,b) = \tr_\g(\ad\ a)(\ad\ b)$ is the 
sum of the $b_i$-components of $(\ad\ a)(\ad\ b)b_i$ over all $b_i \in 
\mathcal(B)$.  Let $a = (a_{xy}) \in \g$.  We first calculate 
$K(a, e_{xy})$ for $x \not= y$ by computing that $[a,[e_{xy},e_{pq}]]$ has 
$e_{pq}$-component $\delta_{py}a_{px} + \delta_{xq}a_{yq} = 
(\delta_{py}+\delta{xq})a_{yx}$, and that $[a,[e_{xy},e_{ii}-e_{nn}]]$ has 
$(e_{ii}-e_{nn})$-component $(\delta_{iy}(1+\delta_{nx}) + 
\delta_{ix}(1+\delta_{ny}))a_{yx}$, so that
\begin{eqnarray*}
K(a,e_{xy})
  &=& \sum_{p \not= q} \sum_{q=1}^n (\delta_{py}+\delta_{xq}) a_{yx}
       + \sum_{i=1}^{n-1} (\delta_{iy}(1+\delta_{nx}) + 
                           \delta_{ix}(1+\delta_{ny}) )a_{yx} \\
  &=& 2(n-1)a_{yx} + (1+\delta_{nx}-\delta_{ny})a_{yx}
                   + (1+\delta_{ny}-\delta_{nx})a_{yx} = 2na_{yx}. 
\end{eqnarray*}
Hence if $a$ is not diagonal, we may pick some $x \not= y$ such that 
$a_{xy} \not= 0$, and then $K(a,e_{yx}) = 2na_{xy} \not= 0$ as long as 
$\charr(\F)\ \not|\ n$.  A similar calculation yields $K(a, e_{ii}-e_{nn}) 
= 2n(a_{ii}-a_{nn})$.  If $a \not= \alpha I_n$ for any $\alpha$, then 
there is some $k < n$ such that $a_{kk} \not= a_{nn}$, and so $K(a, 
e_{kk}-e_{nn}) = 2n(a_{kk}-a_{nn}) \not= 0$.  Otherwise, take $\alpha$ 
such that $a = \alpha I_n$; then $\tr(a) = n\alpha = 0$, so (since 
$\charr(\F)\ \not|\ n$) we must have $\alpha = 0$ and hence $a = 0$.  
Since we can thus find a basis element $b$ for any nonzero $a \in \g$ such 
that $K(a, b) \not= 0$, it follows that $K$ is nondegenerate.  Therefore 
the Killing form on $\g = \sll_n(\F)$ is nondegenerate if and only if 
$\charr(\F)\ \not|\ n$, as desired.

\paragraph{}
Having computed these examples, we now proceed to our first practical 
application of trace forms.  We will need the following lemma:

\begin{lem}[Cartan's lemma]
Let $\g$ be a finite-dimensional Lie algebra over an algebraically closed 
field $\F$ of characteristic 0, and let $\pi$ be a representation of $\g$ 
in a finite-dimensional vector space $V$ over $\F$.  Let $\h$ be a Cartan 
subalgebra of $\g$, yielding generalized root and weight space 
decompositions $\g = \oplus_{\alpha \in \h^*} \g_\alpha$ and $V = 
\oplus_{\lambda \in \h^*} V_\lambda$.  Take $e \in \g_\alpha$, $f \in 
\g_{-\alpha}$, so that if $h = [e,f]$ then $h \in \g_0 = \h$.  Suppose 
that $V_\lambda \not= 0$ for some $\lambda \in \h^*$.  Then $\lambda(h) = 
r \cdot \alpha(h)$, where $r$ is a rational number which depends only on 
$\alpha$ and $\lambda$.
\end{lem}

\begin{proof}
Let $U = \oplus_{n \in \Z} V_{\lambda + n\alpha}$.  Then $U$ is invariant 
with respect to the operators $\pi(e), \pi(f)$, since 
$\pi(\g_{\pm \alpha})V_\mu \subset V_{\pm \alpha+\mu}$.  Hence we can 
restrict $\pi$ to $U$ and compute that $\tr_U(\pi(h)) = \tr_U(\pi([e,f])) 
= 0$.  On the other hand, each $V_{\lambda+n\alpha}$ is invariant with 
respect to $h \in \g_0$, so we can compute $tr_U \pi(h) = \sum_n 
\tr_{V_{\lambda+n\alpha}} \pi(h)$.  But $\pi(h)|_{V_\mu}$ has an upper 
triangular matrix with all diagonal entries $\mu$, so we get
\begin{eqnarray*}
0 = \tr_U(\pi(h)) & = & \sum_n \dim(V_{\lambda + n\alpha})
                        (\lambda+n\alpha)(h) \\
& = & \lambda(h) \sum_n \dim(V_{\lambda+n\alpha})
      + \alpha(h) \sum_n (n \dim(V_{\lambda+n\alpha})).
\end{eqnarray*}
Let the two sums be $P$ and $Q$, respectively; they are clearly both 
integers, since each of the summands is, and $P = \dim(U) > 0$, so 
$\lambda(h) = \frac{-Q}{P} \alpha(h)$, where the ratio $\frac{-Q}{P}$ 
depends only on $\lambda$ and $\alpha$, as desired.
\end{proof}

We may use this to describe solvable subalgebras of $\gl_V$ in several 
ways in terms of trace forms.

\begin{thm}[Cartan's criterion]
Let $\g$ be a subalgebra of $\gl_V$ for $V$ a finite-dimensional vector 
space over a an algebraically closed field $\F$ of characteristic 0.  Then 
the following are equivalent:
\begin{enumerate}
\item $(\g, [\g,\g])_V = 0$.
\item $(a,a)_V = 0$ for any $a \in [\g,\g]$.
\item $\g$ is a solvable Lie algebra.
\end{enumerate}
\begin{proof}
$(1) \Rightarrow (2)$:  Take $a \in [\g,\g]$, and write $a=[b,c]$.  Then 
since $(a,[b,c])_V = 0$, we have $(a,a)_V=0$.

$(2) \Rightarrow (3)$: Suppose that $\g$ is not solvable.  Then the 
derived series of $\g$ stabilizes to some nonzero subalgebra $\p = 
\g^{(N)} = \g^{(N+1)} = \cdots$, so $[\p, \p] = \p$.  In particular, since 
$\p \subset \g$, we have $(a,a)_V = 0$ for any $a \in [\p,\p] = \p$ by our 
assumption.  Let $\h$ be a Cartan subalgebra of $\p$, and consider the 
root and weight space decompositions
\begin{eqnarray*}
\p = \bigoplus_{\alpha \in \h^*} \p_\alpha,\ 
V = \bigoplus_{\lambda \in \h^*} V_\lambda.
\end{eqnarray*}
Since $[\p_\alpha, \p_\beta] \subset \p_{\alpha+\beta}$ and $[\p,\p] = 
\p$, we conclude that $\h = \p_0 = \sum_\alpha [\p_\alpha, 
\p_{-\alpha}]$; that is, $\h$ is a span of elements of the form 
$h_{\alpha,i} = [e_{\alpha,i}, f_{\alpha,i}]$, where $e_{\alpha,i} \in 
\p_\alpha$ and $f_{\alpha,i} \in \p_{-\alpha}$.

Suppose that $V_\lambda \not= 0$ for some fixed $\lambda \in \h^*$.  By 
Cartan's lemma, we can write $\lambda(h_{\alpha,i}) = r_{\alpha,\lambda} 
\alpha(h_{\alpha,i})$ for all $\alpha$ and $i$, where $r_{\alpha,\lambda} 
\in \Q$.  Since we are assuming (2), $(h_{\alpha,i}, h_{\alpha,i})_V = 0$.  
But we compute that $(h_{\alpha,i}, h_{\alpha,i})_V = \sum_\lambda 
\lambda(h_{\alpha,i})^2\dim(V_\lambda)$ by considering the restriction of 
$h_{\alpha,i}$ to each subspace $V_\lambda$ as in the proof of Cartan's 
lemma; this is then equal to $\sum_\lambda r_{\alpha,\lambda}^2 
\alpha(h_{\alpha,i})^2 \dim(V_\lambda)$ by the lemma, so 
$\alpha(h_{\alpha,i})^2 \left( \sum_\lambda r_{\alpha,\lambda}^2 
\dim(V_\lambda) \right) = 0$.  Hence, since all the $r_{\alpha,\lambda}$ 
are rational numbers, either $\alpha(h_{\alpha,i}) = 0$, or 
$r_{\alpha,\lambda} = 0$ whenever $V_\lambda \not= 0$, and in either case 
we have $\lambda(h_{\alpha,i}) = r_{\alpha,\lambda} \alpha(h_{\alpha,i}) = 
0$ whenever $V_\lambda \not= 0$.

For any $\lambda \in \h^*$ such that $V_\lambda \not= 0$, since the 
elements $\{h_{\alpha,i}\}$ span $\h$, and $\lambda = 0$ on all of these 
elements, we conclude that $\lambda \equiv 0$ on $\h$.  Therefore the 
weight space decomposition of $V$ is $V = V_0$.  But for any $\alpha \not= 
0$ we have $\p_\alpha V_0 \subset V_\alpha = 0$, so $\p_\alpha = 0$; hence 
$\p = \p_0 = \h$.  But if $\p = \h$ then $\p$ is nilpotent, and so $[\p, 
\p]$ is a proper subset of $\p$.  This is a contradiction, so $\g$ must in 
fact be solvable.

$(3) \Rightarrow (1)$: By a corollary to Lie's theorem, we may pick a 
basis of $V$ such that all of the matrices in $\g$ are upper triangular,
hence all matrices in $[\g,\g]$ are strictly upper triangular.  Then 
$\tr_V(ab) = 0$ for any $a \in \g$ and $b \in [\g,\g]$, since $ab$ is 
strictly upper triangular, and so $(\g,[\g,\g])_V = 0$.
\end{proof}
\end{thm}

This theorem leads almost immediately to a characterization of solvable 
Lie algebras, as follows:

\begin{cor}
A finite dimensional Lie algebra $\g$ over an algebraically closed 
field of characteristic 0 is solvable if and only if $K(\g,[\g,\g]) = 0$.
\begin{proof}
We know that $\g$ is solvable if and only if $\g/\centerr(\g)$ and 
$\centerr(\g)$ are both solvable, hence if and only if $\g/\centerr(\g)$ 
is solvable. Consider the adjoint representation $\ad: \g \rightarrow 
\gl_{\g}$.  Since $\ker(\ad) = \centerr(\g)$, this proof reduces to 
consideration of the subalgebra $\ad\ \g \subset \gl_\g$; we now apply the 
fact that conditions (1) and (3) of Cartan's criterion are equivalent, and 
we are done.
\end{proof}
\end{cor}

Up until this point, we have restricted many of our results to Lie 
algebras over algebraically closed fields $\F$ of characteristic zero, 
since root space and weight space decompositions exist for such algebras.  
Requiring $\F$ to be algebraically closed is not always necessary, as we 
shall see:

\begin{remark}
Let $\F$ be a field of characteristic zero which is not necessarily 
algebraically closed, and let $\g$ be a Lie algebra over $\F$.  Then the 
following are true:
\begin{enumerate}
\item Cartan's criterion and the corollary which followed it are both true 
over $\F$.
\item $\h_0^a$ is a Cartan subalgebra of $\g$ if $a \in \g$ is regular.
\item $[\g,\g]$ is nilpotent if $\g$ is finite-dimensional and solvable.
\end{enumerate}
\end{remark}
In order to show this, we will introduce some notation: let 
$\overline{\F}$ be the algebraic closure of $\F$, and let $\gb = 
\overline{\F} \otimes_\F \g$.

\paragraph{Exercise 10.3.}
(a) Show that $\g$ is solvable (resp. nilpotent, abelian) if and only if 
$\gb$ is, and that $\overline{[\g,\g]} = [\gb, \gb]$.
(b) Prove the above remark for $\F$ not algebraically closed.

\paragraph{Solution.}
Given two Lie algebras $\g$ and $\h$ over $\F$, we claim that $[\gb, \hb] 
= \overline{[\g,\h]}$.  Clearly since $\g \subset \gb$ and $\h \subset 
\hb$, we have $\overline{[\g,\h]} \subset \overline{[\gb,\hb]} = [\gb, 
\hb]$.  Given bases $\{ g_\alpha \}_{\alpha \in I}$ of $\g$ and $\{ 
h_\beta \}_{\beta \in J}$ of $\h$, these same bases generate $\gb$ and 
$\hb$ when combined over $\overline{\F}$ rather than $\F$.  Hence the set 
$\{[g_\alpha, h_\beta]\}_{\alpha \in I, \beta \in J}$ generates 
$[\gb,\hb]$ over $\overline{\F}$.  But this set also generates $[\g,\h]$ 
with coefficients in $\F$, so when we extend this to $\overline{F}$ we get 
$[\gb,\hb] \subset \overline{[\g,\h]}$.  Thus $[\gb, \hb] = 
\overline{[\g,\h]}$.

Part (a) now follows easily, since we note that $\gb^i = \overline{\g^i}$ 
and $\gb^{(i)} = \overline{\g^{(i)}}$ for all $i$.  Then $\g$ is solvable 
if and only if $\g^{(n)} = 0$ for some $n$; but this is equivalent to 
$\overline{\g^{(n)}} = 0$, or $\gb^{(n)} = 0$, and so $\g$ is solvable iff 
$\gb$ is.  The same is true of nilpotency, as we see by replacing 
$g^{(n)}$ with $g^n$ in this argument, and abelianness, which comes from 
the solvability argument with $n = 1$.  In particular, letting $\h = \g$ 
above yields $\overline{[\g,\g]} = [\gb,\gb]$.

For part (b), we first look at Cartan's criterion.  Since $\g$ is solvable 
if and only if $\gb$ is, we have the equivalent conditions (1) $(\gb, 
[\gb,\gb]_V = 0$; (2) $(a,a)_V = 0$ for all $a \in [\gb,\gb]$; and (3) 
$\g$ is solvable.  But $(\gb, [\gb,\gb])_V = (\gb, \overline{[\g,\g]})_V = 
\overline{(\g,[\g,\g])_V}$, so $(\g,[\g,\g])_V = 0$ if and only if 
$(\gb,[\gb,\gb])_V = 0$.  Furthermore, $(a,a)_V = 0$ for all $a \in 
[\gb,\gb] = \overline{[\g,\g]}$ iff it does for all $a \in [\g,\g]$, since 
$[\g,\g]$ and $\overline{[\g,\g]}$ share the same basis over different 
fields and thus $(a,a)_V = 0$ for all $a$ in one iff it does for all $a$ 
in the other.  Therefore Cartan's criterion holds for $\g$ given that it 
does for $\gb$; the corollary's proof over $\g$ is identical to its 
original proof, since it only needed algebraic closure to satisfy Cartan's 
criterion.

Next, we consider the proof that $\h_0^a$ is a Cartan subalgebra if $a$ is 
regular.  If $a \in \g$ is regular, it is regular in $\gb$, so we know 
that $\overline{\h_0^a}$ is a Cartan subalgebra of $\gb$, or 
$\overline{\h_0^a} = N_{\gb}(\overline{\h_0^a})$.  Since $[b, 
\overline{\h_0^a}] \subset \overline{\h_0^a}$ if and only if $[b, \h_0^a] 
\subset \h_0^a$ for any $b \in \g$, �we conclude that $N_{\g}(\h_0^a) = 
\h_0^a$, and so $\h_0^a$ is a Cartan subalgebra of $\g$.

Finally, assume $\g$ is finite dimensional and solvable.  Then $\gb$ is 
finite dimensional and solvable, so we know that $[\gb,\gb]$ is nilpotent.  
But $[\gb,\gb] = \overline{[\g,\g]}$, so $\overline{[\g,\g]}$ is 
nilpotent, hence $[\g,\g]$ is nilpotent.

\paragraph{}
\begin{remark}
The basic properties of a trace form are that it is symmetric (i.e. 
$(a,b)_V = (b,a)_V$) and invariant (i.e. $([a,b],c)_V = (a,[b,c])_V$).  
The basic results on trace forms (like Cartan's criterion) fail, however, 
if we assume only that the bilinear forms involved are symmetric and 
invariant.
\end{remark}

We can construct an example of this as follows:
\paragraph{Exercise 10.4.}
Consider the 4-dimensional Lie algebra $\D = \F p + \F q + \F c + \F d$, 
where $[p,q] = c$, $c$ is central, $[d,p] = p$, and $[d,q] = d$.  
Construct a nondegenerate symmetric invariant bilinear form on $\D$.

\paragraph{Solution.}
Define a bilinear form $B(x,y)$ on $\D$ by the values $B(p,q) = B(q,p) = 
1$; $B(c,d) = B(d,c) = 1$; $B = 0$ on all other pairs of basis 
elements; and all other values of $B$ follow by symmetry and bilinearity.  
Then $B$ is nondegenerate, since it has matrix
{\tiny $\left( \begin{array}{cccc} 0&1&0&0 \\ 1&0&0&0 \\
0&0&0&1 \\ 0&0&1&0 \end{array} \right)$ }
with respect to the basis $\{p,q,c,d\}$ and this matrix is invertible.  
Invariance is also easy to check, so we are done.

\paragraph{}
The algebra $\D$ is solvable, since $[\D,\D] = \He_1$ is solvable, so by 
Cartan's criterion, any trace form on $\D$ must satisfy $(\D, \He_1)_V = 
0$.  But $B(d,c) = 1$, so $B$ cannot be a trace form.

\paragraph{}
We conclude the lecture by defining a new class of Lie algebras:

\begin{defn}
A Lie algebra $\g$ is called \emph{semisimple} if it contains no nonzero 
solvable ideals.  Equivalently, $\g$ is semisimple if it contains no 
nonzero abelian ideals.
\end{defn}
Equivalence can be proved as follows: If $\g$ has a nonzero abelian ideal, 
then it has a nonzero solvable ideal, since abelian ideals are solvable.  
Conversely, if $\g$ has a nonzero solvable ideal $\h$, take $n$ such that 
$\h^{(n)} \not= 0$ but $\h^{(n+1)} = 0$; then $\h^{(n)}$ is a nonzero 
abelian ideal.

In the next lecture, we'll prove that a finite-dimensional Lie algebra 
$\g$ over a field of characteristic zero is semisimple if and only if the 
Killing form on $\g$ is nondegenerate.

\end{document}
