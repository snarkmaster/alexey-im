\documentclass{amsart}

\usepackage{amssymb}
\usepackage{amsmath}

\newtheorem*{lem}{Lemma}

\newtheorem*{Theorem}{Theorem}

\newtheorem*{Def}{Definition}

\newtheorem*{Prop}{Proposition}

\newcommand*{\g}{\mathfrak{g}}

\newenvironment*{Ex}
{\noindent \textbf{Example.}\hspace{.1 em}}

\newenvironment*{Problem}[1]
{\vspace{1em} \noindent \textbf{Exercise #1:}\hspace{.5em}}

\title{18.745: Lecture 11}
\author{Professor: Victor Ka\v{c}\\ Scribe: Christopher Davis}
\date{October 10th, 2004}

\begin{document}
\maketitle

\section*{Properties of semi-simple Lie algebras}

\begin{Problem}{11.1} In this exercise, sum will always mean sum as vector spaces:

\begin{enumerate}
\item[(a)]
The sum of two ideals of a Lie algebra is an ideal.
\item[(b)]
The sum of finitely many solvable ideals is a solvable ideal.
\item[(c)]
The sum of a collection of solvable ideals of a finite dimensional Lie algebra is a solvable ideal.
\item[(d)]
The sum of an ideal and a sub-algebra is a sub-algebra.
\end{enumerate}
\end{Problem}

\begin{proof}
(a):  Let $\g$ denote a Lie algebra over a field $\mathbb{F}$, let $\mathfrak{a}, \mathfrak{b}$
  be two ideals.   An arbitrary element of
  $\mathfrak{a} + \mathfrak{b}$ has the form $c_{1}a + c_{2}b$, with $c_{1}, c_{2} \in \mathbb{F}$, $a
  \in \mathfrak{a}$, $b \in \mathfrak{b}$.  Let $g \in \g$.  Then 
$$[g, c_{1}a + c_{2}b] = c_{1}[g,a] + c_{2}[g,b],$$
where the latter is clearly in $\mathfrak{a} + \mathfrak{b}$, since
$\mathfrak{a}$ and $\mathfrak{b}$ are ideals.

(b):  Using induction, it will suffice to show that the sum of two
solvable ideals is solvable.  Say $\mathfrak{a}^{(m)} =
\mathfrak{b}^{(n)} = 0$.  Then, considering bilinearity,
$(\mathfrak{a} + \mathfrak{b})^{(m+n)}$ is made of elements which are
sums of terms $$[a_{m+n} \textnormal{ or } b_{m+n}, [a_{m+n-1}
    \textnormal{ or } b_{m+n-1}, [ \ldots [a_{2} \textnormal{ or } b_{2},
	a_{1} \textnormal{ or } b_{1}]\ldots ].$$  Because
    $\mathfrak{a}$ and $\mathfrak{b}$ are ideals, each such term will
    lie in either $\mathfrak{a}^{(m)}$ or $\mathfrak{b}^{(n)}$,
    depending on whether there are at least $m$ terms from
    $\mathfrak{a}$ or at least $n$ terms from $b$.  Thus, each term is
    zero, and we conclude that the sum is solvable.  (It is an ideal
    by part (a).)

(c):  We now assume that $\g$ is finite dimensional, but also that we
    now have an arbitrary collection of ideals,
    $\mathfrak{a}_{\alpha}$.  Consider these ideals as vector
    subspaces.  We clearly have that $\sum \mathfrak{a}_{\alpha}$ is a
    subspace which contains each $\mathfrak{a}_{\alpha}$.  We claim
    that we can find a finite subcollection $\mathfrak{a}_{1}, \ldots,
    \mathfrak{a}_{n}$ of the ideals $\mathfrak{a}_{\alpha}$ so that
    $$\sum \mathfrak{a}_{i} = \sum \mathfrak{a}_{\alpha}.$$

Choose a
    non-empty ideal $\mathfrak{a}_{1}$.  If $\mathfrak{a}_{1} = \sum
    \mathfrak{a}_{\alpha},$ we are done, otherwise, pick another ideal
    $\mathfrak{a}_{2}$ which is not contained in $\mathfrak{a}_{1}$.
    If $\mathfrak{a}_{1} + \mathfrak{a}_{2} = \sum
    \mathfrak{a}_{\alpha},$ we are done, otherwise repeat the
    process.  At each stage, the dimension $\sum \mathfrak{a}_{i}$
    goes up by at least one, and as it is contained in the finite
    dimensional space $\g$, the process must terminate in a finite
    number of steps.  At this point, we can reduce to part (b).

(d):  Let $a_{1}, a_{2} \in \mathfrak{a}$ an ideal, $h_{1}, h_{2} \in
    \mathfrak{h}$ a sub-algebra, and $c_{i} \in \mathbb{F}$ the field we are
    working over.  Then $$[c_{1}a_{1} + c_{2}h_{1}, c_{3}a_{2} +
    c_{4}h_{2}] = c_{1}[a_{1},c_{3}a_{2} + c_{4}h_{2}] +
    c_{2}c_{3}[h_{1}, a_{2}] + c_{2}c_{4}[h_{1},h_{2}].$$  The first
    two terms are in our ideal, and the third term is in our
    sub-algebra.  Thus, we conclude that the sum of an ideal and a
    sub-algebra is indeed a sub-algebra.  (The other properties follow
    immediately from bilinearity of the bracket.)
\end{proof}

\begin{Def}
The radical $R(\g)$ of a finite dimensional Lie algebra $\g$ is the sum (as vector spaces) of all $\g$'s solvable ideals.
\end{Def}

By $11.1$(c), $R(\g)$ is 
a solvable ideal.  As it contains all the other solvable ideals, it is necessarily the maximal solvable
ideal of $\g$.  Recalling that we call a Lie Algebra with no non-zero solvable ideals \textit{semi-simple}, we see that in
the finite dimensional case, $\g$ is semi-simple if and only if $R(g) = 0$.

\begin{Prop} For a finite dimensional Lie algebra $\g$, $\overline{\g}
  = \g / R(\g)$ is a semi-simple Lie algebra.  
\end{Prop}

\begin{proof}  Consider any $g_{1}, g_{2} \in \g$, and let
  $\overline{g_{1}}, \overline{g_{2}}$ be their images in
  $\overline{\g}$.  We claim $\overline{[g_{1},g_{2}]} =
  [\overline{g_{1}}, \overline{g_{2}}]$.  But this is immediate from
  the fact that $R(\g)$ is an ideal, and thus taking different
  representatives of $\overline{g_{1}}$ and $\overline{g_{2}}$ and
  taking their bracket yields an element which differs from
  $[g_{1},g_{2}]$ by an element of $R(\g)$.

 Let $\overline{m}$ be a solvable ideal of
  $\overline{\g}$. Let $m$ denote its preimage in $\g$.  Then the
  above remarks show that $[\overline{m}, \overline{m}] =
  \overline{[m,m]}$, and applying this repeatedly shows that
  $\overline{m}$ is solvable if and only if $m^{(n)} \subseteq R(\g)$
  for some $n$.  Because $R(\g)$ is solvable, this in turn implies
  that $m$ is solvable, and thus, $m \subseteq R(\g)$.  Thus, our
  original $\overline{m}$ must be $0$, and so $\overline{\g}$ is
  indeed semi-simple.
\end{proof}

We will later prove the following, which provides a method for
breaking the study of finite dimensional Lie algebras down into the
study of (somewhat) simpler objects:

\begin{Theorem}[Levi]  If $\g$ is a finite dimensional Lie algebra
  over a field $\mathbb{F}$ of characteristic $0$, then there exists a
  semi-simple sub-algebra $\mathfrak{s}$ of $\g$ such that $$\g =
  \mathfrak{s} + R(\g),$$ a sum as vector spaces.  
\end{Theorem}

\noindent
The above is also a
  direct sum as vector spaces, because $\mathfrak{s} \cap R(\g)$ is
  contained in $R(g)$, hence is solvable, and is contained in the
  semi-simple $\mathfrak{s}$, and hence is $0$.  Such a situation is
  called a $\textit{semi-direct sum}$ of a sub-algebra $\mathfrak{s}$
  and an ideal $R(\g)$.

Note that in such a case, we have a homomorphism of Lie algebras 
\begin{align*}
\mathfrak{s} &\rightarrow \textnormal{Der } R(\g), \\
\intertext{defined by}
s &\rightarrow \textnormal{ad } s|_{R(\g)}.
\end{align*}

\noindent
This situation is reminiscent of the semi-direct product from
group theory, and motivates the following definition.

\begin{Def} Given a triple $(\mathfrak{s}, \mathfrak{r}, \varphi)$,
  with $\mathfrak{s}$, $\mathfrak{r}$ Lie algebras and $\varphi$ a
  homomorphism $\mathfrak{s} \rightarrow \textnormal{Der }
  \mathfrak{r}$, we define the \textit{semi-direct sum} 
 to be the Lie algebra $\g = \mathfrak{s}
  \ltimes \mathfrak{r}$ as follows:

\begin{itemize}
\item
As a vector space, $\g = \mathfrak{s} + \mathfrak{r}$, a direct sum of
vector spaces.
\item
The commutator of two elements of $\mathfrak{s}$ (resp. of
$\mathfrak{r}$) is defined as in $\mathfrak{s}$ (resp. in
$\mathfrak{r}$).
\item
The commutator $[s,r]$, with $s \in \mathfrak{s}$ and $r \in
\mathfrak{r}$, is defined to be $\varphi(s)(r)$.
\end{itemize}
\end{Def}

\noindent
That the above is well-defined follows from the next exercise:

\begin{Problem}{11.2}
The semi-direct sum $\g = \mathfrak{s}
  \ltimes \mathfrak{r}$ as defined above is a Lie algebra.
\end{Problem}

\begin{proof} Implicit in the definition is that we extend all the
  rules to make the bracket bilinear and skew-commutative.  It remains to check the Jacobi
  identity.
\begin{align*}
[\lambda_{1}s_{1} + \mu_{1}r_{1}, [\lambda_{2}s_{2} + \mu_{2}r_{2},
    \lambda_{3}s_{3} + \mu_{3}r_{3}]] + \hspace{1 em} &\\
[\lambda_{2}s_{2} + \mu_{2}r_{2},
    [\lambda_{3}s_{3} + \mu_{3}r_{3}, \lambda_{1}s_{1} +
    \mu_{1}r_{1}]] + \hspace{1 em} &\\
[\lambda_{3}s_{3} + \mu_{3}r_{3}, [\lambda_{1}s_{1} +
    \mu_{1}r_{1}, \lambda_{2}s_{2}  + \mu_{2}r_{2}]]  & =\\
[\lambda_{1}s_{1} + \mu_{1}r_{1}, \lambda_{2}\lambda_{3}[s_{2},s_{3}]
    + \lambda_{2}\mu_{3}\varphi(s_{2})(r_{3}) -
    \lambda_{3}\mu_{2}\varphi(s_{3})(r_{2}) + \mu_{2}\mu_{3}[r_{2},
    r_{3}]] + \hspace{1 em} & \\ 
[\lambda_{2}s_{2} + \mu_{2}r_{2}, -\lambda_{1}\lambda_{3}[s_{1},s_{3}]
    + \mu_{1}\lambda_{3}\varphi(s_{3})(r_{1}) -
    \lambda_{1}\mu_{3}\varphi(s_{1})(r_{3}) - \mu_{1}\mu_{3}[r_{1},
    r_{3}]] + \hspace{1 em} & \\
[\lambda_{3}s_{3} + \mu_{3}r_{3}, \lambda_{1}\lambda_{1}[s_{1},s_{2}]
    + \lambda_{1}\mu_{2}\varphi(s_{1})(r_{2}) -
    \mu_{1}\lambda_{2}\varphi(s_{2})(r_{1}) + \mu_{1}\mu_{2}[r_{1},
    r_{2}]] & =\\
\lambda_{1}\lambda_{2}\lambda_{3}[s_{1},[s_{2},s_{3}]] 
+ \lambda_{1}\lambda_{2}\mu_{3}\varphi(s_{1})(\varphi(s_{2})(r_{3})
- \hspace{1 em} & \\
\lambda_{1}\mu_{2}\lambda_{3}\varphi(s_{1})(\varphi(s_{3})(r_{2})
+\lambda_{1}\mu_{2}\mu_{3}\varphi(s_{1})([r_{2},r_{3}]) - \hspace{1
    em} & \\
\mu_{1}\lambda_{2}\lambda_{3}\varphi([s_{2},s_{3}])(r_{1}) 
+ \mu_{1}\lambda_{2}\mu_{3}[r_{1},\varphi(s_{2})(r_{3})]
- \hspace{1 em} & \\
\mu_{1}\mu_{2}\lambda_{3}[r_{1},(\varphi(s_{3})(r_{2})]
+\mu_{1}\mu_{2}\mu_{3}[r_{1},[r_{2},r_{3}]] - \hspace{1
    em} & \\
\lambda_{1}\lambda_{2}\lambda_{3}[s_{2},[s_{1},s_{3}]] 
+ \mu_{1}\lambda_{2}\lambda_{3}\varphi(s_{2})(\varphi(s_{3})(r_{1})
- \hspace{1 em} & \\
\lambda_{1}\lambda_{2}\mu_{3}\varphi(s_{2})(\varphi(s_{1})(r_{3})
-\mu_{1}\lambda_{2}\mu_{3}\varphi(s_{2})([r_{1},r_{3}]) + \hspace{1
    em} & \\
\lambda_{1}\mu_{2}\lambda_{3}\varphi([s_{1},s_{3}])(r_{2}) 
+ \mu_{1}\mu_{2}\lambda_{3}[r_{2},\varphi(s_{3})(r_{1})]
- \hspace{1 em} & \\
\lambda_{1}\mu_{2}\mu_{3}[r_{2},(\varphi(s_{1})(r_{3})]
-\mu_{1}\mu_{2}\mu_{3}[r_{2},[r_{1},r_{3}]] + \hspace{1
    em} & \\
\lambda_{1}\lambda_{2}\lambda_{3}[s_{3},[s_{1},s_{2}]] 
+ \lambda_{1}\mu_{2}\lambda_{3}\varphi(s_{3})(\varphi(s_{1})(r_{2})
- \hspace{1 em} & \\
\mu_{1}\lambda_{2}\lambda_{3}\varphi(s_{3})(\varphi(s_{2})(r_{1})
+\mu_{1}\mu_{2}\lambda_{3}\varphi(s_{3})([r_{1},r_{2}]) - \hspace{1
    em} & \\
\lambda_{1}\lambda_{2}\mu_{3}\varphi([s_{1},s_{2}])(r_{3}) 
+ \lambda_{1}\mu_{2}\mu_{3}[r_{3},\varphi(s_{1})(r_{2})]
- \hspace{1 em} & \\
\mu_{1}\lambda_{2}\mu_{3}[r_{3},(\varphi(s_{2})(r_{1})]
+\mu_{1}\mu_{2}\mu_{3}[r_{3},[r_{1},r_{2}]] & =\\
\intertext{we can now simplify using the Jacobi identities for
    $\mathfrak{r}$ and $\mathfrak{s}$, and the identity
    $\varphi([s_{1},s_{2}])(r_{3}) =
    \varphi(s_{1})(\varphi(s_{2})(r_{3})) - \varphi(s_{2})(\varphi(s_{1})(r_{3}))$.}
\lambda_{1}\mu_{2}\mu_{3}\varphi(s_{1})([r_{2},r_{3}])  
+ \mu_{1}\lambda_{2}\mu_{3}[r_{1},\varphi(s_{2})(r_{3})]
- \hspace{1 em} & \\
\mu_{1}\mu_{2}\lambda_{3}[r_{1},(\varphi(s_{3})(r_{2})]
-\mu_{1}\lambda_{2}\mu_{3}\varphi(s_{2})([r_{1},r_{3}]) + \hspace{1.5
    em} & \\ 
+ \mu_{1}\mu_{2}\lambda_{3}[r_{2},\varphi(s_{3})(r_{1})]
- \lambda_{1}\mu_{2}\mu_{3}[r_{2},(\varphi(s_{1})(r_{3})]- \hspace{1.5
    em} & \\
\mu_{1}\mu_{2}\lambda_{3}\varphi(s_{3})([r_{1},r_{2}])
+ \lambda_{1}\mu_{2}\mu_{3}[r_{3},\varphi(s_{1})(r_{2})]
- \mu_{1}\lambda_{2}\mu_{3}[r_{3},(\varphi(s_{2})(r_{1})]
    =\textnormal{ } & 0,
\end{align*}
where the last equality follows from the fact that $\varphi(s_{i})$ is
a derivation.
\noindent
\end{proof}

Note in particular that if $\varphi = 0$, then $\g = \mathfrak{s}
  \ltimes \mathfrak{r}$ is just the usual direct sum of $\mathfrak{s}$
  and $\mathfrak{r}$.

Thus, we see that the classification of finite dimensional Lie
algebras over a field of characteristic zero reduces to the
determination of three things:

\begin{enumerate}
\item \label{semi}
All finite-dimensional semi-simple Lie algebras.
\item \label{solve}
All finite-dimensional solvable Lie algebras.
\item \label{hom}
For $\mathfrak{s}$ semi-simple, $\mathfrak{r}$ solvable, all homomorphisms
$\phi: \mathfrak{s} \rightarrow \textnormal{Der }\mathfrak{r}$.
\end{enumerate}

We will completely solve \ref{semi}, but \ref{solve} is a wild
problem. 

\begin{Problem}{11.3}
Let $\g \subseteq gl_{m+n}(\mathbb{F})$ be the Lie algebra consisting
of matrices of the form

$$g = \left( \begin{array}{c|c}
a & b \\ \cline{1-2} 
0 & c 
\end{array}\right),$$

\noindent
with $a \in gl_{m}$, $c \in gl_{n}$, and $b$ an $m \times n$ matrix.
Compute $R(\g)$, the complementary $\mathfrak{s}$, and describe the
homomorphism $\varphi: \mathfrak{s} \rightarrow \textnormal{Der }R(\g)$.
\end{Problem}

\begin{proof}  Because of the multiplication
$$ \left[ \left( \begin{array}{c|c}
a & b \\ \cline{1-2} 
0 & c 
\end{array}\right) , \left( \begin{array}{c|c}
a & b \\ \cline{1-2} 
0 & c 
\end{array}\right) \right] = \left( \begin{array}{c|c}
[a_{0},a_{1}] & * \\ \cline{1-2} 
0 & [c_{0},c_{1}] 
\end{array}\right),$$
we see that if we have an ideal $R(\g)$ of $\g$, then the set of
elements appearing in the ``$a$'' block must be an ideal of $gl_{m}$,
and the elements appearing in the ``$c$'' block must be an ideal of
$gl_{n}$.  Furthermore, as $R(\g)$ is solvable, these sets must be
solvable.  

We will see later in lecture that the maximal solvable ideal of
$gl_{m}$ is simply the ideal of scalar matrices.  (This is not
circular, as we will not use this exercise to prove it.)  We now claim
that $R(\g)$ has no other restrictions, beyond this one.  In
particular, $R(\g)$ consists of all matrices of the form:

$$\left( \begin{array}{c|c}
\lambda I_{m} & b \\ \cline{1-2} 
0 & \mu I_{n} 
\end{array}\right),$$

\noindent
for all $\lambda, \mu \in \mathbb{F}$, and all $m \times n$ matrices
$b$.

To see that this sub-algebra is solvable, consider
$$ \left[ \left( \begin{array}{c|c}
\lambda_{1} & b_{1} \\ \cline{1-2} 
0 & \mu_{1} 
\end{array}\right) , \left( \begin{array}{c|c}
\lambda_{2} & b_{2} \\ \cline{1-2} 
0 & \mu_{2} 
\end{array}\right) \right] = \left( \begin{array}{c|c}
0 & * \\ \cline{1-2} 
0 & 0 
\end{array}\right).$$

Thus, the matrices in the derived algebra are strictly
upper-triangular, and hence our sub-algebra is solvable.  Actually,
the same calculation as above shows that our sub-algebra is an ideal.
We simply note that not only does $[\lambda_{1}, \lambda_{2}] = 0$,
but also $[\lambda_{1}, a_{2}] = 0$, for any $a_{2} \in gl_{m}$.

For a general matrix $a \in \g$, we can, through subtraction, see that
$$a \bmod R(\g) \equiv  \left( \begin{array}{c|c}
d & 0 \\ \cline{1-2} 
0 & e 
\end{array}\right),$$ for some $d \in sl_{m}$ and $e \in sl{n}$.
Because we are in the characteristic zero case, we see that there is
no overlap, and so the above forms the complementary algebra.

Simply by calculating the bracket, we see that 

\begin{align*}
\varphi: \mathfrak{s} & \rightarrow \textnormal{Der }R(\g) \\
\left( \begin{array}{c|c}
d & 0 \\ \cline{1-2} 
0 & e 
\end{array}\right) & \rightarrow \psi_{d,e} \\
\intertext{where } \psi_{d,e}:\left( \begin{array}{c|c}
\lambda & b \\ \cline{1-2} 
0 & \mu 
\end{array}\right) & \rightarrow \left( \begin{array}{c|c}
0 & db-be \\ \cline{1-2} 
0 & 0 
\end{array}\right).
\end{align*}

\end{proof}

The remainder of the lecture will concern methods to identify Lie
algebras as semi-simple.  Recall the following definition:

\begin{Def} Given a subalgebra $\g \subseteq gl_{V}$, we say that $V$
  is irreducible with respect to $\g$ if the only invariant subspaces
  of $V$ with respect to $\g$ are $0$ and $V$.
\end{Def}

\begin{Theorem}
Let $V$ be a finite dimensional Lie algebra over an algebraically
closed field $\mathbb{F}$ of characteristic zero.  Let $\mathfrak{g}
\subseteq gl_{V}$ be a subalgebra such that $V$ is irreducible with
respect to $\g$.  Then one of the following two possibilities holds:
either $\g$ is semi-simple, or 
$\g = \mathfrak{s} \oplus \mathbb{F}\textnormal{I}_{V}$, where
$\mathfrak{s}$ is semi-simple and $\textnormal{I}_{V}$ denotes the
identity operator on $V$.
\end{Theorem}

\begin{proof}

Consider the weight space $$V_{\lambda} = \{ v \in V | a(v) =
\lambda(a)v, a \in R(\g)\}.$$  (Here, the representation 
$\pi$ is the defining representation of $\mathfrak{g}
 \subseteq gl_{V}$, and we omit it.)  
Because $R(\g)$ is solvable, Lie's Theorem assures us that the elements
$a$ have a common non-zero eigenvector.  Simply because $a$ is a linear
operator on $V$, we have that the associated function $\lambda(a)$ is
linear, and thus the weight-space is non-zero for some $\lambda \in
R(\g)^{*}$.

By Lie's Lemma, $V_{\lambda}$ is invariant with respect to $\g$.  From
our assumptions, we can then conclude $V_{\lambda} = 0$ or $V$.  But
we know it's non-zero.  Thus, the elements of $R(\g)$ act either as
scalars or as zero.  In the former case, the complementary semi-simple
Lie algebra must then be $$\mathfrak{s} = \{a \in \g |
  \textnormal{tr}(a) = 0 \},$$ and in this case the sum is actually a
  direct sum, because the scalar matrices are in the center of $\g$.  

\end{proof}

\begin{Ex}
For any two
non-zero vectors $v_{1}$ and $v_{2}$ in some finite dimensional 
vector space $V$, we can find an operator
$g \in gl_{V}$ sending $v_{1}$ to $v_{2}$.  Thus, any non-zero
invariant subspace of $gl_{V}$ must contain every non-zero vector.
Clearly it must also contain the zero vector.  Hence, $V$ is
irreducible with respect to $gl_{V}$.  

If $V$ is a vector
space over an algebraically closed field of characteristic zero, we
can then apply the above theorem.
The algebra $gl_{V}$ contains the scalar matrices, and in particular
$$gl_{V} = sl_{V} \oplus \mathbb{F}\textnormal{I}_{V}.$$  This last
equation follows from the fact that  $sl_{V}$ is an ideal in $gl_{V}$
(this follows from Exercise $1.2$), $\mathbb{F}\textnormal{I}_{V}$,
their intersection is empty, and the sum of their dimensions is the
dimension of $gl_{V}$.  Our theorem thus lets us conclude that
$sl_{V}$ is semi-simple.  (It should be noted that the above argument
does not work if the characteristic of our field is $p > 0$ and $p$
divides the dimension of $V$, because
in that case the scalar matrices are inside of $sl_{V}$.)
\end{Ex}

\begin{Problem}{11.4}
Prove that $sp_{V,B}$ is always semi-simple, and $so_{V,B}$ is
semi-simple if and only if the $\textnormal{dim } V \neq 2$.
(Assuming still that $\textnormal{char } \mathbb{F} = 0$, with
$\mathbb{F}$ algebraically closed.)
\end{Problem}

\begin{proof}


We first consider $sp_{V,B}$.  Choose $B$ so that $B$'s anti-diagonal
elements, i.e. the ones $(b_{n1}, \ldots, b_{1 n})$, are $(-1, -1,
\ldots, -1,1,\ldots, 1)$, with $n/2 := m$ of each.  Then the matrices of
$sp_{V,B}$ are those for which 

\begin{align*}
\left( \begin{array}{cccccc}
-a_{n1} & \ldots & -a_{m+1 1} & a_{m 1} & \ldots & a_{11} \\ 
-a_{n2} & \ldots & -a_{m+1 2} & a_{m 2} & \ldots & a_{12} \\
\vdots & & & & & \vdots\\ 
\vdots & & & & & \vdots\\ 
-a_{nn} & \ldots & -a_{m+1 n} & a_{m n} & \ldots & a_{1n} \\
\end{array}\right) + \hspace{10 em} & \\
\left( \begin{array}{ccccc}
a_{n1} & \ldots & a_{n m+1} & \ldots & a_{nn} \\
\vdots & & & & \vdots \\ 
a_{m+1 1} & \ldots & a_{m+1 m+1} & \ldots & a_{m+1 n} \\
-a_{m 1} & \ldots & -a_{m m+1} & \ldots &-a_{m n} \\
\vdots & & & & \vdots\\  
-a_{11} & & & & -a_{1n} \\
\end{array}\right) 
& = 0.
\end{align*}

\noindent
Because the terms along the anti-diagonal are arbitrary, it's clear
that if $v$ is a vector with no zero components, then we can map $v$
to any vector $w$ by choosing those anti-diagonal elements suitably,
and leaving the rest as zero.  Consider now a vector $v$ with $v_{i} =
0$, and $v_{j} \neq 0$.  Choose a matrix in $sp_{V,B}$ with $a_{ij} =
1$.  If $i+j = n+1$, then set the rest of the entries equal to zero.
Else, set the necessary entry to $-1$, and the rest to zero.  If it
appears at all, the $-1$ occurs in a row different from the $1$, and
so our matrix maps $v$ to a vector $w$ with $w_{i} = v_{j} \neq 0$.
Repeating this process and taking some linear combination, 
we see that any invariant subspace containing
$v$ contains an element which has no zero entries, and as we saw
above, that subspace must therefore be all of $V$.  As the
characteristic is not two, $sp_{V,B}$ does not contain the scalar
matrices, and so, from the previous theorem, $sp_{V,B}$ is semi-simple.

Consider now $so_{V,B}$.  Assume that $B = I_{n}$, and then the
matrices of $so_{V,B}$ are such that $A + A^{T} = 0$.  Say $n=2$.
Then $so_{V,B}$ is one-dimensional, and thus solvable, and in
particular not semi-simple.

Consider the
following equation, where the $a_{ij}$ terms should be viewed as
variables:

$$\left( \begin{array}{ccccc}
0 & -a_{21} & -a_{31} & \ldots &- a_{n1} \\ 
a_{21} & 0 & -a_{32} & &\\
a_{31} & a_{32} & \ddots & &\\ 
\vdots & & & & \\ 
a_{n1} & \ldots & & & 0
\end{array}\right) \left( \begin{array}{c}
v_{1} \\
v_{2} \\
v_{3} \\
\vdots \\
v_{n} \end{array}\right) = \left( \begin{array}{c}
w_{1} \\
w_{2} \\
w_{3} \\
\vdots \\
w_{n} \end{array}\right).$$

Using the same argument as above, we know that a non-zero
invariant subspace contains a vector with no zero components.  It's
then clear that we can choose our matrix $A$ so that $Av = (w_{1},
\ldots, w_{i-1},?,w_{i+1},\ldots,w_{n})^{T}$, i.e., we can choose all
but one of the image's coordinates, and we can choose the location of
the unspecified coordinate.  Can we specify $n$ linearly independent
vectors in this way?  Yes, as long as $n \geq 3$, because we can
specify the vectors $e_{1} + a_{1}e_{n}, e_{2} + a_{2}e_{n}, \ldots,
e_{n-1} + a_{n-1}e_{n}, b_{1}e_{1} + e_{n}$.  These
$n$ being linearly dependent would imply either that 
$b_{1} = 0$, in which case we're done, or else $b_{1}a_{1} = 1$,
and in particular, $a_{1} \neq 0$.  Now, choose the matrix in $so_{V,B}$ with 
$a_{21}=-1$, 
$a_{12} = 1$, and the rest of the elements zero.  This matrix maps
$e_{2}+b_{2}e_{n}$ to $e_{1}$ (this is where we use that $n \geq 3$).  
This vector is linearly independent from the 
first $n-1$ we wrote above as long as $a_{1} \neq 0$, which we just saw, 
so we have our linearly independent set of vectors, so our invariant
subspace must be all of $V$.
We conclude that $V$ is irreducible with respect to
$so_{V,B}$, and thus, if $n \geq 3$, we have that $so_{V,B}$ is semi-simple.

\end{proof}

Here is another theorem useful for determining semi-simplicity:

\begin{Theorem}
A finite dimensional Lie algebra $\g$ over a field $\mathbb{F}$ of
characteristic zero is semi-simple if and only if its Killing form is non-degenerate.
\end{Theorem}

\begin{proof}
Suppose the Killing form $K(a,b)$ is non-degenerate.  Let
$\mathfrak{a}$ denote an abelian ideal of $\g$.  To show that $\g$ is
semi-simple, it is equivalent to show that any such $\mathfrak{a} =
0$.  Now, for $a \in \mathfrak{a}$,
\begin{align*}
\textnormal{ad }a: \g & \rightarrow \mathfrak{a} \\
\mathfrak{a} & \rightarrow 0,
\end{align*}
simply because $\mathfrak{a}$ is an ideal.  Similarly, for $g \in \g$,
we have:  $\textnormal{ad }g: \mathfrak{a} \rightarrow
\mathfrak{a}$.  Write $\g = \mathfrak{a} \oplus V$, a direct sum as
vector spaces.  We then can partially determine the form of 
the following matrices:

\begin{align*}
\textnormal{ad }a & = \left( \begin{array}{c|c}
0 & A \\ \cline{1-2} 
0 & 0 
\end{array}\right)\\
\textnormal{ad }g & = \left( \begin{array}{c|c}
B & C \\ \cline{1-2} 
0 & D 
\end{array}\right),
\end{align*}
\noindent
where the first row and first column have size $\textnormal{dim }
\mathfrak{a}$ and the second row and second column have size $\textnormal{dim }
\mathfrak{V}$.  Their product is a strictly upper triangular matrix,
and as $g \in \g$ was arbitrary, we conclude that either $a = 0$ or
the Killing form is degenerate.  The former must hold, lest we have a
contradiction, and so our $\g$ is semi-simple.

Now, for the converse, assume $\g$ is semi-simple.  Let $$\mathfrak{r} = \{ a \in
\g | K(a,\g) = 0 \}.$$  From lecture $9$, we know that $\mathfrak{r}$ is an
ideal.  Considering the adjoint representation of $\mathfrak{r}$ in $gl_{\g}$,
then implication $(2) \Rightarrow (3)$ in Cartan's Criterion shows
that $\textnormal{ad }\mathfrak{r}$ is solvable.  Then, as usual,
if $(\textnormal{ad }\mathfrak{r})^{(m)} = 0,$ we know
$\mathfrak{r}^{(m)}$ is in the center of $\g$, and hence
$\mathfrak{r}^{(m+1)}=0$, so $\mathfrak{r}$ is solvable, which implies
$\mathfrak{r} = 0$, showing that the Killing form is indeed non-degenerate.

\end{proof}

The preceding theorem allows us to apply the following result to
semi-simple Lie algebras.

\begin{Theorem}
If the Killing form on a finite dimensional Lie algebra is
non-degenerate, then the center of $\g$, $Z(\g)$, equals zero, and all derivations
of $\g$ are inner.  In particular, these claims hold for a semi-simple
Lie algebra over a field of characteristic zero.
\end{Theorem}

\begin{proof}

If $c \in Z(\g)$, then $\textnormal{ad }c = 0$, and so 
$$K(c,a) = tr_{\g}(\textnormal{ad }c\textnormal{ ad
}a) = 0,$$ hence $c = 0$.  Hence the adjoint map $$\textnormal{ad}:
\g \rightarrow gl_{\g}$$ is injective, and we may identify $\g$ with
its image.  Thus, we from now on assume $\g \subseteq \textnormal{Der
}\g \subseteq gl_{\g}$.  It remains for us to prove $\g =
\textnormal{Der } \g$.

The Killing form on $\g$ is by definition the trace form on $gl_{\g}$,
and its restriction to $\g$ is, by our assumption, non-degenerate.  

Denote by $g^{\perp}$ the set $\{D \in \textnormal{Der } \g |
(D,\g)_{\g}=0 \}$.

\begin{Problem}{11.5}  If $V$ is a finite dimensional vector space
  with a symmetric bilinear form  $B$ and a subspace $U$ on which $B$
  is non-degenerate, then $V = U \oplus U^{\perp},$ where $$U^{\perp}
  = \{ v \in V | B(v,u) = 0 \textnormal{ for all } u \in U \}.$$
\end{Problem}

\begin{proof}
Through a change of basis if necessary, we can assume we have a basis
$\{u_{1}, \ldots, u_{n}, w_{1}, \ldots, w_{m} \}$ for $V$ so that 
$\{u_{1}, \ldots, u_{n} \}$ is a basis for $U$ on which $B$ is the
identity.

In this basis, $B$ takes the form of the matrix $$\left( \begin{array}{c|c}
I_{n} & 0 \\ \cline{1-2} 
0 & ? 
\end{array}\right),$$ where we use symmetry to fill in the top right
box.  From here, the assertion is obvious.
\end{proof}

Applying the exercise, $\g = \g \oplus \g^{\perp}$, a direct sum as
vector spaces.  From exercise $1.5$(b), $\g$ is an ideal of
$\textnormal{Der } \g$.  From Lecture $9$, we know that the trace form
is invariant, and that $\g^{\perp}$ is an ideal of $\g$.  Thus, our
direct sum as vector spaces is actually a direct sum of ideals, too,
and furthermore, for any $D \in \g^{\perp}$, we have $[\g,D] = 0$.
This means that $D$ acts as $0$ on every element $g \in \g$.  But
since $D$ is by definition just a derivation of $\g$, we see that $D$
is the trivial derivation, which completes the proof.

\end{proof}

Finally, we end with another exercise, which shows that the previous
theorem does not have a converse:

\begin{Problem}{11.6}
All derivations of the two-dimensional non-abelian Lie algebra are
inner, but its Killing form is degenerate.
\end{Problem}

\begin{proof}
Consider a derivation $D$ of the non-abelian, two-dimensional Lie
algebra.  We know $$Db = D[a,b] = [Da,b] + [a,Db].$$  Say $D(a) =
\lambda_{11}a + \lambda_{21}b$ and $D(b) = \lambda_{12}a +
\lambda_{22}b$.  Then the above equation means $$\lambda_{12}a +
\lambda_{22}b = \lambda_{11}a +
\lambda_{22}b,$$ which in turn implies $\lambda_{12} = \lambda_{11} =
0$.  

We now simply check that 
\begin{align*}
[\lambda_{22}a -\lambda_{21}b,a] &= \lambda_{21}b \\
\intertext{and} [\lambda_{22}a -\lambda_{21}b,b] &= \lambda_{22}b,
\end{align*}
which matches our derivation $D$.  Thus we conclude that all
derivations are inner.  Also, note that the center of our algebra is
trivial.

Now consider the Killing Form.  We have 
$$\textnormal{ad }a = \left( \begin{array}{cc}
0 & 1 \\
0 & 0 
\end{array}\right) \textnormal{and ad }b = \left( \begin{array}{cc}
0 & 0 \\  
-1 & 0 
\end{array}\right).$$  So, 
$K(b,a) = K(b,b) = 0$,
and the Killing form is thus degenerate.  We conclude that the
previous theorem does not have a converse.
\end{proof}

\end{document}
