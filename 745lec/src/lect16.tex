\documentclass[11pt]{article}
\usepackage{cancel}
\usepackage{amsmath}
\usepackage{amssymb}
\usepackage[dvips]{graphicx}
\usepackage[dvips]{color}

\newcommand{\handout}[5]{
  \noindent
  \begin{center}
  \framebox{
    \vbox{
      \hbox to 5.78in { {\bf 18.745 Introduction to Lie Algebras } \hfill #2 }
      \vspace{4mm}
      \hbox to 5.78in { {\Large \hfill #5  \hfill} }
      \vspace{2mm}
      \hbox to 5.78in { {\em #3 \hfill #4} }
    }
  }
  \end{center}
  \vspace*{4mm}
}

\newcommand{\lecture}[4]{\handout{#1}{#2}{#3}{Transcribed by: #4}{#1}}


\DeclareSymbolFont{AMSb}{U}{msb}{m}{n}
\DeclareMathSymbol{\N}{\mathbin}{AMSb}{"4E}
\DeclareMathSymbol{\Z}{\mathbin}{AMSb}{"5A}
\DeclareMathSymbol{\R}{\mathbin}{AMSb}{"52}
\DeclareMathSymbol{\Q}{\mathbin}{AMSb}{"51}
\DeclareMathSymbol{\I}{\mathbin}{AMSb}{"49}
\DeclareMathSymbol{\C}{\mathbin}{AMSb}{"43}
\DeclareMathSymbol{\F}{\mathbin}{AMSb}{"46}

\newcommand{\GL}{\mbox{GL}}
\newcommand{\tr}{\mbox{tr\ }}
\newcommand{\Mat}{\mbox{Mat}}
\newcommand{\Lie}{\mbox{Lie}}
\newcommand{\Der}{\mbox{Der\ }}
\newcommand{\End}{\mbox{End\ }}
\newcommand{\ad}{\mbox{ad\ }}
\newcommand{\im}{\mbox{im\ }}
\newcommand{\Ker}{\mbox{ker\ }}

\newcommand{\sll}{\ensuremath{\mathfrak{sl}}}
\newcommand{\gl}{\ensuremath{\mathfrak{gl}}}
\newcommand{\g}{\ensuremath{\mathfrak{g}}}
\newcommand{\h}{\ensuremath{\mathfrak{h}}}
\newcommand{\m}{\ensuremath{\mathfrak{m}}}
\newcommand{\He}{\ensuremath{\mathcal{H}}}
\newcommand{\be}{\ensuremath{\mathfrak{b}_2}}
\newcommand{\bk}{\ensuremath{\mathfrak{b}}}
\newcommand{\nk}{\ensuremath{\mathfrak{n}}}
%\renewcommand{\ad}[1]{\ensuremath{{\bf ad}(#1)}}



\newcommand{\sk}{\vspace*{1em}}

%\newtheorem{theorem}{Theorem}[section]
%\newtheorem{lemma}[theorem]{Lemma}
%\newtheorem{proposition}[theorem]{Proposition}
%\newtheorem{corollary}[theorem]{Corollary}

\newenvironment{theorem}[1][Theorem]{\begin{trivlist}
\item[\hskip \labelsep {\bfseries #1}]}{\end{trivlist}}
\newenvironment{proof}[1][Proof]{\begin{trivlist}
\item[\hskip \labelsep {\bfseries #1}]}{\end{trivlist}}
\newenvironment{definition}[1][Definition]{\begin{trivlist}
\item[\hskip \labelsep {\bfseries #1}]}{\end{trivlist}}
\newenvironment{example}[1][Example]{\begin{trivlist}
\item[\hskip \labelsep {\bfseries #1}]}{\end{trivlist}}
\newenvironment{remark}[1][Remark]{\begin{trivlist}
\item[\hskip \labelsep {\bfseries #1}]}{\end{trivlist}}
\newenvironment{exercise}[1][Exercise]{\begin{trivlist}
\item[\hskip \labelsep {\bfseries #1}]}{\end{trivlist}}
\newenvironment{solution}[1][Solution]{\begin{trivlist}
\item[\hskip \labelsep {\bfseries #1}]}{\end{trivlist}}
\newenvironment{claim}[1][Claim]{\begin{trivlist}
\item[\hskip \labelsep {\bfseries #1}]}{\end{trivlist}}

\newcommand{\qed}{\nobreak \ifvmode \relax \else
      \ifdim\lastskip<1.5em \hskip-\lastskip
      \hskip1.5em plus0em minus0.5em \fi \nobreak
      \vrule height0.75em width0.5em depth0.25em\fi}

% 1-inch margins, from fullpage.sty by H.Partl, Version 2, Dec. 15, 1988.
\topmargin 0pt \advance \topmargin by -\headheight \advance
\topmargin by -\headsep \textheight 8.9in \oddsidemargin 0pt
\evensidemargin \oddsidemargin \marginparwidth 0.5in \textwidth
6.5in

\parindent 0in
\parskip 1.5ex
%\renewcommand{\baselinestretch}{1.25}

\begin{document}

\lecture{Lecture 16: More On Root Systems}{Fall 2004}{Prof.\
Victor Ka\v{c}}{J.W. Powell}

\emph{The exercises will be stated as they occurred during the
course of the lecture, embedded within the rest of the notes.
Then, they will be restated at the end of all the notes with
solutions.}

\begin{remark}
We proved previously that the root system $(V, \,\Delta)$ coming
from a finite dimensional Lie algebra is of the form $V=\R
\otimes_{\Q} \h^{*}_{\Q}$ and $(\alpha,\beta)$ $\in \Q$ for
$\alpha,\beta \in \Delta$.  Also, choosing basis $B=\{
\beta_1,\beta_2,\ldots,\beta_r \}$ of $V$ consisting of roots, any
other root is a linear combination of elements from $B$ with
rational coefficients.

Indeed, if $\alpha = \sum_i c_i \beta_i$, then $(\alpha,\beta_j) =
\sum_i c_i (\beta_i,\beta_j)$, where $\left[
(\beta_i,\beta_j)\right]_{{i,j}=1}^r$ is a non-singular matrix
over $\Q$, and hence the $c_i$ are rational numbers by Cramer's
Rule.

As for indecomposable abstract root systems, it follows from the
proof of uniqueness of $(\cdot,\cdot)$ up to a factor that, after
multiplying it by a factor so that $(\alpha,\alpha)\in \Q$ for one
root $\alpha$, we will have $(\alpha,\beta)\in \Q$ for all
$\alpha,\beta \in Q$ and the above argument again applies.

\begin{example}
1,$\sqrt 2 $ are linearly independent over $\Q$ but are dependent
over $\R$.  Also, the set $\Z \cdot 1 + \Z \cdot \sqrt 2 $ is not
discrete in $\R$.
\end{example}
\end{remark}

\begin{definition}
A lattice $L$ in a Euclidean space $V$ over $\R$ is a discrete
subgroup of $V$ which spans $V$ over $\R$.
\end{definition}

\begin{example}
Let $\Delta$ be a finite subset of $V$ such that
$(\alpha,\beta)\in \Q$ for $\forall \alpha,\beta \in \Delta$.
Then, $\Z \, \Delta$ is a lattice in $V$.  Indeed, choose a basis
$B \subset \Delta$ of $V$.  All vectors of $\Delta$ will have
rational coordinates in the basis, and hence $\Z \, \Delta$ is a
discrete subgroup of $V$.
\end{example}
%
Equivalently, a lattice $L$ in a Euclidean space V over $\R$ of
dimension $r$ is a free abelian subgroup of rank $r$, that is $L =
\Z \beta_1 + \cdots + \Z \beta_r$, where $\beta_1,\ldots,\beta_r$
are linearly independent.
%
The root lattice of an abstract room system $(V,\Delta)$ is $Q =
\Z \, \Delta$.  We have constructed the following root systems:
%
\begin{itemize}
\item $\Delta_{A_r} = \{ \epsilon_i - \epsilon_j |
i,j=1,\ldots,r+1, \,\, i\not = j \}$ and $Q_{A_r} = \{ \sum _{i=1}
^{r+1} a_i \epsilon_i | \sum _{i=1}^{r+1}a_i=0 \mbox{ for } a_i
\in \Z \}$ in the vector space $V = \{ \sum a_i \epsilon_i | a_i
\in \R, \sum a_i = 0 \}$.
%
\item $\Delta_{B_r} = \{ \pm \epsilon_i \pm \epsilon_j, \pm
\epsilon_i | i,j=1,\ldots,r, \,\, i\not = j \}$ and $Q_{B_r} = \{
\sum _{i=1} ^{r} a_i \epsilon_i | a_i \in \Z \}$ in the vector
space $V = \sum \R \epsilon_i $.
%
\item $\Delta_{C_r} = \{ \pm \epsilon_i \pm \epsilon_j, \pm 2
\epsilon_i | i,j=1,\ldots,r, \,\, i\not = j \}$ and $Q_{C_r} = \{
\sum _{i=1} ^{r} a_i \epsilon_i | a_i \in \Z, \,\, \sum_{i=1}^{r}
a_i \in 2 \Z \}$ in the vector space $V = \sum \R \epsilon_i $.
%
\item $\Delta_{D_r} = \{ \pm \epsilon_i \pm \epsilon_j |
i,j=1,\ldots,r, \,\, i \not = j \}$ and $Q_{D_r} = Q_{C_r}$ in the
vector space $V = \sum \R \epsilon_i $.  Here $r \not = 2$.
\end{itemize}
%
\begin{exercise}
Prove that $Q_{A_r}$ and $Q_{C_r}$ are as above.
\end{exercise}
%
In addition to the lattices $Q_{A_r}, Q_{B_r}, Q_{C_r}, Q_{D_r}$,
there are exceptional lattices.  For instance, $\Gamma_r = \{
\sum_{i=1}^r a_i \epsilon_i \, | \, \mbox{all } a_i \in \Z \mbox{
or all } a_i \in \Z + \frac{1}{2}, \,\, \sum_{i=1}^{r} a_i \in 2
\Z$ \}.  This lives in the vector space $V = \bigoplus_{i=1}^{r}
\R \epsilon_i$ with $(\epsilon_i, \epsilon_j) = \delta_{ij}$.  The
most important example of such a lattice is $\Gamma_8$.
%
\begin{definition}
A lattice is called \emph{integral} (resp. \emph{even}) if
$(\alpha,\beta)\in \Z$ (resp. $(\alpha,\alpha)\in 2\Z$) for all
$\alpha,\beta \in Q$.
\end{definition}
Obviously an even lattice is integral: $(\alpha + \beta,\alpha +
\beta) = (\alpha,\alpha) + 2(\alpha,\beta)+(\beta,\beta)
\rightarrow (\alpha,\beta) \in \Z$, because $(\alpha +
\beta,\alpha + \beta), (\alpha,\alpha), (\beta,\beta) \in 2\Z$.
%
\begin{example}
$\Gamma_{8n} \,\, (n=1,2,\ldots)$ are even lattices, since
$(\alpha,\alpha) = \sum_{i=1}^{n} a_i^2 = \sum_{i=1}^{n} a_i \mod
2$.  Recall that $a_i^2 \equiv a_i \mod 2$ if $a_i$ is integral.
All other vectors are of the form $\rho + \alpha$ where $\rho =
(\frac{1}{2},\frac{1}{2},\cdots,\frac{1}{2})$ and $\alpha$ has
integer coordinates.  But then $(\rho + \alpha, \rho + \alpha) =
(\rho,\rho)+\sum_{i=1}^n a_i + (\alpha,\alpha) = 8n/4 + \mbox{even
integer} \in 2 \Z$.
\end{example}
%
\begin{theorem}
Let $Q$ be an even lattice in a Euclidean space $V$.  Let
$\Delta=\{ \alpha \in Q \, | \, (\alpha,\alpha)=2 \}$ and assume
$\Delta$ spans $V$ over $\R$.  Then $(V,\Delta)$ is an abstract
root system.
\end{theorem}
%
\begin{proof}
We show that each of the requirements for an abstract root system
is met.
\begin{itemize}
\item $\Delta$ spans $V \rightarrow$ assumed \item $\Delta$ is
finite $\rightarrow$ $\Delta$ is intersection of discrete $Q$ and
compact sphere $\{ v \in V \, | \, (v,v)=2 \}$ \item $\alpha \in
\Delta \rightarrow k \alpha \in \Delta$ if $k=\pm 1$.  This is
obvious.
\end{itemize}
It remains to check the string property.  Take $\alpha,\beta \in
\Delta$.  Then, $(\alpha \pm \beta, \alpha \pm \beta) = 4 \pm
2(\alpha,\beta) \geq 0$.  Hence, $\alpha + \beta$ is a root iff
$(\alpha,\beta) = -1$, and $\alpha - \beta$ is a root iff
$(\alpha,\beta)=+1$.\qed
\end{proof}
%
All possible inner products $(\alpha,\beta)$ for $\alpha,\beta \in
\Delta$ are:
\begin{itemize}
\item +2 iff $\alpha = + \beta$ ($p=2,q=0$) \item -2 iff $\alpha =
- \beta$ ($p=0,q=2$) \item +1 iff $\alpha - \beta \in \Delta$,
$\alpha + \beta \not \in \Delta$ ($p=1,q=0$) \item -1 iff $\alpha
+ \beta \in \Delta$, $\alpha - \beta \not \in \Delta$ ($p=0,q=1$)
\item 0 iff $\alpha + \beta, \alpha - \beta \not \in \Delta$
($p=q=0$)
\end{itemize}
Hence the string property, $p-q = (\alpha,\beta)$, holds in all
cases. \qed
%
Let $\Delta_{E_8} = \{ \alpha \in \Gamma_8 | (\alpha,\alpha)=2 \}
= \{ \pm \epsilon_i \pm \epsilon_j (i \not = j), \frac{1}{2} ( \pm
\epsilon_1 \cdots \pm \epsilon_8 ) (\mbox{even \# of - signs})
\}$. Then $(V,\Delta_{E_8})$ is an abstract root system.
Moreover, $\Z\,\Delta_{E_8} = \Gamma_8$ so $\Gamma_8 = Q_{E_8}$.
%
\begin{exercise}
Show that $\Z \, \Delta_{E_8} = \Gamma_8 = Q_{E_8}$.
\end{exercise}
%
There are 240 roots in $\Delta_{E_8}$.  This makes it larger than
$\Delta_{A_8}$, $\Delta_{B_8}$, $\Delta_{C_8}$, and
$\Delta_{D_8}$.  Respectively, these have $8\cdot (8+1)=72$,
$2\cdot 8^2= 128$, $2\cdot 8^2 = 128$ and $2\cdot (8^2-8)=112$
roots.
%
\begin{exercise}
Confirm the sizes of $\Delta_{A_8}$ through $\Delta_{D_8}$.
\end{exercise}
%
The case $\Gamma_8$ is the largest exception to the otherwise
exhaustive classification into $A_r$ through $D_r$.
%
\begin{example}
Define the following as a sublattice of $Q_{E_8}$:  $Q_{E_7} = \{
\alpha \in \Gamma_{E_8} \, | \, (\rho,\alpha)=0 \}$, where $\rho =
(\frac{1}{2},\frac{1}{2},\ldots,\frac{1}{2})$ and $\Delta_{E_7} =
\{ \alpha \in Q_{E_7} \, | \, (\alpha,\alpha)=2 \}$.  This is
another exceptional algebra that does not fit into the $A_r$
through $D_r$ classification scheme.
\end{example}
%
\begin{example}
Define the following as a sublattice of $Q_{E_7}$:  $Q_{E_6} = \{
\alpha \in Q_{E_7} \, | \, (\epsilon_7 + \epsilon_8,\alpha)=0 \}$,
$\Delta_{E_6} = \{ \alpha \in Q_{E_6} \, | \, (\alpha,\alpha)=2
\}$.  This is a third exceptional algebra that does not fit into
the $A_r$ through $D_r$ classification scheme.
\end{example}
%
\begin{exercise}
Find $\Delta_{E_7}$ and $\Delta_{E_6}$.  Show they respectively
span $Q_{E_7}$ and $Q_{E_6}$ over $\Z$.  Also, show that $\left|
\Delta_{E_7} \right| = 126$ and $\left| \Delta_{E_6} \right| =
72$.
\end{exercise}
%
Another important lattice is $Q_{F_4}= \{ \sum_{i=1}^4 a_i
\epsilon_i \, | \, \mbox{all } a_i \in \Z \mbox{ or all } a_i \in
\Z + \frac{1}{2} \}$.  Let $\Delta_{F_4} = \{ \alpha \in Q_{F_4}
\, | \, (\alpha,\alpha) = 2 \mbox{ or } 1 \} = \{ \pm \epsilon_i
\pm \epsilon_j \, (i \not = j), \,\, \pm \epsilon_i , \,\,
\frac{1}{2} (\pm \epsilon_1 \pm \epsilon_2 \pm \epsilon_3 \pm
\epsilon_4 ) \} $. The vector space is $V = \bigoplus_{i=1}^{4} \R
\epsilon_i$ with $(\epsilon_i,\epsilon_j)=\delta_{ij}$.
%
\begin{exercise}
Check that $\Delta_{F_4}$ is an abstract root system and that it
spans $Q_{F_4}$.  Also check that $\left| \Delta_{F_4} \right| =
48$.
\end{exercise}
%
The last exceptional root system is $\Delta_{G_2} = \{ \alpha \in
Q_{A_2} \, | \, (\alpha,\alpha) = 2 \mbox{ or } 6 \} = \{
\epsilon_i - \epsilon_j \, (i,j=1,2,3; \, i \not = j),\, \pm
(\epsilon_i + \epsilon_j - 2\epsilon_k) \, (i \not = j \not = k
\not = i) \}$.  We can calculate that $\left| \Delta_{G_2} \right|
= 12$.
%
\begin{exercise}
Check that the string property holds for $\Delta_{G_2}$.
\end{exercise}
%
There are three distinct indecomposable root systems $(V,\Delta)$
that live in a vector space $V$ with dimension 2.  These are $A_2
= \{ \epsilon_i - \epsilon_j \, (i,j = 1,2,3; \, i \not = j) \}$,
$B_2 = \{ \pm \epsilon_i \pm \epsilon_j, \, \pm \epsilon_i \, (i
\not = j; \, i,j=1,2) \} \simeq C_2$ and $G_2$ as defined above.
They are depicted below.
%
\begin{figure}[ht]
%\includegraphics[width=14.48cm]{diagrams.pdf}
\begin{center}
\input{diagram.pstex_t}
\end{center}
\caption{Here $\{ \alpha, \beta \}$ is a subset of the simple
roots.}
\end{figure}
%
\section{Exercises and Solutions}
%
\begin{exercise}
Prove that $Q_{A_r}$ and $Q_{C_r}$ are as above.
\end{exercise}
\begin{solution} First the solution regarding $Q_{A_r}$, and then that regarding $Q_{C_r}$.
\begin{claim} $Q_{A_r} \subset \Z \, \Delta_{A_r}$ \end{claim}
\begin{proof} Go through the list for $i=1,\ldots,r+1$,
attributing $a_i \epsilon_i$ to $-a_i \epsilon_{i+1}$.  That is:

\vspace{3mm}
\begin{tabular}{l}
$a_1 \epsilon_1 \rightarrow a_1 (\epsilon_1 - \epsilon_2) + a_1
\epsilon_2$ \\
$a_2 \epsilon_2 + a_1 \epsilon_2 \rightarrow (a_1 +
a_2)(\epsilon_2-\epsilon_3) + (a_1+a_2)\epsilon_3$ \\
$(a_1+a_2+a_3)\epsilon_3 \rightarrow
(a_1+a_2+a_3)(\epsilon_3-\epsilon_4)+(a_1+a_2+a_3)\epsilon_4$ \\
\hspace{10mm} $\vdots$ \\
$(a_1 + \cdots + a_r)\epsilon_r \rightarrow (a_1+\cdots
a_r)(\epsilon_r - \epsilon_{r+1} +
(a_1+\cdots+a_r)\epsilon_{r+1} = (a_1+\cdots a_r)(\epsilon_r - \epsilon_{r+1}$\\
$(a_1+\cdots+a_{r+1}) \epsilon_{r} = 0 \cdot \epsilon_r $ \qed
\end{tabular}
\end{proof}
\begin{claim} $\Z \, \Delta_{A_r} \subset Q_{A_r}$ \end{claim}
\begin{proof}
Each $\alpha \in \Z \Delta_{A_r}$ is of the form $a (\epsilon_i -
\epsilon_j)$ for some $a \in \Z$.  Then, any linear combination of
such roots come in pairs of $\epsilon$'s, one with coefficient
$+a$ and the other with $-a$.  Thus, the sum of the coefficients
is 0 and so is contained in $\Z$.\qed
\end{proof}
%
$\Z \, \Delta_{A_r} \subset Q_{A_r}$ and $Q_{A_r} \subset \Z \,
\Delta_{A_r} \Rightarrow \Z \, \Delta_{A_r} = Q_{A_r}$. \qed
%
\begin{claim}$Q_{C_r} \subset \Z \, \Delta_{C_r}$\end{claim}
\begin{proof}
Use the same process as with the $Q_{A_r}$ proof, getting roots of
the form $a_1 (\epsilon_1 - \epsilon_2),
(a_1+a_2)(\epsilon_2-\epsilon_3),\ldots,(a_1+\cdots+a_r)(\epsilon_r
- \epsilon_{r+1})$.  We have $(a_1+\cdots+a_r)\epsilon_{r+1}$
left.  By the definition of $Q_{C_r}$, this sum $\in 2 \Z$.  This,
then, is of the form $\Z \, \Delta_{C_r}$ since $2 \epsilon_{r+1}
\in \Delta_{C_r}$.\qed
\end{proof}
%
\begin{claim}$\Z \, \Delta_{C_r} \subset Q_{C_r}$\end{claim}
\begin{proof}
Each term $\alpha \in \Z \, \Delta_{C_r}$ is of the form:
%
\begin{itemize}
\item $ a \,(+\epsilon_i + \epsilon_j) \rightarrow $ contributes
$+2$ to sum \item $a \,(-\epsilon_i + \epsilon_j) \rightarrow $
contributes $0$ to sum \item $a \,(-\epsilon_i - \epsilon_j)
\rightarrow $ contributes $-2$ to sum \item $2 a \epsilon_i
\rightarrow $ contributes $+2$ to sum
\end{itemize}
for $a \in \Z$.  Each of these possibilities contribute an even
integer to the sum of all the coefficients, showing that $\sum_i
a_i \in 2 \Z$. \qed
\end{proof}
%
Thus, $\Z \, \Delta_{C_r} \subset Q_{C_r}$ and $Q_{C_r} \subset \Z
\, \Delta_{C_r} \Rightarrow Q_{C_r} = \Z \, \Delta_{C_r}$ QED.
\end{solution}
%
%
\begin{exercise}
Show that $\Z \, \Delta_{E_8} = \Gamma_8 = Q_{E_8}$.
\end{exercise}
%
\begin{solution}
We will use a similar technique as with the previous exercise.
Recall that $\Gamma_8 = \{ \sum_{i=1}^8 a_i \epsilon_i \, | \,
\forall a_i \in \Z \mbox{ or } \forall a_i \in \Z + \frac{1}{2};
\,\, \sum_{i=1}^8 a_i \in 2 \Z \}$ and $\Delta_{E_8} = \{ \alpha
\in \Gamma_8 \, | \, (\alpha,\alpha) = 2 \} = \{ \pm \epsilon_i
\pm \epsilon_j \, (i \not = j); \,\, \frac{1}{2}(\pm \epsilon_1
\cdots \pm \epsilon_8) \mbox{ (even \# of - signs)} \}$.
%
\begin{claim}
$Q_{E_8} \equiv \Z \, \Delta_{E_8} \subset \Gamma_8$
\end{claim}
%
\begin{proof}
Clearly both general types of roots in $\Delta_{E_8}$ satisfy
$\forall a_i \in \Z$ or $\forall a_i \in \Z + \frac{1}{2}$.  Now
we check the constraint on the sum of the coefficients.
%
\begin{itemize}
\item $+\epsilon_i + \epsilon_j \rightarrow +2\Z$ \item $\pm
(\epsilon_i - \epsilon_j ) \rightarrow 0$ \item $-\epsilon_i -
\epsilon_j \rightarrow -2 \Z$ \item $\frac{1}{2}(\pm \epsilon_1
\cdots \pm \epsilon_8) \rightarrow (4,2,0,-2,-4) \, \Z \in 2 \Z$

\end{itemize}
\end{proof}
%
\begin{claim}
$\Gamma_8 \subset \Z \, \Delta_{E_8} \equiv Q_{E_8}$
\end{claim}
%
\begin{proof}
Separate elements of $\Gamma_8$ into two cases, $\forall a_i \in
\Z$ and $\forall a_i \in \Z + \frac{1}{2}$, and treat separately.
\begin{enumerate}
\item cd Decompose $\sum_{i=1}^8 a_i \epsilon_i \in \Gamma_8$ into
$a_1 (\epsilon_1 - \epsilon_2) + (a_1+a_2)(\epsilon_2-\epsilon_3)
+ \cdots + (a_1+\cdots +a_7)(\epsilon_7 - \epsilon_8) +
(a_1+\cdots + a_8)\epsilon_8$.  But we know that $a_1+\cdots + a_8
\equiv 2 S \in 2 \Z$.  Then, write $2 S \epsilon_8 = S (\epsilon_8
- \epsilon_1) + S (\epsilon_8 - \epsilon_2) + S (\epsilon_1 +
\epsilon_2)$.  We have thus expressed this element of $\Gamma_8$
as an element of $\Z \, \Delta_{E_8}$.

\item First, we know that since $\sum _{i=1}^{8} \in 2 \Z$,
$\sum_{i=1}^{8} a_i \epsilon_i$ can be written as a sum of terms
of the form $\frac{1}{2}(\pm \epsilon_1 \cdots \pm \epsilon_8)$.
Translate the problem to the ``nearest'' problem of the former
type by sending $a_i \rightarrow a_i + \frac{1}{2}$. Then, we know
that $\sum _{i=1}^{8} a_i \in 2 \Z$ by the above argument. Since
$(-\frac{1}{2},\ldots,-\frac{1}{2}) \in \Delta_{E_8}$, then $\sum
_{i=1}^{8}$ must also $\in \Z \, \Delta_{E_8}$.
\end{enumerate}
\end{proof}
Thus, $\Gamma_8 \subset \Z \, \Delta_{E_8}$ and $\Z \,
\Delta_{E_8} \subset \Gamma_8 \Rightarrow \Z \, \Delta_{E_8}
\subset \Gamma_8$. \qed
\end{solution}
%
%
\begin{exercise}
Confirm the sizes of $\Delta_{A_8}$ through $\Delta_{D_8}$.
\end{exercise}

\begin{solution}
\begin{itemize}
\item $\Delta_{A_r}$:  Recall that $\Delta_{A_r} = \{ \epsilon_i -
\epsilon_j \, | \, i,j=1,\ldots,r+1, \,\, i \not = j \}$.
%
There are $\frac{(r+1)!}{(r-1)! \, 2!}$ ways of choosing a pair of
indices $i$ and $j$. Each pair has 2 ways of being arranged around
the minus sign. Thus, $r (r+1)$ roots exist.
%
\item $\Delta_{B_r} = \{ \pm \epsilon_i \pm \epsilon_j; \, \pm
\epsilon_i \, | \, i,j=1,\ldots,r; i \not = j \}$.  The first sort
of root has $2^2$ choices of sign arrangement and
$\frac{r(r+1)}{2}$ choices of indices $i,j$.  The second sort of
root has $2$ choices of sign and $r$ choices of index.  This gives
a total of $2r(r-1) + 2r = 2r^2$ roots.
%
\item $\Delta_{C_r} = \{ \pm \epsilon_i \pm \epsilon_j; \, \pm 2
\epsilon_i \, | \, i,j=1,\ldots,r; i \not = j \}$.  Identical to
the above case.
%
\item $\Delta_{D_r} = \{ \pm \epsilon_i \pm \epsilon_j \, | \,
i,j=1,\ldots,r; i \not = j \}$.  This case just has $2r(r-1)$
roots, for reasons stated in the $\Delta_{B_r}$ case.
%
\end{itemize}
These solutions all hold in the specific case $r=8$.  \qed.
\end{solution}
%
%
\begin{exercise}
Find $\Delta_{E_7}$ and $\Delta_{E_6}$.  Show they respectively
span $Q_{E_7}$ and $Q_{E_6}$ over $\Z$.  Also, show that $\left|
\Delta_{E_7} \right| = 126$ and $\left| \Delta_{E_6} \right| =
72$.
\end{exercise}
%
\begin{solution}
$\Delta_{E_7} = \{\alpha \in Q_{E_7} \, | \, (\alpha,\alpha)=2 \}
\subset \Delta_{E_8} = \{ \pm \epsilon_i \pm \epsilon_j \, (i \not
= j), \,\, \frac{1}{2} (\pm \epsilon_1 \cdots \pm \epsilon_8) \}$.
To get $\Delta_{E_7}$ take $\Delta_{E_8}$ and impose the condition
$(\rho,\alpha)=0$.  Impose the same constraint on $Q_{E_8}$ to
obtain $Q_{E_7}$.  Since $\Delta_{E_8}$ spanned $Q_{E_8}$, it is
clear that $\Delta_{E_7}$ obtained as above will span $Q_{E_7}$
obtained this same way.

$\left| \Delta_{E_7} \right| = \frac{8!}{4! \, 4!}+2 \frac{8!}{6!
\, 2!} = 70 + 2\cdot 28 = 126$.

$\Delta_{E_6}$ is obtained from $\Delta_{E_7}$ by imposing the
condition that $(\epsilon_7 + \epsilon_8, \alpha) = 0$.  Applying
the same condition to $Q_{E_7}$ gives $Q_{E_6}$.  Then, the same
argument as above indicates that $\Delta_{E_6}$ spans $Q_{E_6}$.

Actually imposing this condition on the given expression of
$\Delta_{E_6}$ gives the alternative expression $\Delta_{E_6} = \{
\frac{1}{2}(\pm \epsilon_1 \cdots \pm \epsilon_6) \pm
\frac{1}{2}(\epsilon_7 - \epsilon_8) \mbox{ (with a total of 4 -
signs)}; \,\, \pm (\epsilon_i - \epsilon_j) \, (i,j=1,\ldots,6; \,
i \not =j); \,\, \pm (\epsilon_7 - \epsilon_8) \}$.  Then, we
calculate $\left| \Delta_{E_6} \right| = 2 \frac{6!}{3! \, 3!} + 2
\frac{6!}{2! \, 4!} + 2 = 40 + 30 + 2 = 72$.
\end{solution}
%
%
\begin{exercise}
Check that $\Delta_{F_4}$ is an abstract root system and that it
spans $Q_{F_4}$.  Also check that $\left| \Delta_{F_4} \right| =
48$.
\end{exercise}
%
\begin{solution}
$\Delta_{F_4} = \{ \alpha \in Q_{F_4} \, | \, (\alpha,\alpha) = 2
\mbox{ or } 1 \} = \{ \pm \epsilon_i \pm \epsilon_j \, (i \not =
j); \,\, \pm \epsilon_i ; \,\, \frac{1}{2}(\pm \epsilon_1 \cdots
\pm \epsilon_4) \}$ and $Q_{F_4} = \{ \sum_{i=1}^{4} a_i
\epsilon_i \, | \, \forall a_i \in \Z \mbox{ or } \forall a_i \in
\Z + \frac{1}{2} \}$.

\begin{enumerate}
\item $\left| \Delta_{F_4} \right| = 2^2 \frac{4!}{2! \, 2!} +
2\frac{4!}{3! \, 1!} + 2^4 \frac{4!}{4! \, 0!} = 48$

\item Roots in $\Delta_{F_4}$ of the first two types have integer
coefficients.  Sums of integers times these roots have only
integer coefficients, therefore.  If an odd number of roots of the
last type with half integer coefficients occur, the overall sum
has half-integer coefficients in front of the $\epsilon_i$'s.
Otherwise, they are still integers.  Thus, $Q_{F_4} \subset \Z \,
\Delta_{F_4}$.

\item Properties 1 and 2 of an abstract root system obviously
satisfied. All that remains is to show that the string property
holds.
%
\begin{itemize}
\item $\alpha = \epsilon_i + \epsilon_j, \,\, \beta = \epsilon_k -
\epsilon_m \,\, (i\not = m, \, j \not = k)$.  $\langle \beta,
\alpha \rangle = \delta_{ik} - \delta_{jm}$.  $p = \delta_{ik}$
and $q = \delta_{jm}$.

\item $\alpha = \epsilon_i + \epsilon_j, \,\, \beta = \epsilon_k$.
$\langle \beta, \alpha \rangle = \delta_{ik} + \delta_{jk}$. $p =
\delta_{ik} + \delta_{jk}$ and $q = 0$.

\item $\alpha = \epsilon_i, \,\, \beta = \epsilon_j + \epsilon_k$.
$\langle \beta, \alpha \rangle = 2\delta_{ij} + 2\delta_{ik}$. $p
= 2\delta_{ij} + 2\delta_{ik}$ and $q = 0$.

\item $\alpha = \epsilon_i, \,\, \beta = \epsilon_j - \epsilon_k$.
$\langle \beta, \alpha \rangle = 2\delta_{ij} - 2\delta_{ik}$. $p
= 2\delta_{ij}$ and $q = 2\delta_{ik}$.

\item $\alpha = \epsilon_i, \,\, \beta = \frac{1}{2} ( \pm
\epsilon_1 \cdots \pm \epsilon_4 )$. $ ( \beta, \alpha ) = \pm
\frac{1}{2} \,\, ( \mbox{+ if +} \epsilon_i \mbox{ in } \alpha
\mbox{ and - if } -\epsilon_i \mbox{ in } \alpha ) \Rightarrow
\langle \beta, \alpha \rangle = \pm 1$ with same conditions. $p =
1 ( \mbox{if +}\epsilon_i), \, 0 (\mbox{ if }-\epsilon_i)$ and $q
= 1 (\mbox{if -}\epsilon_i), \, 0 (\mbox{ if }+\epsilon_i)$.

\item $\alpha = \frac{1}{2} ( \pm \epsilon_1 \cdots \pm \epsilon_4
), \,\, \beta = \epsilon_i + \epsilon_j$. $\langle \beta, \alpha
\rangle = \mbox{sign}_{\alpha}(\epsilon_i)+
\mbox{sign}_{\alpha}(\epsilon_j)$. $p = 1 ( \mbox{if } +\epsilon_i
+ \epsilon_j \mbox{ in } \alpha, \,\, 0 ( \mbox{else} ) )$ and $q
= 1 ( \mbox{if } -\epsilon_i - \epsilon_j \mbox{ in } \alpha),
\,\, 0 (\mbox{else} )$.

\item $\alpha = \frac{1}{2} ( \pm \epsilon_1 \cdots \pm \epsilon_4
), \,\, \beta = \epsilon_i$. $\langle \beta, \alpha \rangle =
\mbox{sign}_{\alpha}(\epsilon_i)$. $p = 1 (\mbox{if } +\epsilon_i
\mbox{ in } \alpha) , \,\, 0 (\mbox{else})$ and $q = 1 (\mbox{if }
-\epsilon_i\mbox{ in } \alpha), \,\, 0 (\mbox{else})$.
\end{itemize}
%
\end{enumerate}
\end{solution}
%
%
\begin{exercise}
Check that the string property holds for $\Delta_{G_2}$.
\end{exercise}
%
\begin{solution}
$\Delta_{G_2} = \{ \alpha \in Q_{A_2} \, | \, (\alpha,\alpha) = 2
\mbox{ or } 6 \} = \{ \epsilon_i - \epsilon_j \, (i,j=1,2,3 \, i
\not = j); \,\, \pm (\epsilon_i + \epsilon_j - 2\epsilon_k) \, (i
\not = j \not = k \not = i) \}$

\begin{itemize}
\item $\alpha = \epsilon_i - \epsilon_j$, $\beta = \epsilon_i -
\epsilon_k \rightarrow \langle \beta , \alpha \rangle = 1$.  $p=2,
q=1$.

\item $\alpha = \epsilon_i - \epsilon_j$, $\beta = \epsilon_k -
\epsilon_k \rightarrow \langle \beta , \alpha \rangle = 1$.  $p=2,
q=1$.

\item $\alpha = \epsilon_i - \epsilon_j$, $\beta = \epsilon_i +
\epsilon_j - 2\epsilon_k \rightarrow \langle \beta , \alpha
\rangle = 0$. $p=0, q=0$.

\item $\alpha = \epsilon_i - \epsilon_j$, $\beta = \epsilon_i +
\epsilon_k - 2\epsilon_j \rightarrow \langle \beta , \alpha
\rangle = 3$. $p=3, q=0$.

\item $\alpha = \epsilon_i + \epsilon_j - 2\epsilon_k$, $\beta =
\epsilon_i + \epsilon_k - 2\epsilon_j \rightarrow \langle \beta ,
\alpha \rangle = -1$. $p=0, q=1$.

\item $\alpha = \epsilon_i + \epsilon_j - 2\epsilon_k$, $\beta =
\epsilon_i - \epsilon_j \rightarrow \langle \beta , \alpha \rangle
= 0$. $p=0, q=0$.

\item $\alpha = \epsilon_i + \epsilon_j - 2\epsilon_k$, $\beta =
\epsilon_i - \epsilon_k \rightarrow \langle \beta , \alpha \rangle
= 1$. $p=1, q=0$.

\item $\alpha = \epsilon_i + \epsilon_j -2 \epsilon_k$, $\beta =
\epsilon_k - \epsilon_i \rightarrow \langle \beta , \alpha \rangle
= -1$. $p=0, q=1$.

\end{itemize}
\end{solution}

\end{document}
