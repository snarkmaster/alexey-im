%% LyX 1.3 created this file.  For more info, see http://www.lyx.org/.
%% Do not edit unless you really know what you are doing.
\documentclass[english]{amsart}
\usepackage[T1]{fontenc}
\usepackage[latin1]{inputenc}
\usepackage{amssymb}

\makeatletter
%%%%%%%%%%%%%%%%%%%%%%%%%%%%%% Textclass specific LaTeX commands.
 \theoremstyle{plain}    
 \newtheorem{thm}{Theorem}[section]
 \numberwithin{equation}{section} %% Comment out for sequentially-numbered
 \numberwithin{figure}{section} %% Comment out for sequentially-numbered
 \theoremstyle{plain}
 \theoremstyle{definition}
  \newtheorem{example}[thm]{Example}
 \theoremstyle{definition}
  \newtheorem{xca}[section]{Exercise}%%Delete [section] for sequential numbering
 \theoremstyle{plain}    
 \newtheorem{lem}[thm]{Lemma} %%Delete [thm] to re-start numbering

%%%%%%%%%%%%%%%%%%%%%%%%%%%%%% User specified LaTeX commands.
\usepackage{ulem}
\usepackage[dvips]{graphicx}
\usepackage[dvips]{color}
%\usepackage{mathdots}
\newcommand{\g}{\mathfrak{g}}
\newcommand{\fa}{\mathfrak{a}}
\newcommand{\s}{\mathfrak{s}}
\newcommand{\h}{\mathfrak{h}}
\newcommand{\sut}{\mathfrak{n}}
\newcommand{\ut}{\mathfrak{b}}

\newcommand{\F}{\mathbb{F}}
\newcommand{\N}{\mathbb{N}}
\newcommand{\R}{\mathbb{R}}
\newcommand{\A}{\mathcal{A}}
\newcommand{\D}{\mathbb{D}}
\newcommand{\C}{\mathbb{C}}

\let\ker=\undefined
\DeclareMathOperator{\ker}{Ker}
\DeclareMathOperator{\im}{Im}
\DeclareMathOperator{\Z}{Z}
\DeclareMathOperator{\ad}{ad}
\DeclareMathOperator{\Mat}{Mat}
\DeclareMathOperator{\End}{End}
\DeclareMathOperator{\E}{E}
\DeclareMathOperator{\Der}{Der}
\DeclareMathOperator{\rank}{rank}
\DeclareMathOperator{\gl}{gl}
\DeclareMathOperator{\sla}{sl}
\DeclareMathOperator{\spa}{sp}
\DeclareMathOperator{\so}{so}
\DeclareMathOperator{\chr}{char}

\newcommand{\gln}{\gl_n \F}
\newcommand{\sln}{\sla_n \F}
\newcommand{\son}{\so_n \F}
\newcommand{\spn}{\spa_n \F}

\let\xca=\undefined
\newtheorem{xca}{Exercise}

\usepackage{babel}
\makeatother
\begin{document}
\title[Classification of Cartan matrices / Dynkin Diagrams]{Lecture 18: Classification of Abstract Cartan matrices / Dynkin Diagrams}

\author[Prof. Victor Ka\v{c}; Scribes: Steven Sivek, Alexey Spiridonov]{
Professor: Victor Ka\v{c}\\
Transcribed by: Steven Sivek and Alexey Spiridonov}

\maketitle


\section{Examples of Dynkin diagrams}

%
\begin{figure}
\input{alldds.pstex_t}


\caption{\label{dd}Dynkin diagrams of all the indecomposable root systems,
from top to bottom: $A_{r},B_{r},C_{r},D_{r},E_{8},E_{7},E_{6},F_{4},G_{2}.$}
\end{figure}
In the examples and exercises that follow, we will compute the Cartan
matrices (as described in the previous lecture) for the indecomposable
root systems that we have encountered earlier. We record these as
elegant Dynkin diagrams, summarized in Figure \ref{dd}. Later in
the lecture, we will prove that these are actually all the possible
indecomposable root systems. We also compute extended Dynkin diagrams
specifically for the purposes of this proof. 

In the following examples, the rank of the root system is always denoted
by $r$, and we get simple roots $\alpha_{1},\dots,\alpha_{r}$. The
lowest negative root, used in the extended Dynkin diagram is denoted
$\alpha_{0}$. Perhaps the most detailed explanation of how to compute
a Cartan matrix is given in Exercise \ref{exce7e6}. 

\begin{example}
$A_{r}$: $(\varepsilon_{i},\varepsilon_{j})=\delta_{ij}$, $\Delta=\Delta_{sl_{r+1}(\F)}=\{\varepsilon_{i}-\varepsilon_{j}|i,j\in[r+1]\}\subset V$,
where $V$ is the subspace of $\bigoplus_{i=1}^{r+1}\R\varepsilon_{i}$,
on which the sum of coordinates (in the basis $\{\varepsilon_{i}\}$)
is zero.

Take $f\in V^{*}$ given by $f(\varepsilon_{1})=r+1,f(\varepsilon_{2})=r,\dots,f(\varepsilon_{r+1})=1$;
hence, $f\ne0$ on $\Delta$, and $f$ is integer-valued on all the
roots.

Then, $\Delta_{+}=\{\varepsilon_{i}-\varepsilon_{j}|i<j\}$. What
is $\Pi$? Clearly, $\alpha=\varepsilon_{i}-\varepsilon_{i+1}\in\Pi$,
since $f(\alpha)=1$. There are $r$ simple roots altogether (since
$\Pi$ spans $V$) and $r$ such $\alpha$, so $\Pi=\{\alpha_{i}=\varepsilon_{i}-\varepsilon_{i+1}\}_{i=1}^{r}$. 

For the Cartan matrix, recall that:\[
A=\left(\frac{2(\alpha_{i},\alpha_{j})}{(\alpha_{i},\alpha_{i})}\right)_{i,j=1}^{r}=\left(\begin{array}{ccccc}
2 & -1 & 0 & \cdots & 0\\
-1 & 2 & -1 &  & \vdots\\
0 & -1 & 2 & \ddots & 0\\
\vdots &  & \ddots & \ddots & -1\\
0 & \cdots & 0 & -1 & 2\end{array}\right).\]


Hence, the Dynkin diagram in Figure \ref{dd}a.

The largest root is $\theta=\varepsilon_{1}-\varepsilon_{r+1}$, so
$\alpha_{0}=\varepsilon_{r+1}-\varepsilon_{1}$ ($\alpha_{0}$ denotes
the lowest negative root). This yields the extended Dynkin diagram
(from the matrix on roots $\alpha_{0},\alpha_{1}\dots,\alpha_{r}$,
denoted $\tilde{A}=\left(\frac{2(\alpha_{i},\alpha_{j})}{(\alpha_{i},\alpha_{i})}\right)$)
in Figure \ref{dd}a.
\end{example}
%

\begin{example}
$B_{r}$: $\Delta=\Delta_{so_{2r+1}(\F)}=\{\pm\varepsilon_{i}\pm\varepsilon_{j}(i\ne j),\pm\varepsilon_{i}\}_{i,j=1}^{r}\subset V=\bigoplus_{i=1}^{r}\R\varepsilon_{i}.$ 

Take the $f$ analogously to $A_{r}$ (but we have $r$, rather than
$r+1$ roots now): $f(\varepsilon_{1})=r,\dots,f(\varepsilon_{r})=1$. 

Then, $\Delta_{+}=\{\varepsilon_{i}+\varepsilon_{j}(i\ne j),\varepsilon_{i}-\varepsilon_{j}(i<j),\varepsilon_{i}\}$,
and $f=1$ for $\varepsilon_{i}-\varepsilon_{i+1}$ and $\varepsilon_{r}$.
Hence $\Pi=\{\varepsilon_{i}-\varepsilon_{i+1}\ (i=1,\dots,r-1),\varepsilon_{r}\}$.
The Cartan matrix is almost identical, except that $A_{r,r-1}=\frac{2(\alpha_{r},\alpha_{r-1})}{(\alpha_{r},\alpha_{r})}=2(\alpha_{r},\alpha_{r-1})=-2$,
so \[
A=\left(\begin{array}{ccccc}
2 & -1 & 0 & \cdots & 0\\
-1 & 2 & -1 &  & \vdots\\
0 & -1 & 2 & \ddots & 0\\
\vdots &  & \ddots & \ddots & -1\\
0 & \cdots & 0 & -2 & 2\end{array}\right).\]
 The lowest root is $\alpha_{0}=-(\varepsilon_{1}+\varepsilon_{2})$,
so we get the Dynkin diagrams in Figure \ref{dd}b.
\end{example}
%

\begin{example}
$C_{r}$: $\Delta=\Delta_{sp_{2r}(\F)}$. The roots are the same as
above, except $\pm\varepsilon_{i}$ becomes $\pm2\varepsilon_{i}$;
we also take the same $f$. Then, $f(\varepsilon_{i}-\varepsilon_{i+1})=1$
for $i=1,2,\dots,r-1$ (label these simple roots $\alpha_{1},\dots,\alpha_{r-1}$),
and we need an $r$th simple root. We might only possibly get $f=2$
by summing some of the preceding $\alpha_{i}$. However, it's impossible
to obtain $2\varepsilon_{r}$ this way. Hence, $\Pi=\{\varepsilon_{i}-\varepsilon_{i+1}\ (i=1,\dots,r-1),\ 2\varepsilon_{r}\}$.
The new Cartan integers are $\frac{2(\alpha_{r-1},\alpha_{r})}{(\alpha_{r-1},\alpha_{r-1})}=-2$,
and $\frac{2(\alpha_{r},\alpha_{r-1})}{(\alpha_{r},\alpha_{r})}=-1$,
which gives us: \[
A=\left(\begin{array}{ccccc}
2 & -1 & 0 & \cdots & 0\\
-1 & 2 & -1 &  & \vdots\\
0 & -1 & 2 & \ddots & 0\\
\vdots &  & \ddots & \ddots & -2\\
0 & \cdots & 0 & -1 & 2\end{array}\right),\]
 and the Dynkin diagram in Figure \ref{dd}c. The highest root is
$\theta=2\varepsilon_{1}$, so $\alpha_{0}=-2\varepsilon_{1}$, and
we get the extended Dynkin diagram in Figure \ref{dd}c.
\end{example}
%

\begin{example}
$D_{r}$: $\Delta=\Delta_{so_{2r}(\F)}=\{\pm\varepsilon_{i}\pm\varepsilon_{j}(i\ne j)\}_{i,j=1}^{r}$.
Define $f\in V^{*}$ by $f(\varepsilon_{1})=r-1,\dots,f(\varepsilon_{r})=0$.

Then $\Delta=\{\varepsilon_{i}\pm\varepsilon_{j}(i<j)\}$, and $\Pi=\{\varepsilon_{1}-\varepsilon_{2},\dots,\varepsilon_{r-1}-\varepsilon_{r},\varepsilon_{r-1}+\varepsilon_{r}\}$
since we get $f(\alpha_{i})=1$ for all $i$.

Compute \[
A=\left(\begin{array}{cccccc}
2 & -1 & 0 & \dots & \dots & 0\\
-1 & 2 & -1 &  &  & \vdots\\
0 & -1 & \ddots & \ddots & 0 & 0\\
\vdots &  & \ddots & 2 & -1 & -1\\
\vdots &  & 0 & -1 & 2 & 0\\
0 & \dots & 0 & -1 & 0 & 2\end{array}\right),\]
 so we have the Dynkin diagram in Figure \ref{dd}d.

Then, $\theta=\varepsilon_{1}+\varepsilon_{2}$, so $\alpha_{0}=-\varepsilon_{1}-\varepsilon_{2}$,
so we get the $\tilde{D_{r}}$ in Figure \ref{dd}d.
\end{example}
%

\begin{example}
$E_{8}$: $\Delta=\Delta_{E_{8}}=\{\pm\varepsilon_{i}\pm\varepsilon_{j}(i\ne j),\frac{1}{2}\underset{\text{(even number of + signs)}}{(\pm\varepsilon_{1}\pm\varepsilon_{2}\pm\dots\pm\varepsilon_{8})}\}_{i,j=1}^{8},$
with $V=\bigoplus_{i=1}^{8}\R\varepsilon_{i}$. Let $f(\varepsilon_{1})=23,f(\varepsilon_{2})=6,f(\varepsilon_{3})=5,\dots f(\varepsilon_{8})=0$;
we have an even number of odd $f(\varepsilon_{i})$, so $f$ is integer
on all roots. Then, \[
\Delta_{+}=\{\varepsilon_{i}\pm\varepsilon_{j}(i<j),\frac{1}{2}\underset{\text{(even number of + signs)}}{(\varepsilon_{1}\pm\varepsilon_{2}\pm\dots\pm\varepsilon_{8})}\}_{i,j=1}^{8}\]
 and \[
\Pi=\{\varepsilon_{2}-\varepsilon_{3},\varepsilon_{3}-\varepsilon_{4},\dots\varepsilon_{7}-\varepsilon_{8},\frac{1}{2}(\varepsilon_{1}-\varepsilon_{2}-\dots-\varepsilon_{7}+\varepsilon_{8})\}.\]
 In particular, $f(\alpha_{i})=1$ for all $i$; $\theta=\varepsilon_{1}+\varepsilon_{2}$,
so $\alpha_{0}=-\varepsilon_{1}-\varepsilon_{2}$ and we get the diagrams
$E_{8}$ and $\tilde{E}_{8}$ in Figure \ref{dd}e.
\end{example}
\begin{xca}
\label{exce7e6}From Exercise 16.4, we have:\begin{eqnarray*}
\Delta_{E_{7}} & = & \{\varepsilon_{i}-\varepsilon_{j}(1\le i\ne j\le8),\frac{1}{2}\underset{\text{4+signs}}{(\pm\varepsilon_{1}\pm\varepsilon_{2}\pm\dots\pm\varepsilon_{8})}\}\\
\Delta_{E_{6}} & = & \{\varepsilon_{i}-\varepsilon_{j}(1\le i\ne j\le6),\pm(\varepsilon_{7}-\varepsilon_{8}),\frac{1}{2}(\underset{\text{3+signs}}{\pm\varepsilon_{1}\pm\varepsilon_{2}\pm\dots\pm\varepsilon_{6}})\pm\frac{1}{2}(\varepsilon_{7}-\varepsilon_{8}))\}.\end{eqnarray*}
 Perform the same analysis as in the Examples for $E_{7}$ and $E_{6}$.
Show that their diagrams are as in Figure \ref{dd}f,g.
\end{xca}
For $E_{7}$, we will pick $f(\varepsilon_{i})$ so that all roots
have non-zero integer values. For this, we need $f(\varepsilon_{i})$
to be distinct integers (hence $\pm\varepsilon_{i}\pm\varepsilon_{j}$
are all integer and non-zero). Additionally, we want an even number
of the $f(\varepsilon_{i})$ to be odd. That way, roots of the second
type also have integer values. Take $f(\varepsilon_{2})=7,f(\varepsilon_{3})=6,\dots,f(\varepsilon_{8})=1$;
then, $\alpha_{1}=\varepsilon_{2}-\varepsilon_{3},\alpha_{2}=\varepsilon_{3}-\varepsilon_{4},\dots,\alpha_{6}=\varepsilon_{7}-\varepsilon_{8}$
are all simple roots with value $1$. We have $(\alpha_{i},\alpha_{j})=-1$
if $|i-j|=1$ and $0$ otherwise so far, and $(\alpha_{i},\alpha_{i})\equiv2$.
There is just one more simple root, and it must be of the second type.
We need to choose $f(\varepsilon_{1})$ appropriately. In order for
$f$ to be integer-valued on all roots, $f(\varepsilon_{1})$ must
be even. To avoid a zero-valued root, we will make it sufficiently
large that \[
\frac{1}{2}(f(\varepsilon_{1})-7-6-5-4+3+2+1)>0,\]
since that's the closest we can get to zero if $f(\varepsilon_{1})$
is large. Simplifying, we have: $f(\varepsilon_{1})$ even, and $f(\varepsilon_{1})>16$,
so we'll take $f(\varepsilon_{1})=18$. Then $f(\alpha_{7})=1,\alpha_{7}=\frac{1}{2}(\varepsilon_{1}-\varepsilon_{2}-\varepsilon_{3}-\varepsilon_{4}-\varepsilon_{5}+\varepsilon_{6}+\varepsilon_{7}+\varepsilon_{8})$,
and we have the last simple root. We have $(\alpha_{i},\alpha_{7})=0$
for all $1\le i\ne4\le6$, with $(\alpha_{4},\alpha_{7})=-1$ and
$(\alpha_{7},\alpha_{7})=2$. Hence, all the connections are simple,
and we have the right diagram. The highest root is clearly $\theta=\varepsilon_{1}-\varepsilon_{8}\overset{f}\mapsto17,$
so $\alpha_{0}=\varepsilon_{8}-\varepsilon_{0}$, which connects (simply)
only with $\alpha_{7}$, so we have the right extended diagram too. 

For $E_{6}$, we choose the same $f(\varepsilon_{2}),\dots f(\varepsilon_{8})$.
This gives us the following simple roots: $\alpha_{1}=\varepsilon_{2}-\varepsilon_{3},\dots,\alpha_{4}=\varepsilon_{5}-\varepsilon_{6},\alpha_{5}=\varepsilon_{7}-\varepsilon_{8}$,
with $(\alpha_{i},\alpha_{j})=-\delta_{ij}$ if $1\le i,j\le4$; all
$(\alpha_{i},\alpha_{i})=2$, and $\alpha_{5}$ doesn't connect with
any of these roots. Now, to choose the right $f(\varepsilon_{1})$,
we assume it's large and take the smallest non-negative number we
can get: \[
\frac{1}{2}(f(\varepsilon_{1})-7-6-5+4+3-2+1>0,\]
 so $f(\varepsilon_{1})>12$. As before, we want it to be even, and
so pick $f(\varepsilon_{1})=14$, which gives the last simple root
$f(\alpha_{6})=1,\alpha_{6}=\frac{1}{2}(\varepsilon_{1}-\varepsilon_{2}-\varepsilon_{3}-\varepsilon_{4}+\varepsilon_{5}+\varepsilon_{6}-\varepsilon_{7}+\varepsilon_{8})$.
$\alpha_{7}$ connects only with $\alpha_{3}$ and $\alpha_{5}$;
$(\alpha_{3},\alpha_{6})=-1,(\alpha_{5},\alpha_{6})=-1$ and $(\alpha_{6},\alpha_{6})=2$.
So, the connections are all simple, and we get the correct diagram.
The highest root in this case is $\theta=\varepsilon_{1}-\varepsilon_{6}\overset{f}\mapsto11$,
so $\alpha_{0}=\varepsilon_{6}-\varepsilon_{1}$, which connects only
with $\alpha_{4}$, which, again, gives the right extended diagram. 

\begin{xca}
From Exercises 16.5 and 16.6, we have:\begin{eqnarray*}
\Delta_{F_{4}} & = & \{\pm\varepsilon_{i}\pm\varepsilon_{j},\pm\varepsilon_{i},\frac{1}{2}(\pm\varepsilon_{1}\pm\varepsilon_{2}\pm\varepsilon_{3}\pm\varepsilon_{4})\},1\le i\ne j\le4\\
\Delta_{G_{2}} & = & \{\varepsilon_{i}-\varepsilon_{j},\pm(\varepsilon_{i}+\varepsilon_{j}-2\varepsilon_{k})\},1\le i,j,k\le3,\text{all distinct}\end{eqnarray*}
Perform the same calculations for $F_{4}$ and $G_{2}$, and show
that they have the diagrams in Figure \ref{dd}h,i.
\end{xca}
Here, we proceed exactly as in the previous exercise. For $F_{4}$,
we pick $f(\varepsilon_{1})=8,f(\varepsilon_{2})=3,f(\varepsilon_{3})=2,f(\varepsilon_{4})=1$,
which gives us the simple roots $\alpha_{1}=\varepsilon_{2}-\varepsilon_{3},\alpha_{2}=\varepsilon_{3}-\varepsilon_{4},\alpha_{3}=\varepsilon_{4},\alpha_{4}=\frac{1}{2}(\varepsilon_{1}-\varepsilon_{2}-\varepsilon_{3}-\varepsilon_{4})$,
with simple connections between $\alpha_{1}$ and each of $\alpha_{2}$,
$\alpha_{3}$, $\alpha_{4}$. Additionally, there is a double-arrow
connection from $\alpha_{2}$ to $\alpha_{3}$, so we get the correct
diagram. $\alpha_{0}=-\varepsilon_{1}-\varepsilon_{2}\overset{f}\mapsto-11$
connects up only with $\alpha_{1}$, which gets us the extended diagram.

For $G_{2}$, pick $f(\varepsilon_{1})=4,f(\varepsilon_{2})=2,f(\varepsilon_{3})=1$,
which gives us simple roots $\alpha_{1}=\varepsilon_{2}-\varepsilon_{3}$
and $\alpha_{2}=\varepsilon_{1}+\varepsilon_{3}-2\varepsilon_{2}$,
with $(\alpha_{1},\alpha_{1})=2$,$(\alpha_{2},\alpha_{2})=6$, and
$(\alpha_{1},\alpha_{2})=-3$, which gives the desired connection.
$\alpha_{0}=\varepsilon_{2}+\varepsilon_{3}-2\varepsilon_{1}\overset{f}\mapsto-5$,
with a simple connection to $\alpha_{2}$, as in the extended diagram.


\section{Classification of Dynkin Diagrams}

\begin{thm}
The Dynkin diagrams of all indecomposable Cartan matrices are $A_{r},B_{r},C_{r},D_{r},E_{8},E_{7},E_{6},F_{4},G_{2}$. 

%
\begin{figure}
\input{conntypes.pstex_t}


\caption{\label{conntype}The four Dynkin diagram connection types, corresponding
to the four types of $2\times2$ Cartan matrix minors.}
\end{figure}

\end{thm}
\begin{proof}
We have to choose connected graphs with connections of the 4 types
depicted in Figure \ref{conntype}, such that the matrix of any subgraph
has a positive determinant. In particular, our graphs contain no extended
Dynkin diagrams as induced subgraphs, since these have determinant
0. 

%
\begin{figure}
\input{simplylaced.pstex_t}


\caption{\label{simplac}Some diagrams for the simply laced case.}
\end{figure}


\paragraph*{Part 1}

Classify all {}``simply laced'' Dynkin diagrams, i.e. using only
0- or 1-edge connections (which correspond to a symmetric $A$). Such
a graph contains no cycles, since those are $\tilde{A}_{r}$. If there
are no branching points, we get $A_{r}$. Next, such a graph contains
at most 1 branching point, since otherwise it contains $\tilde{D}_{r}$.
If there is a branching point, one has at most 3 branches, since $\tilde{D}_{4}$
is the 4-star in Figure \ref{simplac}a. Thus, we are left with a
graph of the form $T_{p,q,r}(p\ge q\ge r\ge2)$ depicted in Figure
\ref{simplac}b. But the graph cannot contain any of $\tilde{E}_{6}=T_{3,3,3},\tilde{E}_{7}=T_{4,4,2},\tilde{E}_{8}=T_{6,3,2},$
hence the only possibilities are $D_{r},E_{6},E_{7},E_{8}$. Why?
Suppose $r=3$, then $p\ge q\ge3$, so it contains $T_{3,3,3}$. Hence,
$r=2$. From $\tilde{E}_{7}$, we can't have $q\ge4\Rightarrow q=2,3$,
and we get $p\le5$ likewise. 

%
\begin{figure}
\input{dblloop.pstex_t}


\caption{\label{dblloop}A loop with one double-connection.}
\end{figure}



\paragraph*{Part 2}

Classify all non-simply laced diagrams, i.e. those containing double-
or triple-edge connections (corresponding to non-symmetric $A$).
This can be done by a more complicated case analysis. However, we'd
like an effective way of ruling out diagrams like the one in Figure
\ref{dblloop}. This can be done by computing many large determinants,
but we would rather argue using the following:

\begin{xca}
Prove the following two useful lemmas:
\end{xca}
\begin{lem}
\label{lemmaa}Let \[
A=\left(\begin{array}{cc}
2 & \underline{\begin{array}{cccc}
-a & 0 & \cdots & 0\end{array}}\\
\left.\begin{array}{c}
-b\\
0\\
\vdots\\
0\end{array}\right\vert  & A_{n-1}\end{array}\right).\]
 Then, $\det A=2\det A_{n-1}-ab\det A_{n-2}$, where $A_{n-2}$ is
$A$ with the first and second rows and columns deleted. 
\end{lem}
\begin{proof}
Denote by $A\backslash(i,j)$ the matrix $A$ with the $i$th row
and $j$th column removed. In particular, $A_{n-2}=A\backslash(1,2)\backslash(1,1)$. 

To prove this lemma, we expand the determinant by minors along the
first row, getting $\det A=2\det A_{n-1}+a\det A\backslash(1,2)$;
then, we expand $\det A\backslash(1,2)$ along its first column, which
gives $\det A=2\det A_{n-1}-ab\det A_{n-2}$, as desired.
\end{proof}
\begin{lem}
\label{lemmab}Let\[
A=\left(\begin{array}{cccccc}
c_{1} & -a_{1} & 0 & \cdots & 0 & -b_{n}\\
-b_{1} & c_{2} & -a_{2} & 0 &  & 0\\
0 & -b_{2} & c_{3} & \ddots & \ddots & \vdots\\
\vdots & 0 & \ddots & \ddots & -a_{n-2} & 0\\
0 &  & \ddots & -b_{n-2} & \,\,\,\,\, c_{n-1} & -a_{n-1}\\
-a_{n} & 0 & \cdots & 0 & -b_{n-1} & c_{n}\end{array}\right).\]
Then, $\det(A-\epsilon E_{12})=\det A-\epsilon(b_{1}\det A_{n-2}+a_{2}a_{3}\dots a_{n})$.
In particular, if $\det A_{n-2}>0$, $b_{1}>0$, $a_{2}\dots a_{n}>0$,
and $\epsilon>0$, then $\det A-\epsilon E_{12}<\det A$.
\end{lem}
\begin{proof}
To prove this lemma, we expand the determinant of $A-\epsilon E_{12}$
along the second column to get \[
(a_{1}+\epsilon)\det A\backslash(1,2)+c_{2}\det A\backslash(2,2)+b_{2}\det A\backslash(3,2)=\det A+\epsilon\det A\backslash(1,2).\]
 $A\backslash(1,2)$ looks like this:\[
\left(\begin{array}{ccccc}
-b_{1} & -a_{2} & 0 & \cdots & 0\\
0 & c_{3} & \ddots & \ddots & \vdots\\
\vdots & \ddots & \ddots & -a_{n-2} & 0\\
0 & \ddots & -b_{n-2} & \,\,\,\,\, c_{n-1} & -a_{n-1}\\
-a_{n} & \cdots & 0 & -b_{n-1} & c_{n}\end{array}\right).\]
 If we expand its determinant along the first column, we get $-b_{1}\det A\backslash(1,2)\backslash(1,1)+(-1)^{n-1}a_{n}\det A\backslash(1,2)\backslash(n-1,1)$.
The first part of the expression is simply $-b_{1}\det A_{n-2}$.
As for the second part, the matrix we get is:\[
\left(\begin{array}{cccc}
-a_{2} & 0 & \cdots & 0\\
c_{3} & \ddots & \ddots & \vdots\\
\ddots & \ddots & -a_{n-2} & 0\\
\ddots & -b_{n-2} & \,\,\,\,\, c_{n-1} & -a_{n-1}\end{array}\right),\]
 which is lower triangular with a diagonal consisting of all $a_{i}$,
$2\le i\le n-1$, hence with determinant $(-1)^{n-2}a_{2}a_{3}\dots a_{n-1}$.
Putting all this together, we get the desired \[
\det A+\epsilon(-b_{1}\det A_{n-2}-a_{2}a_{3}\dots a_{n}).\]

\end{proof}
Lemma \ref{lemmab} implies that in the non-simply laced case, there
are no cycles as well. This is because $\det\tilde{A}_{r-1}=0$; so,
any other cycle with subdiagrams of positive determinant must have
determinant $<0$ by the lemma. For example, the diagram in Figure
\ref{dblloop} has $A=\tilde{A}_{4}-E_{12}$, so $\det A<0$. 

%
\begin{figure}
\input{g2cases.pstex_t}


\caption{\label{g2cases}Possible neighbors of $G_{2}$ in a diagram. }
\end{figure}


Next, looking at the extended Dynkin diagrams we calculate that if
the diagram contains $G_{2}$ (Figure \ref{dd}i) it must be $G_{2}$,
by Lemma \ref{lemmaa}. Otherwise, it must contain one of the possibilities
in Figure \ref{g2cases}, none of which have suitable determinant:
their Cartan matrices are of the form \[
A=\left(\begin{array}{ccc}
2 & -a & 0\\
-b & 2 & -1\\
0 & -3 & 2\end{array}\right)\Rightarrow A_{n-1}=\left(\begin{array}{cc}
2 & -1\\
-3 & 2\end{array}\right)\text{ and }A_{n-2}=\left(\begin{array}{c}
2\end{array}\right).\]
Hence, by Lemma \ref{lemmaa}, $\det(A)=2\det(A_{n-1})-ab\det(A_{n-2})=2(1-ab)$,
which is zero or negative since $a,b>0$. (The matrix of the second
graph in the figure is actually $A^{T}$, but all the determinants
are the same.)

It remains to look at the case when we have only simple or double
connections. Looking at the extended Dynkin diagrams, $\tilde{C}_{r}$
(in Figure \ref{dd}c) cannot be a subdiagram. The variants with flipped
arrow directions also don't work. They are obtained from the extended
Cartan matrix $\tilde{A}$ of $\tilde{C}_{r}$by replacing some of
$\tilde{A},\tilde{A}_{n-1},\tilde{A}_{n-2}$ by their transposes,
which doesn't change any of the determinants in the calculation. Thus,
by Lemma \ref{lemmaa}, their determinants are also 0.

Therefore only one double connection is possible. But, then we cannot
have branching points, since $\tilde{B}_{r}$ contains a double edge
and a branching point. So, the only remaining case is a line with
a left-right double edge, having $p$ single edges to the left, and
$q$ single edges to the right. $\tilde{F}_{4}$ has $p=2,q=1$; its
transpose has $p=1,q=2$. Hence, if the diagram contains $\tilde{F}_{4}$,
it must be $\tilde{F}_{4}$. Otherwise, either $p=0$ or $q=0$, and
we have $B_{r}$ or $C_{r}$.

\end{proof}
We have now shown that any finite-dimensional simple Lie algebra yields
one of a very restricted set of Dynkin diagrams (and hence Cartan
matrices). The next step in the classification of semisimple Lie algebras
will be to give a construction associating an abstract Cartan matrix
to a Lie algebra, and hence to prove that these four classes plus
five exceptional algebras are in fact the only finite-dimensional
simple Lie algebras.
\end{document}
