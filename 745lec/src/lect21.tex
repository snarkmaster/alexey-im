\documentclass[10pt,twoside]{article}

\usepackage{amsmath,amssymb,amsthm, fullpage}
\usepackage{graphics}
\usepackage{graphicx}
\usepackage{amsmath,amsfonts}

\newcommand{\Rr}{\mathbb R}
\newcommand{\Cc}{\mathbb C}
\newcommand{\Zz}{\mathbb Z}
\newcommand{\mg}{\mathfrak{g}}
\newcommand{\mh}{\mathfrak{h}}

\newtheorem{theorem}{Theorem}
\newtheorem{corollary}[theorem]{Corollary}
\newtheorem{lemma}[theorem]{Lemma}
\newtheorem{prop}[theorem]{Proposition}
\theoremstyle{definition}
\newtheorem{definition}[theorem]{Definition}
\theoremstyle{remark}
\newtheorem{remark}[theorem]{Remark}
\newtheorem{example}[theorem]{Example}

\newcommand{\exercise}[1]{
	\vspace{9pt}\noindent
	{\bf Exercise #1.}
}

\title{18.745: Lecture 21}
\author{Professor: Victor Ka\v{c} \\Scribe: Ruochuan Liu, Zhongtao Wu}
\date{}

\begin{document}
\maketitle
Let $(V$,$\triangle$) be a root system. Define a reflection
$r_{\alpha}\in $End$(V)$ for every $\alpha\in\triangle$ by:
$$r_{\alpha}(v)=v-\frac{2(v,\alpha)}{(\alpha,\alpha)}\alpha$$
$r_{\alpha}$ has properties:

\renewcommand{\labelenumi}{(\roman{enumi})}
\begin{enumerate}
\item $r_{\alpha}$ fixes pointwise the hyperplane
  $\tau_{\alpha}=\{u\in V|(u,\alpha)=0\}$.
\item $r_{\alpha}=-\alpha$.
\item $r_{\alpha}^{2}=1$ and $\gamma_{\alpha}\in
  O(V,(\cdot,\cdot))$, i.e $r_{\alpha}$ is invertible and
  $(\gamma_{\alpha}(v),\gamma_{\alpha}(v))=(v,v)$.
\item $r_{\alpha}(\triangle)=\triangle$.
\end{enumerate}
(i), (ii) and (iii) are obvious. (iv) follows from the string property.

\exercise{21.1 (optional)}
Show that property (iv) along with $\frac{2(\alpha, \beta)}{(\alpha, \alpha)}\in \Zz$ for all $\alpha, \beta \in \Delta$ implies the string property.

\begin{proof}

Denote $<\beta, \alpha> := \frac{2(\beta, \alpha)}{(\alpha,\alpha)}$.  Then $<\beta, \alpha>\in \Zz$ for all $\alpha, \beta \in \Delta$, and $r_{\alpha}(\beta)=\beta-<\beta,\alpha>\alpha$.

\renewcommand{\descriptionlabel}[1]{\hspace\labelsep #1:}

\begin{description}

\item[Step I]  Let $\alpha, \beta$ be nonproportional roots.  If $(\alpha, \beta)>0$, then $\alpha-\beta$ is a root.  If $(\alpha,\beta)<0$, then $\alpha+\beta$ is a root.

\item[Proof]
If  $(\alpha, \beta)>0$, then both $<\alpha, \beta>$ and $<\beta,\alpha>$ are positive.  By Cauchy's inequality, $$<\alpha, \beta><\beta,\alpha>=\frac{4(\alpha, \beta)^2}{(\alpha,\alpha)(\beta, \beta)}< 4, $$
where the inequality is strict since  $\alpha$ and $\beta$ are nonproportional roots.  Hence, at least one of $<\alpha,\beta>$, $<\beta,\alpha>$ equals 1.  If $<\alpha,\beta>=1$, then $r_\beta(\alpha)=\alpha-\beta\in \Delta$; similarly, if $<\beta, \alpha>=1$, then $\beta-\alpha\in \Delta$, hence $r_{\beta-\alpha}(\beta-\alpha)=\alpha-\beta \in \Delta$.

The case  $(\alpha, \beta)<0$ is similar.

\item[Step II]  The $\alpha$-string through $\beta$ is unbroken, i.e., if $p,q\in \Zz_+$ are the largest integers for which $\beta-p\alpha, \beta+q\alpha \in \Delta$, then $\beta+i\alpha\in \Delta$, $\forall -p\leq i \leq q$.

\item[Proof]
If not, we can find $-p \leq i <j \leq q$ such that $\beta+i\alpha\in \Delta$, $\beta+(i+1)\alpha\notin \Delta$, $\beta+(j-1)\alpha\notin \Delta$, $\beta+j\alpha\in \Delta$.  But then the claim in Step I implies both $(\beta+i\alpha, \alpha)\geq 0$, $(\beta+j\alpha, \alpha)\leq 0$.  Hence, $(\alpha, \alpha) \leq 0$.  Contradiction!

\item[Step III]  $p, q$ as in Step II, then $p-q=<\beta,\alpha>$.

\item[Proof]
Since $r_\alpha$ just adds or subtracts a multiple of $\alpha$ to any root, the string is invariant under $r_\alpha$.  In particular, $r_\alpha(\beta+q\alpha)=\beta-p\alpha$.  The left hand side is also equal to $\beta-<\beta,\alpha>\alpha-q\alpha$.  Hence, $p-q=<\beta,\alpha>$.

\end{description}
Thus, we proved the string property for $\Delta$.
\end{proof}

\begin{definition}
The {\em Weyl group} of a root system is a subgroup $W$ of
$O(V,(\cdot,\cdot))$ generated by all reflections
$\gamma_{\alpha}$ for $\alpha\in\triangle$. This is a finite group
since it permutes elements of a finite set
which spans $V$.
\end{definition}

\begin{example}
$\triangle_{A_{r}}=\{\varepsilon_{i}-\varepsilon_{j}|1\leq i,j\leq
r+1, i\neq j\}$. Let $\alpha=\varepsilon_{i}-\varepsilon_{j}$,
then
$$
r_{\alpha}(\varepsilon_{k})=\left\{
\begin{array}{r@{\quad:\quad}l}
\varepsilon_{k} & k\neq i,j\\ 
\varepsilon_{j}& k=i \\
\varepsilon_{i} & k=j 
\end{array}
\right\}
$$
Hence $r_{\alpha}=(ij)$, i.e., it just permutes
$\varepsilon_{i}$ and $\varepsilon_{j}$. Thus the Weyl group of
$A_{r}$ is $S_{r+1}$.
\end{example}

\exercise{21.2} Compute the Weyl group of the root systems $B$, $C$ and $D$.

\begin{proof}\ 
\renewcommand{\labelenumi}{(\alph{enumi})}
\begin{enumerate}
\item Root system $\Delta_{B_r}=\{\pm\epsilon_i\pm\epsilon_j \ ( i\neq j), \ \pm\epsilon_i \}$.
\begin{itemize}
\item
When $\alpha=\epsilon_i-\epsilon_j$, $r_\alpha(\epsilon_k)=\begin{cases}
\epsilon_k & \text{if $k\neq i,j$;}\\
\epsilon_j & \text{if $k=i$;}\\
\epsilon_i & \text{if $k=j$.}
\end{cases}$

\item
When $\alpha=\epsilon_i+\epsilon_j$, $r_\alpha(\epsilon_k)=\begin{cases}
\epsilon_k & \text{if $k\neq i,j$;}\\
-\epsilon_j & \text{if $k=i$;}\\
-\epsilon_i & \text{if $k=j$.}
\end{cases}$

\item
When $\alpha=\epsilon_i$,  $r_\alpha(\epsilon_k)=\begin{cases}
\epsilon_k & \text{if $k\neq i$;}\\
-\epsilon_i & \text{if $k=i$.}
\end{cases}$

\end{itemize}

Hence, the Weyl group is generated by all permutations of the set $\{ \epsilon_1, \epsilon_2,\dotsc, \epsilon_r \}$ and the operations of sign changes.  In terms of group, it is isomorphic to $S_r \times \Zz_2^r$.

\item Root system $\Delta_{C_r}=\{\pm\epsilon_i\pm\epsilon_j \ ( i\neq j), \ \pm 2\epsilon_i \}$.

We can check that $r_\alpha(\epsilon_k)$ has exactly the same form as the previous case, hence its Weyl group is also $S_r \ltimes \Zz_2^r$.

\item Root system $\Delta_{D_r}=\{\pm\epsilon_i\pm\epsilon_j, \ i\neq j \}$.

In this case, the Weyl group is generated by the first two types of reflections in (a).  Also, notice that $$r_{\epsilon_i-\epsilon_j}\circ r_{\epsilon_i+\epsilon_j}(\epsilon_k)=\begin{cases}
\epsilon_k & \text{if $k\neq i,j$;}\\
-\epsilon_i & \text{if $k=i$;}\\
-\epsilon_j & \text{if $k=j$.}
\end{cases}$$
This is the same as changing signs for a pair $(\epsilon_i, \epsilon_j)$.
Hence, each element in the Weyl group is a permutation of set $\{ \epsilon_1, \epsilon_2,\dotsc, \epsilon_r \}$, composed with an even number of sign changes.  In terms of group, it is isomorphic to $S_r \ltimes \Zz_2^{r-1}$.
\end{enumerate}
\end{proof}


\exercise{21.3}
 If $A\in O(V,(\cdot,\cdot))$, then $Ar_\alpha A^{-1}=r_{A(\alpha)}$.

\begin{proof}

Let $AT_\alpha :=\{A(u)| \ (\alpha,u)=0 \}$.  Since $A\in O(V,(\cdot,\cdot))$, we have $(A(u), A(\alpha))=(u,\alpha)=0$ for all $A(u)\in AT_\alpha$.  Hence $r_{A(\alpha)}(AT_\alpha)=AT_\alpha$.

Also, $Ar_\alpha A^{-1}(AT_\alpha)=Ar_\alpha (T_\alpha)=AT_\alpha$.  So both reflections $Ar_\alpha A^{-1}$ and $r_{A(\alpha)}$ fix the hyperplane $AT_\alpha$, hence $Ar_\alpha A^{-1}=r_{A(\alpha)}$.

\end{proof}


Recall that, given a choice of $f\in V^{*}$ such that
$f(\alpha)\neq 0$ for all $\alpha\in \triangle$, we get a subset
$\triangle_{+}=\{\alpha\in \triangle|f(\alpha)>0\}$ and
$\Pi=\{\alpha_{1},\alpha_{2},\ldots,\alpha_{r}\}$ of simple roots
such that any $\alpha\in \triangle_{+}$ is of the form
$\alpha=\Sigma_{i}k_{i}\alpha_{i}$, $k_{i}\in \mathbb{Z}_{\geq
0}$. The reflections $s_{i}=r_{\alpha_{i}}$ are called {\em simple
reflections}. We use the notation $\operatorname{height}(\alpha):=\sum_{i}k_{i}$.

\begin{theorem}\ 
\renewcommand{\labelenumi}{(\alph{enumi})}
\begin{enumerate}
\item $\triangle_{+}\backslash \{\alpha_{i}\}$ is
$s_{i}$-invariant.

\item If $\alpha\in \triangle_{+}\backslash \Pi$, there
is a $s_{i}$ such that height$(s_{i}(\alpha))<$height$(\alpha)$.

\item If $\alpha\in \triangle_{+}\backslash \Pi$, we can
choose a sequence of simple reflections
$s_{i_{1}},\ldots,s_{i_{k}}$ such that $s_{i_{1}}\ldots
s_{i_{k}}(\alpha)\in \Pi$ and $s_{i_{j}}\ldots
s_{i_{k}}(\alpha)\in \triangle_{+}$ for each $1\leq j\leq k$.

\item $W$ is generated by simple reflections.
\end{enumerate}

\end{theorem}


\begin{proof}\ 
\renewcommand{\labelenumi}{(\alph{enumi})}
\begin{enumerate}
\item Applying simple reflection $s_{i}$ changes sign of at
most one coefficient $k_{i}$ of $\alpha\in\triangle_{+}$. If
$k_{i}$ changes to negative, then $s_{i}(\alpha)$ wouldn't be a root,
hence
$s_{i}(\alpha)\in \triangle_{+}$.

\item If $\operatorname{height}(s_{i}(\alpha))$ doesn't decrease for all
$s_{i}$, then from
$s_{i}(\alpha)=\alpha-\frac{2(\alpha,\alpha_{i})}{(\alpha_{i},\alpha_{i})}\alpha_{i}$
we get $(\alpha,\alpha_{i})\leq0$ for all $i$. Hence
$(\alpha,\alpha)=\Sigma_{i}k_{i}(\alpha,\alpha_{i})\leq0$, a contradiction! 

\item follows from $(b)$ and $(a)$.

\item Denote $W^{'}$ the subgroup of $W$ generated
by simple reflections. By $(c)$, for any $\alpha\in\triangle_{+}$,
there exists $w\in W^{'}$ such that $w(\alpha)=\alpha_{i}\in \Pi$
for some $i$. By Ex21.3,
$wr_{\alpha}w^{-1}=r_{w(\alpha)}$, hence
$r_{\alpha}=w^{-1}s_{i}w \in W^{'}$.

\end{enumerate}
\end{proof}


Consider $V\backslash
\bigcup_{\alpha\in\triangle_{+}}T_{\alpha}=\coprod_{j}C_{j}$,
where $C_{j}$ are connected components of this set. $C_{j}$'s are
called open chambers, $\overline{C_{j}}$'s are called (closed)
chambers. Also, define the {\em fundamental chamber}: $\overline{C}=\{v\in
V|(\alpha_{i},v)\geq 0$ $i=1,\ldots,r\}$.

\exercise{21.4}
Show that $T_\alpha \cap C=\varnothing$, where $C=\{ v\in V | \ (\alpha_i, v)>0, i=1,\dotsc,r\}$ is the open fundamental chamber.  Hence the fundamental chamber is a chamber.

\begin{proof}
Suppose $v\in T_\alpha \cap C$, then $(v,\alpha)=0$ and $(v,\alpha_i)>0$ for $i= 1,\dotsc, r$.  But $\alpha=\sum_{i=1}^{r} k_i\alpha_i$, where $k_i\in \Zz_+$.  Hence, $k_i=0$ for all $i$, and consequently $v=0$.

So we proved $T_\alpha \cap C=\varnothing$.  Since $C$ is connected, $C \subset C_j$ for some $j$.  On the other hand, $(\alpha_i,v)\neq 0$, $\forall i, v\in C_j$ by definition.  And since the inner product is a continuous function of $v$, we conclude that $(\alpha_i, v)>0$ $\forall v\in C_j$.  Hence $C_j\subset C$.

Thus, $\bar{C}=\bar{C_j}$, i.e, the fundamental chamber is a chamber.

\end{proof}


\begin{theorem}\ 
\renewcommand{\labelenumi}{(\alph{enumi})}
\begin{enumerate}
\item $W$ permutes all chambers transitively, i.e for any
chamber $\overline{C_{1}}$ and $\overline{C_{2}}$, there exists
$w\in W$ such that $w\overline{C_{1}}=\overline{C_{2}}$.

\item Let $\triangle_{+}$ and $\triangle_{+}^{'}$ be
subsets of positive roots of $\triangle$ defined by $f$ and
$f^{'}$ respectively . Then there exists $w\in W$ such that
$w(\triangle_{+})=\triangle_{+}^{'}$. In particular, the Cartan
matrix is independent of the choice of $f$.
\end{enumerate}
\end{theorem}

\begin{proof}\ 
\renewcommand{\labelenumi}{(\alph{enumi})}
\begin{enumerate}
\item Choose $P_{i}\in C_{i}$ ($i=1,2$) such that the
interval $[P_{1},P_{2}]$ doesn't intersect any of
$\tau_{\alpha}\cap\tau_{\beta}$, where $\alpha$,
$\beta\in\triangle_{+}$ and $\alpha\neq\beta$. Now move along the
interval $[P_{1},P_{2}]$ until we hit a hyperplane
$\tau_{\alpha}$. Apply reflection $r_{\alpha}$ to
$\overline{C_{1}}$. Keep moving and applying reflections until we
reach $\overline{C_{2}}$.

\item $\triangle_{+}$ and $\triangle_{+}^{'}$ define
fundamental chamber $\overline{C}$ and $\overline{C^{'}}$
respectively. By $(a)$, there exists $w\in W$ such that
$w(\overline{C})=\overline{C^{'}}$. Hence
$w(\triangle_{+})=w(\triangle_{+}^{'})$, since
$\overline{C}=\{v\in V|(\alpha_{i},v)\geq 0$ $i=1,\ldots,r\}$.
\end{enumerate}
\end{proof}



\begin{definition}
 Fix $\Pi\subset\triangle_{+}\subset\triangle$, then we
have simple reflections $s_{1},\ldots,s_{r}\in W$. Given $w\in W$,
an expression $w=s_{i_{1}}\ldots s_{i_{l}}$ is called {\em reduced} if
$l$ is minimal possible. We let $l=l(w)$ called the length of $w$.
Note that $\det w=(-1)^{l(w)}$ since $\det s_{i}=-1$.
\end{definition}

\begin{example}
 $l(s_{i})=1$, $l(s_{i}s_{j})=2$ if $i\neq j$,
but $l(s_{i}^{2})=0$.
\end{example}

\begin{lemma} [Exchange Lemma]
 Suppose that $s_{i_{1}}\ldots
s_{i_{t-1}}(\alpha_{i_{t}})\in \triangle_{-}$, then the expression
$w=s_{i_{1}}\ldots s_{i_{t}}$ is not reduced. Namely, there exists
$1\leq r<t$ such that $w=s_{i_{1}}\ldots
s_{i_{r-1}}s_{i_{r+1}}\ldots s_{i_{t-1}}$.

\end{lemma}

\begin{proof}
Consider the roots $\beta_{k}=s_{i_{k+1}}\ldots
s_{i_{t-1}}(\alpha_{i_{t}})$ for $0\leq k\leq t-1$. Then
$\beta_{0}\in\triangle_{-}$ by assumption and
$\beta_{t-1}=\alpha_{i_{t}}\in\triangle_{+}$. Hence there exists
$0\leq r\leq t-1$ such that $\beta_{r-1}\in\triangle_{-}$ and
$\beta_{r}\in\triangle_{+}$. But by definition
$\beta_{r}=s_{i_{r}}(\beta_{r})$, hence
$\beta_{r}=\alpha_{i_{r}}$. Recall that, by definition,
$\beta_{r}=s_{i_{r+1}}\ldots
s_{i_{t-1}}(\alpha_{i_{t}})=\alpha_{i_{r}}$. Thus if we denote
$\overline{w}=s_{i_{r+1}}\ldots s_{i_{t-1}}$, using Ex21.3 we see
that $\overline{w}s_{i_{t}}\overline{w}^{-1}=s_{i_{r}}$, thus
$\overline{w}s_{i_{t}}=s_{i_{r}}\overline{w}$. Now multiplying both
sides by $s_{i_{1}}\ldots s_{i_{r}}$, we get the result.
\end{proof}

\begin{corollary}
 $W$ acts simply transitive on chambers, i.e if
$w(\overline{C})=\overline{C}$, then $w=1$. In particular if
$\lambda\in C$(open chamber), $w(\lambda)=\lambda$, then
$w=1$.
\end{corollary}

\begin{proof}
 If $w\neq1$, take its reduced expression:
$w=s_{i_{1}}\ldots s_{i_{l}}$. If $w(\overline{C})=\overline{C}$,
then $w(\Pi)=\Pi$, in particular $w(\alpha_{i_{l}})\in
\triangle_{+}$. On the other hand,
$w(\alpha_{i_{l}})=s_{i_{1}}\ldots s_{i_{l-1}}(-\alpha_{i_{l}})\in
\triangle_{+}$ means $s_{i_{1}}\ldots
s_{i_{l-1}}(\alpha_{i_{l}})\in \triangle_{-}$, hence by exchange
lemma $s_{i_{1}}\ldots s_{i_{l}}$ is not a reduced expression.
That's a contradiction!
\end{proof}






\end{document}
