\documentclass[11pt]{article}
\usepackage{cancel}
\usepackage{amsmath}
\usepackage{amssymb}
\usepackage{epsfig}
\usepackage{pstricks}

\newcommand{\handout}[5]{
  \noindent
  \begin{center}
  \framebox{
    \vbox{
      \hbox to 5.78in { {\bf 18.745 Introduction to Lie Algebras } \hfill #2 }
      \vspace{4mm}
      \hbox to 5.78in { {\Large \hfill #5  \hfill} }
      \vspace{2mm}
      \hbox to 5.78in { {\em #3 \hfill #4} }
    }
  }
  \end{center}
  \vspace*{4mm}
}

\newcommand{\lecture}[4]{\handout{#1}{#2}{#3}{Scribes: #4}{Lecture #1}}


\DeclareSymbolFont{AMSb}{U}{msb}{m}{n}
\DeclareMathSymbol{\N}{\mathbin}{AMSb}{"4E}
\DeclareMathSymbol{\Z}{\mathbin}{AMSb}{"5A}
\DeclareMathSymbol{\R}{\mathbin}{AMSb}{"52}
\DeclareMathSymbol{\Q}{\mathbin}{AMSb}{"51}
\DeclareMathSymbol{\I}{\mathbin}{AMSb}{"49}
\DeclareMathSymbol{\C}{\mathbin}{AMSb}{"43}
\DeclareMathSymbol{\F}{\mathbin}{AMSb}{"46}

\newcommand{\sll}{\mbox{sl}}
\newcommand{\gl}{\mbox{gl}}
\newcommand{\GL}{\mbox{GL}}
\newcommand{\tr}{\mbox{tr\ }}
\newcommand{\Mat}{\mbox{Mat}}
\newcommand{\Lie}{\mbox{Lie}}
\newcommand{\Der}{\mbox{Der\ }}
\newcommand{\End}{\mbox{End\ }}
\newcommand{\ad}{\mbox{ad\ }}
\newcommand{\im}{\mbox{im\ }}
\newcommand{\Ker}{\mbox{ker\ }}

\newcommand{\g}{\mathfrak{g}}
\newcommand{\h}{\mathfrak{h}}
\newcommand{\m}{\mathfrak{m}}
\newcommand{\He}{\mathfrak{H}}

\newcommand{\sk}{\vspace*{1em}}

\newtheorem{defn}{Definition}
\newtheorem{remark}{Remark}
\newtheorem{example}{Example}
\newtheorem{proof}{Proof}
\newtheorem{prop}{Proposition}

% 1-inch margins, from fullpage.sty by H.Partl, Version 2, Dec. 15, 1988.
\topmargin 0pt
\advance \topmargin by -\headheight
\advance \topmargin by -\headsep
\textheight 8.9in
\oddsidemargin 0pt
\evensidemargin \oddsidemargin
\marginparwidth 0.5in
\textwidth 6.5in

\parindent 0in
\parskip 1.5ex
%\renewcommand{\baselinestretch}{1.25}

\begin{document}

\lecture{24}{Fall 2004}{Prof.\ Victor Ka\v{c}}{Maksim Lipyanskiy and Yakov Shapiro}

For the last two lectures ${\mathfrak g}$ will be a finite-dimensional semisimple Lie algebra over an algebraically closed field $\mathbb{F}$ of characteristic $0$.

Choose a Cartan subalgebra ${\mathfrak h}$ in ${\mathfrak g}$. Let $\Delta \subset {\mathfrak h}^*$ be the set of roots, $\Delta_+$ the subset of positive roots. Let ${\mathfrak g}= {\mathfrak h} \oplus (\bigoplus_{\alpha \in \Delta} {\mathfrak g}_\alpha)$ be the root space decomposition of ${\mathfrak g}$. 

Recall that ${\mathfrak g}_\alpha = \mathbb{F}E_\alpha$. Let ${\mathfrak n}_+ = \bigoplus_{\alpha \in \Delta_+} {\mathfrak g}_\alpha$, and ${\mathfrak n}_- = \bigoplus_{\alpha \in \Delta_-} {\mathfrak g}_\alpha$. We then have the \emph{triangular decomposition} ${\mathfrak g}={\mathfrak n}_- \oplus {\mathfrak h} \oplus {\mathfrak n}_+$. Let ${\mathfrak b} = {\mathfrak h} \oplus {\mathfrak n}_+$; ${\mathfrak b}$ is called a \emph{Borel subalgebra} of ${\mathfrak g}$. Note that $[{\mathfrak b}, {\mathfrak b}] = {\mathfrak n}_+$. Since ${\mathfrak n}_+$ is a nilpotent subalgebra of ${\mathfrak g}$, ${\mathfrak b}$ is a solvable subalgebra.

If $H_1,\, H_2,\, \ldots \, H_r$ is a basis of ${\mathfrak h}$, then $\{E_{-\beta} \, (\beta \in \Delta_+),\,  H_i \, (i=1,\, 2,\, \ldots \, r),$ $ E_{\beta} \, (\beta \in \Delta_+)\}$ form an ordered basis of ${\mathfrak g}$ if we choose an ordering on the positive roots $\beta_1,\, \beta_2,\, \ldots\, \beta_N$ (where $r+2N= \dim {\mathfrak g}$). Then, by the Poincar\'{e}-Birkhoff-Witt theorem, the elements
$$
E_{-\beta_1}^{m_1}E_{-\beta_2}^{m_2} \ldots E_{-\beta_N}^{m_n} h_1^{s_1} h_2^{s_2} \ldots h_r^{s_r} E_{\beta_1}^{n_1} E_{\beta_2}^{n_2} \ldots E_{\beta_r}^{n_r} \quad (m_i,\, s_i,\, n_i \in \mathbb{Z}_{+}) 
$$
form a basis of the universal enveloping algebra $U({\mathfrak g})$ of ${\mathfrak g}$. (When all $m_i$, $s_i$, and $n_i$ are zero, the product is $1$).

A \emph{highest weight ${\mathfrak g}$-module} with highest weight $\Lambda \in {\mathfrak h}^*$ is defined by the property that it contains a non-zero vector $v_\Lambda$ such that
\begin{enumerate}
\item[(1)]  $h v_\Lambda = \Lambda(h) v_\Lambda$ for all $h \in {\mathfrak h}$.
\item[(2)] ${\mathfrak n}_+ v_\Lambda = 0$
\item[(3)] $U({\mathfrak g})v_\lambda = V$
\end{enumerate}
By the above description of the basis of $U({\mathfrak g})$, properties (1) and (2) imply that (3) is equivalent to the following:
\begin{enumerate}
\item[(3')] $U({\mathfrak n}_-)v_\Lambda = V$
\end{enumerate}

For $\mu \in {\mathfrak h}^*$, the weight subspace of a ${\mathfrak g}$-module $V$ is given by $V_\mu = \{ v \in V : h v =  \mu (h)v \mbox{ for all } h \in {\mathfrak h} \}$. If $V_\mu$ is non-zero, $\mu$ is called a weight of the ${\mathfrak g}$-module $V$.

A non-zero vector $v \in V_\mu$ is called \emph{singular} if ${\mathfrak n}_+ v = 0$. If one exists, $\mu$ is called a \emph{singular weight}.

\textbf{Example.} Any $\Lambda \in {\mathfrak h}^*$ is a singular weight of a highest weight ${\mathfrak g}$-module with highest weight $\Lambda$.

\textbf{Notation.} Given $\Lambda \in {\mathfrak h}^*$, let $D(\Lambda) = \{ \Lambda - \sum_{i=1}^r k_i \alpha_i : k_i \in \mathbb{Z}_+ \} \subset {\mathfrak h}^*$, where $\Pi = \{ \alpha_1,\, \alpha_2,\, \ldots \, \alpha_r \}$ is the set of simple roots of ${\mathfrak g}$.

\begin{prop} Let $V$ be a highest weight ${\mathfrak g}$-module with highest weight $\Lambda$. Then
\begin{enumerate}
\item[(a)] $V = \bigoplus_{\lambda \in D(\Lambda)}V_\lambda$
\item[(b)] $V_\Lambda = \mathbb{F}v_\Lambda$ and $\dim V_\lambda < \infty$
\item[(c)] $V$ is an irreducible ${\mathfrak g}$-module if and only if $\Lambda$ is its only singular weight.
\item[(d)] $V$ contains a unique proper maximal submodule.
\item[(e)] If $v$ is a singular vector with weight $\lambda$, then $\Omega (v) = (\lambda + 2\rho, \lambda)v$, where $(\cdot \, , \cdot )$ is a non-degenerate symmetric invariant bilinear form on ${\mathfrak g}$, $\Omega$ is the corresponding Casimir operator, and $2\rho = \sum_{\alpha \in \Delta_+} \alpha$.
\item[(f)] $\Omega|_V = (\Lambda + 2\rho, \Lambda) Id_V$
\item[(g)] If $\lambda$ is a singular weight, then $(\lambda + \rho, \lambda + \rho)=(\Lambda + \rho, \Lambda + \rho)$.
\item[(h)] If $\Lambda \in {\mathfrak h}^*_{\mathbb{R}} = \mathbb{R} \Delta$, then $V$ has finitely many singular weights.
\end{enumerate}
\end{prop}

\emph{Proof.} (a) and (b) follow from the property (3'), which shows that any vector in $V$ is a linear combination of elements of the form $E_{-\beta_{i_1}}E_{-\beta_{i_2}} \ldots E_{-\beta_{i_k}}v_\Lambda$, and each such element has weight $\Lambda - \beta_{i_1}-\beta_{i_2} - \ldots - \beta_{i_k}$.

(c) By a lemma from lecture 19, for any ${\mathfrak g}$-submodule $U$ of $V$
\begin{equation} \label{TwentyFour_Sum}
U = \bigoplus_{\lambda \in D(\Lambda)} (U \cap V_\lambda)
\end{equation}
Now let $\lambda = \Lambda - \sum_i k_i \alpha_i$, where all $k_i$-s are in $\mathbb{Z}_+$. Choose $\lambda$ to be an element of $D(\Lambda)$ of minimal height for which $U \cap V_\lambda \neq 0$. Let $v \in (U \cap V_\lambda)$ be non-zero. Then for any $\alpha \in \Delta_+$ we will $E_\alpha v \in V_{\lambda + \alpha}$, and $E_\alpha v$ has a weight smaller than the weight of $v$. Therefore, by our choice of $\lambda$, $E_\alpha v = 0$ for all $\alpha \in \Delta_+$, so $v$ must be a singular vector.

Conversely, if $v$ is a singular vector of weight $\lambda$, then $U({\mathfrak g})v = U({\mathfrak n}_-)v$, and this is a proper submodule of $V$ unless $\lambda = \Lambda$.

(d) also follows from (\ref{TwentyFour_Sum}), because the sum of all proper submodules of $V$ satisfies (\ref{TwentyFour_Sum}) and does not contain $v_\lambda$, so it is the only maximal proper submodule.

(e) Recall that $\Omega = \sum_i u_i v_i$, where $(u_i, v_j)=\delta_{ij}$, $\{ u_i \}$ is a basis of ${\mathfrak g}$ and $\{ v_i \}$ is the dual basis. Since $\Omega$ is independent on the choice of basis, we can make any selection we want. We will use $\{ E_\alpha \ (\alpha \in \Delta_+),\ E_{-\alpha} \ (\alpha \in \Delta_+), \ H_i \ (i=1,2, \ldots n )\}$ as a basis $\{ u_i \}$ for ${\mathfrak g}$. The dual basis is $\{ E_{-\alpha} \ (\alpha \in \Delta_+),\ E_{\alpha} \ (\alpha \in \Delta_+), \ H^i \ (i=1,2, \ldots n )\}$, where $(E_\alpha, E_{-\alpha}) = 1$ for all $\alpha \in \Delta_+$ and $(H_i, H^j)=\delta_{ij}$. We now have
$$
\Omega = \sum_{\alpha \in \Delta_+}(E_\alpha E_{-\alpha} + E_{-\alpha}E_\alpha) + \sum_i H_i H^i = 2 \sum_{\alpha \in \Delta_+} E_{-\alpha}E_\alpha + 2\nu^{-1}(\rho) + \sum_i H_iH^i
$$
because $[E_{-\alpha}, E_\alpha]=(E_{-\alpha}, E_\alpha)\nu^{-1} (\alpha) = \nu^{-1}(\alpha)$.

Therefore, if $v$ is a singular vector with weight $\lambda$, then $\Omega (v) = 2 \lambda (\nu^{-1}(\rho)) + \sum_i\lambda(H_i) \lambda (H^i) = 2 (\lambda, \rho) + (\lambda, \lambda)$. This proves (e).

(f) By (e), if $v_\Lambda$ is the highest weight vector in $V_\Lambda$, then $\Omega v_\Lambda = (\Lambda + 2\rho, \Lambda) v_\Lambda$. But any other vector in $V$ has the form $g v_\Lambda$, where $g \in U({\mathfrak g})$. However, $\Omega$ commutes with ${\mathfrak g}$ and hence with $U({\mathfrak g})$. So $\Omega(gv)=g\Omega(v)$. Therefore, $\Omega (g v_\Lambda) = g \Omega (v_\Lambda) = (\Lambda + 2\rho, \Lambda) g v_\Lambda$. Thus (f) holds.

(g) follows from (e) and (f): together they imply that $(\Lambda + 2\rho, \Lambda) = (\lambda + 2\rho, \lambda)$.

(h) All weights of $V$ lie in $D_\Lambda$, hence they also lie in ${\mathfrak h}^*_{\mathbb{R}}$. So the set of singular weights must lie in the intersection of a discrete subset $D(\Lambda)$ of the Euclidean space with the compact subset given by $\{\lambda| \space |\lambda + \rho|^2 = |\Lambda + \rho|^2 \}$. Hence the set of singular weights is contained in a finite set. \hfill $\Box$

A \emph{Verma module} with highest weight $\Lambda \in {\mathfrak h}^*$, denoted by $M(\Lambda)$, is a highest weight module with highest weight $\Lambda$ such that any other highest weight module with highest weight $\Lambda$ is a quotient of $M(\Lambda)$.

\begin{prop}
\begin{enumerate}
\item[(a)] For any $\Lambda \in {\mathfrak h}^*$, $M(\Lambda)$ exists and is unique up to isomorphism.
\item[(b)] The vectors $E_{-\beta_1}^{k_1} E_{-\beta_2}^{k_2} \ldots E_{-\beta_n}^{k_n}$, where $k_i \in \mathbb{Z}_+$ and $\Delta_+ = \{ \beta_1, \beta_2, \ldots \beta_n \}$, form a basis of $M(\Lambda)$.
\item[(c)] $M(\Lambda)$ has a unique irreducible quotient, $L(\lambda) = M(\lambda)/N(\lambda)$, where $N(\lambda)$ is the unique maximum submodule of $M(\lambda)$.
\item[(d)] $M(\Lambda) \cong M(\Lambda ')$ (respectively $L(\Lambda) \cong L(\Lambda')$) if and only if $\Lambda = \Lambda'$.
\end{enumerate}
\end{prop}

\emph{Proof.} (a) We construct $M(\Lambda)$ as $U({\mathfrak g}) / U({\mathfrak g}) <{\mathfrak n}_+, h- \Lambda (h) \ \ (h \in {\mathfrak h})>$, where ${\mathfrak g}$ acts by multiplication on the left, and the highest weight vector $v_\Lambda$ is the image of $1$ in $U({\mathfrak g})$. Uniqueness is proved by universality.

(b) is clear from the description of the basis of $U({\mathfrak g})$, which shows that no non-zero linear combination of the vectors from (b) can belong to the left ideal described above.

(c) follows from part (d) of the previous proposition.

(d) follows from (c), since the set of roots of $M(\Lambda)$ is $D(\Lambda)$, but $D(\Lambda)= D(\Lambda')$ iff $\Lambda = \Lambda'$. The second part of (d) also follows from (c), because $L(\Lambda)$ determines its highest weight $\Lambda$ uniquely. \hfill $\Box$

\textbf{Example.} For $sl_2 = <E, F, H>$, ${\mathfrak h} = \mathbb{C}H$ and each $\Lambda \in {\mathfrak h}^*$ is just given by a number $\Lambda=\Lambda(H)$. Also, we have $\rho(H)=1$. Then $M(\Lambda)$ is the span of $v_\Lambda,\, Fv_\Lambda,\, F^2 v_\Lambda,\, \ldots $. $F$ acts on it in the obvious way, and the action of $E$ and $H$ is determined by the key lemma: $E(F^k v_\Lambda) = k (\Lambda - k + 1)F^{k-1} v_\Lambda$ and $H (F^k v) = (\Lambda - 2k)v_\Lambda$. By the key lemma, $M(\Lambda)$ is irreducible unless $\Lambda \in \mathbb{Z}_+$. If $\Lambda \in \mathbb{Z}_+$, then $F^{k+1} v_\Lambda$ is a singular vector, and ${\cal F}(\Lambda) = < F^{\Lambda + i}v_\Lambda : i=1,\, 2,\, \ldots >$ is a submodule of $M(\Lambda)$, isomorphic to the Verma module with highest weight $\Lambda - 2 (\Lambda + 1) = -\Lambda -2$ .

\end{document} 
