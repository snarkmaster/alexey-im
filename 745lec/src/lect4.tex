\documentclass[11pt]{article}
\usepackage{cancel}
\usepackage{amsmath}
\usepackage{amssymb}
\usepackage{epsfig}
\usepackage{pstricks}

\newcommand{\handout}[5]{
  \noindent
  \begin{center}
  \framebox{
    \vbox{
      \hbox to 5.78in { {\bf 18.745 Introduction to Lie Algebras } \hfill #2 }
      \vspace{4mm}
      \hbox to 5.78in { {\Large \hfill #5  \hfill} }
      \vspace{2mm}
      \hbox to 5.78in { {\em #3 \hfill #4} }
    }
  }
  \end{center}
  \vspace*{4mm}
}

\newcommand{\lecture}[4]{\handout{#1}{#2}{#3}{Transcribed by: #4}{#1}}


\DeclareSymbolFont{AMSb}{U}{msb}{m}{n}
\DeclareMathSymbol{\N}{\mathbin}{AMSb}{"4E}
\DeclareMathSymbol{\Z}{\mathbin}{AMSb}{"5A}
\DeclareMathSymbol{\R}{\mathbin}{AMSb}{"52}
\DeclareMathSymbol{\Q}{\mathbin}{AMSb}{"51}
\DeclareMathSymbol{\I}{\mathbin}{AMSb}{"49}
\DeclareMathSymbol{\C}{\mathbin}{AMSb}{"43}
\DeclareMathSymbol{\F}{\mathbin}{AMSb}{"46}

\newcommand{\GL}{\mbox{GL}}
\newcommand{\tr}{\mbox{tr\ }}
\newcommand{\Mat}{\mbox{Mat}}
\newcommand{\Lie}{\mbox{Lie}}
\newcommand{\Der}{\mbox{Der\ }}
\newcommand{\End}{\mbox{End\ }}
\newcommand{\ad}{\mbox{ad\ }}
\newcommand{\im}{\mbox{im\ }}
\newcommand{\Ker}{\mbox{ker\ }}

\newcommand{\sll}{\ensuremath{\mathfrak{sl}}}
\newcommand{\gl}{\ensuremath{\mathfrak{gl}}}
\newcommand{\g}{\ensuremath{\mathfrak{g}}}
\newcommand{\h}{\ensuremath{\mathfrak{h}}}
\newcommand{\m}{\ensuremath{\mathfrak{m}}}
\newcommand{\He}{\ensuremath{\mathcal{H}}}
\newcommand{\be}{\ensuremath{\mathfrak{b}_2}}
\newcommand{\bk}{\ensuremath{\mathfrak{b}}}
\newcommand{\nk}{\ensuremath{\mathfrak{n}}}
%\renewcommand{\ad}[1]{\ensuremath{{\bf ad}(#1)}}

\newcommand{\sk}{\vspace*{1em}}

\newtheorem{defn}{Definition}
\newtheorem{remark}{Remark}
\newtheorem{example}{Example}
\newtheorem{proof}{Proof}
\newtheorem{prop}{Proposition}

% 1-inch margins, from fullpage.sty by H.Partl, Version 2, Dec. 15, 1988.
\topmargin 0pt
\advance \topmargin by -\headheight
\advance \topmargin by -\headsep
\textheight 8.9in
\oddsidemargin 0pt
\evensidemargin \oddsidemargin
\marginparwidth 0.5in
\textwidth 6.5in

\parindent 0in
\parskip 1.5ex
%\renewcommand{\baselinestretch}{1.25}

\begin{document}

\lecture{Lecture 4: Nilpotent and Solvable Lie Algebras}{Fall 2004}{Prof.\ Victor
Ka\v{c}}{Anthony Tagliaferro}




{\bf Definition}
Given a Lie algebra \g, define the following descending sequences
of ideals of \g;
\begin{align}
\text{(central series)}~~~~~ \g \supset [\g,\g] &= \g^1 \supset 
[[\g,\g],\g]=\g ^3 \supset [\g ^3,\g] = \g^4 
\supset \dots \supset [\g ^{k-1},\g] = \g^{k} \supset \dots \\
\text{(derived series)}~~~~~ \g \supset [\g,\g] &= \g' \supset [\g',\g']=\g''
\supset \dots \supset [\g^{(k-1)},\g^{(k-1)}] = g^{(k)} \supset \dots
\end{align}

{\bf Exercise 4.1} Show that: \\
a) All members of the central and derived series are ideals of  \g \\
b) $\g^{k+1} \supset \g^{(k)}$

\paragraph{Solution}

a) First, let's show that \g$^k$ is an ideal of  \g. Notice that:
\begin{equation}
[\g^k,\g]=\g^{k+1} \subset \g^k
\end{equation}
It therefore follows that $\g^k$ is an ideal for all $k$. \\
Now, let's show that $\g^{(k)}$ is an ideal. To do this, we'll use induction on $k$.
First notice that:
\begin{equation}
[\g^{(1)},\g] = [[\g,\g],\g] \subset [\g,\g] = \g^{(1)}
\end{equation}
this is true since we know $[\g,\g] \subset \g$. It follows that
$\g^{(1)}$ is an ideal of \g.  Now all we have to prove is that
$\g^{(k)}$ is an ideal if $\g^{(k-1)}$ is one. Therefore, consider an
arbitrary element $x=[a,b] \in [\g^{(k)},\g]$ where $a \in \g^{(k)}$~
and $b \in \g$. Then, by definition of $\g^{(k)}$, $a= [c,d]$ where
$c,d \in \g^{(k-1)}$. Using the Jacobi identity, we find that:
\begin{equation}
[[c,d],b] = [[c,b],d] + [c,[d,b]]
\end{equation}
but, since $\g^{(k-1)}$ is an ideal, $[c,b],[d,b] \in \g^{(k-1)}$. Therefore, it follows that:
\begin{equation}
[a,b] = [[c,d],b] = [[c,b],d] + [c,[d,b]] \in [\g^{(k-1)},\g^{(k-1)}] = \g^{(k)}
\end{equation}
Hence, since the element $x=[a,b]$ was arbitrary, $[\g^{(k)},\g] \subset \g^{(k)}$ and therefore
$\g^{(k)}$ is an ideal for all $k$.

b)We need to show that $\g^{k+1} \supset \g^{(k)}$. Again, we'll use
induction on $k$.
\begin{equation}
\g^{(1)} = [\g,\g] = \g^2 ~~ \Rightarrow ~~ \g^{(1)} \subset \g^2
\end{equation}
Now consider an arbitrary element of $\g^(k)$, $x = [a,b]$ where $a,b
\in \g^{(k-1)}$, we're assuming that since $a \in \g^{(k-1)}$, $a \in
\g^k$ and also notice that $b \in \g$. Therefore $[a,b] \in [\g^k, \g]
= \g^{k+1}$. Hence $\g^{(k)} \subset \g^{k+1}$ for all $k$. $\square$

\paragraph{Definition}
A Lie algebra is called {\it nilpotent} (resp. {\it solvable}) if $\g^k = 0$ 
(resp. $\g^{(k)} = 0$) for $k$ sufficiently large.

\paragraph{Remark}
Notice that it follows from {\it Exercise 4.1b} that a nilpotent Lie algebra is also
solvable.

{\bf Examples} \\
1) Abelian Lie algebras are nilpotent. \\
2) The Heisenberg Lie Algebras $\He_k$ are nilpotent since 
$\He_k^{(1)} = \C$ and $[[\He_k,\He_k],\He_k]=0$ \\
3) \be $= \F a +\F b$, $[a,b]=0$ is solvable.
since \be $^{(1)} = \F b$, $\be^{(2)}=0$ but it is not
nilpotent since $\be^n = \F b$ for $n \geq 2$.


\paragraph{Exercise 4.2} Let $\bk_k(\F)$ be the subalgebra of $\gl_k(\F)$ consisting
of all upper-triangular matrices and $\nk_k(\F)$ be the subalgebra of $\gl_k(\F)$ consisting
of all strictly upper triangular matrices. Show that: \\
a) $\bk_k(\F)$~ is a solvable Lie algebra (but not nilpotent) \\
b) $\nk_k(\F)$~ is a nilpotent Lie algebra (in particular, $\nk_3(\F) \approx \He_1$)

\paragraph{Solution}
The gist of this problem is to show that the Lie bracket of any two upper triangular
matrices is even more upper triangular. It will actually prove easier to prove part (b)
first. \\
b) First notice that if $e_{ij}$ is a basis element of $\gl_k(\F)$, then the quantity $j-i$
tells us which diagonal the nonzero entry of $e_{ij}$ lies on. Let's compute the Lie
bracket $[e_{ij},e_{mn}] = \delta_{jm}e_{in} - \delta_{ni}e_{mj}$. Let's say that
$j-i=r$ and $n-m=s$. Then, when $j=m$, $n-i=n+r-j=n+r-m=s+r$, so $n-i=r+s$ and similarly
if $n=i$, then $j-m = s+r$. The payoff from this little bit of arithmetic comes from the
following observation. First, define:
\begin{equation}
\h_{k,m} = \Bigl\{\sum_{i,j}x_{ij}e_{ij} |x_{ij} \in \F; j-i \geq m; i,j \leq k \Bigr\}
\end{equation}
Then, we've shown that $[\h_{k,m},\h_{k,n}] \subset \h_{k,m+n}$. Notice also that
$\nk_k(\F) = \h_{k,1}$ and $\bk_k(\F) = \h_{k,0}$. Also notice from the definition
that $\h_{k,k+1} = 0$. Proving that $\nk_k(\F)$ is nilpotent is now trivial:
\begin{align*}
\nk_k(\F)^2 & = \h_{k,1}^2 \subset \h_{k,2} \\
\nk_k(\F)^3 & = [\h_{k,1}^2,\h_{k,1}] \subset [\h_{k,2},\h_{k,1}] \subset \h_{k,3} \\
\vdots & \\
\nk_k(\F)^k &= \h_{k,1}^k \subset \h_{k,k+1} = 0 \\
\end{align*}
Therefore $\nk_k(\F)$ is nilpotent. Now, on to proving $\bk_k(\F)$ is solvable. \\
a) To prove this, we will use a proposition that will appear later in these notes. The
proposition states that if a Lie algebra contains a solvable ideal and if the Lie algebra
modulo this ideal is solvable, then the ideal itself is solvable. Notice that 
$\nk_k(\F) \subset \bk_k(\F)$ and that $\nk_k(\F)$ is solvable (since it's nilpotent).
Also notice that $\bk_k(\F)/\nk_k(\F) = $ the set of diagonal matrices, which is abelian, hence
solvable. Therefore, by the proposition, $\bk_k(\F)$ is solvable. \\
Notice that in particular, 
\begin{equation}
\nk_3(\F) = span \left\{ q = \begin{pmatrix} 0 & 1 & 0\\ 0&0&0\\0&0&0 \end{pmatrix},
p = \begin{pmatrix} 0&0&0\\ 0&0&1\\0&0&0 \end{pmatrix},
c = \begin{pmatrix} 0 & 0 & 1\\ 0&0&0\\0&0&0 \end{pmatrix} \right\}
\end{equation}
These matrices exactly satisfy the conditions for the Heisenberg algebra $\He_1$. Hence
$\nk_3(\F) \approx \He_1$.

Also, we have proved that $\bk_k(\F)$ is solvable, but we have yet to show that is not nilpotent.
To demonstrate this, consider the commutator: $[e_{ii},e_{ij}] = e_{ij}$ for $i < j$. Since
$e_{ii}$ is in $\bk_k(\F)$, it follows that $e_{ij} \in \bk_k(\F)^2$, and therefore, 
$e_{ij} \in [\bk_k(\F),\bk_k(\F)^2] = \bk_k(\F)^3$. Hence $e_{ij} \in \bk_k(\F)^k$ for all
$k$ follows by induction. Thus $\bk_k(\F)$ is not nilpotent. $\square$


\paragraph{Remark} Notice that any subalgebra and factor (quotient) 
algebra of a nilpotent (resp. solvable)
Lie algebra is a nilpotent (resp. solvable) Lie algebra.

\paragraph{Proposition}
1) The center of a nonzero nilpotent Lie algebra \g~is nonzero. \\
2) If \g/center(\g) is a nilpotent Lie algebra, then \g~is a nilpotent Lie algebra.


\paragraph{Proof}
a) Let $k \geq 2$ be the minimal integer such that $\g^k = 0$. Then
$\g^{k-1} \neq 0 $ and $[\g^{k-1},\g] = \g^k = 0$. Hence $\g^{k-1}
\subset center(\g)$ \\ b) If \g/center(\g) is a nilpotent Lie algebra
then any commutator in \g/center(\g) of length $k$~$(k \gg 0)$ is
zero: \\
\begin{align}
 0 &= [[ \dots [a_1+e,a_2+e],a_3+e], \dots ,a_k+e] \\
&=[\dots [a_1,a_2],a_3],\dots ,a_k] +e
\end{align}
Hence in \g, we have: $[\dots [a_1,a_2], \dots ,a_k] \in e$, where $e$ denotes 
the center of \g. Hence any commutator of length $k+1$ in \g~is zero, in other words,
$\g^{k+1} =0$.

\paragraph{Theorem} ({\it The Engel Characterization of Nilpotent Lie Algebras}) \\
A finite dimensional Lie algebra \g~is nilpotent if and only if $\ad{a}$ is a nilpotent
operator for any $a \in \g$.
\paragraph{Proof} If \g ~ is nilpotent, then for $k \gg 0$, any commutator of length
$k$ is zero, in particular, $(\ad{a})^kb=0$, being a commutator of length $k+1$. \\
Conversely, suppose $(\ad{a})^k = 0$ for all $a \in \g$, $k \gg 0$. Conside the
adjoint representation defined by:
\begin{align}
\g & \rightarrow \gl_{\g} \\
a & \mapsto \ad{a}
\end{align}
Then $\ad{\g} \subset \gl_{\g}$ is isomorphic to $\g/center(\g)$ since $center(\g) = \Ker \ad$.
So it suffices to prove that $\ad{\g}$ is a nilpotent Lie algebra. Now, we can apply Engel's 
Theorem: all operators from $\ad{\g} \subset \gl_{\g}$ are nilpotent, hence 
$\ad{\g} \subset \nk_{\g}$ (the Lie algebra of strictly upper-triangular matrices, for some choice
of a basis of \g), and so \ad{\g} is a subalgebra of a nilpotent Lie algebra, hence is a nilpotent
Lie algebra itself. In particular, we see that $(\ad{a})^{dim \g}=0$.

\paragraph{Discussion} What follows is a discussion of the classification of finite dimensional 
nilpotent Lie algebras.

Consider only the 2-step nilpotent Lie algebras: $\g \supset \mathfrak{e}$ where $\mathfrak{e}=center(\g)$ and 
$\bar{\g} = \g/\mathfrak{e}$ is abelian (if and only if $\g^3 = 0$). \\
We have the following invariant of \g : the map $\varphi$,
\begin{align}
\varphi : \bar{\g} \otimes \bar{\g} & \rightarrow e \\
\varphi(\bar{a} \otimes \bar{b}) & = [a,b]
\end{align}
Where $a$ and $b$ are the pre-images in \g~ of $\bar{a}$ and $\bar{b}$.
The only condition on $\varphi$ is skew-symmetry:
\begin{equation}
\varphi(\bar{b} \otimes \bar{a}) = -\varphi(\bar{a} \otimes \bar{b})
\end{equation}
So we have a bijective correspondence between the set of 2-step nilpotent Lie algebras and the
set of skew-symmetric maps ($\varphi: \bar{\g} \otimes \bar{\g} \rightarrow center(\g)$).

\paragraph{Exercise 4.3} Show that if \g~ is 2-step nilpotent and dim center(\g) = 1, then
$\g \approx \He_k$.

\paragraph{Solution}
By definition, \g~is nilpotent if $\g/center(\g)$ is abelian. Therefore, 
$[\g,\g] \subset center(\g)$. If the dimension of the center of \g~is one, the we can conclude
that the derived algebra of \g also has dimension one. Therefore, by a Exercise 2.3, we can conclude
that: $\g = \be \oplus ab_{n-2}$ or $\g = \He_k \oplus ab_{n-2k-1}$ where $ab_n$ denotes
the n-dimensional Lie abelian algebra. However, $\be \oplus ab_{n-2}$ is not nilpotent. Therefore
$\g = \He_k \oplus ab_{n-2k-1}$. However, we know that dim($center(\g)$)= 1, hence $\g=\He_k$. $\square$





\paragraph{Remark} As discussed in lecture, the next case, dim center(\g) is a ``wild'' 
problem and the general
classification of nilpotent Lie algebras is believed to be impossible, if not very difficult.

\paragraph{Proposition} Let \g~be a Lie algebra and let \h~be an ideal of \g. Then \g~is
solvable if and only if \g/\h~is solvable and \h~is solvable.

\paragraph{Remark} This is not true if ``solvable'' is replaced by ``nilpotent'', as the case
of \be, where $[a,b]=b$, demonstrates.

\paragraph{Proof} \g~is solvable, therefore \h~is solvable and \g/\h~is solvable. \\
Conversely, if $\bar{\g} = \g/\h$ is solvable, then $\g^{(k)} \subset \h$. However, if in
addition, \h~is solvable (i.e. $\h^{(l)}=0$ for some $l \geq 1$), then
$\g^{(k+l)} \subset \h^{(l)} = 0$, so $\g^{(k+l)}=0$. Hence \g~is solvable.

\end{document}
