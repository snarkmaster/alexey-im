% -*- Mode: LaTeX -*-
\documentclass[12pt, fullpage]{article}
\usepackage{fullpage}
\usepackage{latexsym}
\usepackage{amsthm}
\usepackage{amssymb}
%\setlength{\parskip}{\baselineskip}
\setlength{\parindent}{0em}
%\renewcommand{\baselinestretch}{2}
%\renewcommand{\theenumi}{(\arabic{enumi})}
\newtheorem{theorem}{Theorem}
\newtheorem{corollary}{Corollary}
%\newtheorem{coroflem}{Corollary of the Lemma}
\newtheorem{proposition}{Proposition}
\newtheorem{lemma}{Lemma}
\newtheorem{definition}{Definition}
\newtheorem{example}{Example}
\newtheorem{exercise}{Exercise}[section]

%\newcounter{Lcount}
%\newenvironment{letterlist}{\setcounter{Lcount}{1}\begin{list}{\alph{Lcount})}}{\end{list}}

\title{Lecture 8 -- October 5, 2004}
\author{Prof. Victor Ka\v{c}\\ Scribe: Genya Zaytman}
\date{}

\renewcommand{\b}{\mathfrak{b}}
\newcommand{\g}{\mathfrak{g}}
\newcommand{\h}{\mathfrak{h}}
\newcommand{\n}{\mathfrak{n}}
\newcommand{\gl}{\mathfrak{gl}}
\newcommand{\sL}{\mathfrak{sl}}
\newcommand{\ad}{{\rm ad\ }}
\newcommand{\rank}{{\rm rank\ }}
\renewcommand{\char}{{\rm char\ }}
\newcommand{\F}{\mathbb{F}}
\renewcommand{\H}{\mathfrak{H}}
\newcommand{\U}{\mathcal{U}}
\newcommand{\V}{\mathcal{V}}
\newcommand{\Ngh}{N_\g(\h)}
\newcommand{\nFI}{\n_n+\F I_n}
\newcommand{\go}{\g_0}
\newcommand{\goa}{{\g_0^a}}

\begin{document}

\maketitle

\setcounter{section}{8}

\begin{definition}
Let $\h$ be a subalgebra of a Lie algebra $\g$.  The \emph{normalizer} of
$\h$ in $\g$ is the subalgebra $N_\g(\h)=\{a\in\g|[a,\h]\in\h\}$. 
\end{definition}

It is a subalgebra by the Jacobi identity and one has $\h\subset
N_\g(\h)$.

\begin{lemma}
If $\g$ is nilpotent and $\h\subsetneq\g$ is a subalgebra then
$\h\subsetneq N_\g(\h)$.
\end{lemma}
\begin{proof}
We have $\g\supset\g^2\supset\ldots
\supset\g^j\supset\g^{j+1}\supset\ldots \supset\g^N=0$ for $N\gg 0$.
Consider the maximum $j$ such that $\g^j\not\subset\h$, so
$\g^{j+1}\subset\h$.  Then $[\h,\g^j]\subset\g^{j+1}$.  Hence
$\g^j\subset\Ngh$.
\end{proof}

\begin{definition}
A \emph{Cartan subalgebra} of a Lie algebra $\g$ is a nilpotent
subalgebra, $\h$ such that $\h=\Ngh$.
\end{definition}

\begin{corollary}
Any Cartan subalgebra is maximal among nilpotent subalgebras.
\end{corollary}

\begin{exercise}
The subalgebra $\n_n$ (of strictly upper triangular matrices) is a
maximal nilpotent subalgebra in $\b_n$, $\sL_n$ and $\gl_n$, it is not
Cartan in any of them.
\end{exercise}
\begin{proof}[Solution]
$\n_n$ is not Cartan since, as was shown earlier,
$[\b_n,\n_n]\subset\n_n$ and thus $\b_n\subset N_\g(\n_n)$.

Unfortunately, $\n_n$ is not maximal.  Consider the algebra $\n_n+\F
I_n\supsetneq\n_n$.  Since $I_n$ commutes with everything, this
algebra is equal to $\n_n\oplus\F I_n$.  Hence, it is nilpotent and
strictly contains $\n_n$.
\end{proof}

{\bf Correction to Exercise.} $\n_n+\F I_n$ is a maximal nilpotent
subalgebra in $\b_n$ and $\gl_n$.  Unless $n=\char \F=2$, in which the
embedding of $\H_1$ given in Exercise 5.2 is a counter-example.  (Note:
that $\n_n+\F I_n$ is not Cartan either since $\b_n\subset
N_\g(\n_n)\subset N_\g(\n_n+\F I_n)$.

\begin{proof}[Solution]
We have shown earlier that $\n_n+\F I_n$ is nilpotent.  Note that it
suffices to show $\nFI$ is maximal in $\gl_n$.  Suppose there exists
some nilpotent $\h\supsetneq\nFI$.  First note that in fact we have
$\b_n=N_{\gl_n}(\nFI)$, since if $b=\sum_{i,j}c_{ij}E_{ij}\in
N_\g(\nFI)$ with $c_{i'j'}\neq 0$ for $i'>j'$, then
$E_{j'i'}\in\n_n\subset \nFI$ and $[b,E_{j'i'}]= \sum_i
c_{ij'}E_{ii'}-\sum_j c_{i'j}E_{j'j}$.  Note that the $(i',i')^{\rm
th}$ and the $(j',j')^{\rm th}$ entries are $c_{i'j'}$ and
$-c_{i'j'}$, both of which are non-zero.  Thus
$[b,E_{j'i'}]\not\in\nFI$, unless $n=\char \F=2$.  Now by the lemma
above we must have $\h\cap N_{\gl_n}(\nFI)\supsetneq \nFI$.  Thus $\h$
contains some element of $\b_n\setminus \nFI$, but all elements of
$\b_n\setminus \nFI$ have at least two distinct eigenvalues, and hence
are not ad-nilpotent, contradicting Engel's theorem.
\end{proof}

Examples of Cartan subalgebras:

\begin{enumerate}%{(\arabic{\theemumi})}{}
\item Let $\g$ be a nilpotent Lie algebra.  Then all Cartan
  subalgebras are just $\g$.
\item Let $\g\subset\gl_n$ be a subalgebra containing a diagonal
  matrix with distinct eigenvalues.  Then $\h=\g\cap\{{\rm subalgebra\
  of\ all\ diagonal\ matrices}\}$ is a Cartan subalgebra of $\g$.

\begin{proof}
$\h$ is commutative, hence nilpotent.  Let
  $b=\sum_{i,j}c_{ij}E_{ij}\in\Ngh$ then $[a,b]$ is a diagonal matrix,
  but $[a,E_{ij}=(\lambda_i-\lambda_j)E_{ij}$ where
  $a=diag(\lambda_1,\ldots,\lambda_n)$.  Therefore, $c_{ij}=0$ for
  $i\neq j$ and $b$ is a diagonal matrix.
\end{proof}
\end{enumerate}

\begin{proposition}
Assume $\F$ is an algebraicly closed field of characteristic $0$.  Let
$\h\subset\g$ be a Cartan subalgebra.  Since $\h$ is nilpotent, we may
consider the generalized root space decomposition:
$\g=\bigoplus_{\alpha\in\h^*} \g_\alpha$, $\g_\alpha=\{v\in\g|(\ad
h-\alpha(h))^Nv=0\ {\rm for}\ N\gg0\ {\rm and\ all}\ h\in\h\}$.  Then
$\g_0=\h$.
\end{proposition}
\begin{proof}
$\h\subset\go$ since $(\ad h)^Nh'=0$ for $h,h'\in\h$ by nilpotency.

Now suppose $\h\neq\go$.  Consider the adjoint representation of $\h$
on $\go$ and $\go/\h$.  Since $\ad h$ is a nilpotent operator on $\go$
for each $h\in\h$, it is also nilpotent on $\go/\h$.  Hence by Engel's
theorem, we have $\bar{a}\in\go/\h$ $\bar{a}\neq 0$, which is
annihilated by $\h$.  But this means that if $a$ is a pre-image of
$\bar{a}$ in $\go$ then $[\h,a]\subset\h$ so $a\in\Ngh\setminus\h$, a
contradiction.
\end{proof}

\begin{theorem}
\emph{(E. Cartan)}
Let $\g$ be a finite-dimensional Lie algebra over an algebraicly
closed  field $\F$.  Let $a$ be a regular element of $\g$, and let
$\g=\bigoplus_{\lambda\in\F} \g_\lambda^a$ be the generalized
eigenspace decomposition for $\ad a$.  Then $\go^a$ is a Cartan
subalgebra of $\g$.  Consequently, any finite-dimensional Lie algebra
over an algebraicly closed field contains a Cartan subalgebra.
\end{theorem}
\begin{proof}
The fact that $a$ is regular means that $\dim \go^a=\rank \g\leq\dim
\go^b$ for all $b\in\g$.

We can decompose $\g=\goa\oplus V$, where $V=\bigoplus_{\lambda\neq
0} \g_\lambda^a$.  Since $[\g_\lambda^a,\g_\mu^a]\subset
\g_{\lambda+\mu}^a$, we see that $[\go^a,V]\subset V$.  Hence the
adjoint representation induces a representation $\pi$ of $\goa$ on
$V$.

First we show that $\goa$ is a nilpotent Lie algebra.  Consider the
following subsets of $\goa$:\\
$\U=\{u\in\goa|(\ad u)_\goa {\rm
  is\,not\,a\,nilpotent\,operator}\}$,\\
$\V=\{v\in\goa|(\ad v)_V {\rm is\,a\,nonsingular\,operator}\}$.\\
$a\in\V$ by definition of $V$.  Notice that both $\U$ and $\V$ are
Zariski open subsets and that $\V$ is not empty.  Suppose that $\goa$
is not a nilpotent Lie algebra.  By Engel's theorem this means $\ad u$
on $\goa$ is a non-nilpotent operator for some $u\in\goa$.  This means
that $\U$ is non-empty.  Thus $\U\cap\V\neq\emptyset$.  Take
$b\in\U\cap\V$.  Then $\ad b$ is non-singular on $V$ and not nilpotent
on $\goa$.  Hence, $\go^b\subsetneq\goa$.  This contradicts the
regularity of $a$.

It remains to show that $N_\g(\goa)=\goa$.  If $b\in N_\g(\goa)$ then
(in particular) $[a,b]\in\goa$.  Therefore, $(\ad a)^N[a,b]=0$ for
$N\gg0$, thus $(\ad a)^{N+1}b=0$.  Hence $b\in\goa$, and
$N_\g(\goa)=\goa$.
\end{proof}

{\bf Application.}  Classification of $3$-dimensional Lie algebras
over an algebraicly closed field.  Let $\g$ be a $3$-dimensional Lie
algebra.  Let $\h$ be a Cartan subalgebra obtained by the above
procedure so that $r=\rank\g=\dim\h$.

\begin{description}
\item[Case 1: $\rank\g=3$.]  Then $\g$ is nilpotent.  Hence, either
$\g$ is abelian, or it is non-abelian hence $\dim Z(\g)=1$.
Therefore, by Exercise 3.1, $\g\cong\H_1$, the Heisenberg algebra.
\item[Case 2: $\rank\g=2$.]  Then $\h$ is a $2$-dimensional nilpotent
algebra.  Hence $\h$ is abelian.  Consider its generalized eigenspace
decomposition, $\g=\h\oplus\F b$.

\begin{exercise}
Prove that in this case $\g\cong\langle a,b|[a,b]=b\rangle\oplus \F c$
{\rm (}$\h=\F a+\F c${\rm )}.
\end{exercise}
\begin{proof}[Solution]
Since $\b\not\in\h=\Ngh$, there must exits some $a\in\h$ such that
$[a,b]\not\in\h$.  Thus $[a,b]=b+d$ for $d\in\h$.  Replacing $b$ by
$b+d$ we may assume $[a,b]=b$.  Since $\dim\h=2$, and $\dim\g/\h=1$
there must exist a $c\in\h$ such that $[b,c]\in\h$.  Then we have: $0=
[b,0]= [b,[a,c]]= [[b,a],c]+[a,[b,c]]= [-b,c]+0$, since $\h$ is
abelian.  Thus $[b,c]=0$, and $\g$ must have the given structure.
\end{proof}

\item[Case 3: $\rank\g=1$.]  Then $\h=\F a$.

\begin{exercise}
One has $3$ possibilities:
\begin{enumerate}
\item $[a,b]=b$,  $[a,c]=c+b$,  $[b,c]=0$.
\item $[a,b]=b$,  $[a,c]=\lambda c$,  $[b,c]=0$,\\
where $\lambda\in\F\setminus\{0\}$ is a parameter.
\item $[a,b]=b$,  $[a,c]=-c$,  $[b,c]=a$.
\end{enumerate}
\end{exercise}
\begin{proof}[Solution]
Let $V=[a,\g]$.  Since $\F a$ is a Cartan subalgebra, $V$ must have
dimension precisely $2$, and $\ad a$ acts on $V$ without singularities.
Scaling $a$ if necessary, we may assume one of the eigenvalues of $\ad
a|_V$ is $1$.  Suppose first that $\ad a|_V$ is not semisimple.  Thus
$\ad a|_V= \left( \begin{array}{ccc} 1 & 1 \\ 0 & 1 \end{array}
\right)$ in some basis, $\{b,c\}$, that is, $[a,b]=b$ and $[a,c]=b+c$.
Also $[a,[b,c]]= [[a,b],c]+[b,[a,c]]= [b,c]+[b,b+c]= 2[b,c]$.  But $2$
can't be an eigenvector of $\ad a$, thus $[b,c]=0$, this is the first
possibility.  Now suppose $\ad a|_V$ is semisimple.  Then $\ad a|_V=
\left( \begin{array}{ccc} 1 & 0 \\ 0 & \lambda \end{array} \right)$ in
some basis $\{b,c\}$, that is, $[a,b]=b$ and $[a,c]=\lambda c$.  Now
$[a,[b,c]]= [[a,b],c]+[b,[a,c]]= [b,c]+[b,\lambda c]=
(1+\lambda)[b,c]$.  Thus we must have either $[b,c]=0$, possibility 2,
and in this case $\lambda$ is arbitrary and uniquely defined by $\g$
up to to inverting it (swap $b$ and $c$ and scale $a$ appropriately),
or $1+\lambda$ an eigenvalue of $\ad a$, this requires $1+\lambda=0$,
thus $\lambda=-1$ and that $[b,c]$ be a multiple of $a$, scaling $b$
if necessary we may assume $[b,c]=a$, possibility 3.
\end{proof}

\end{description}

\begin{exercise}
Show that the third algebra in case 3 is isomorphic to $\sL_2(\F)$
{\rm (}if the characteristic of $\F$ is not $2${\rm )}, and that all
the algebras from exercise 8.3 are not isomorphic.
\end{exercise}
\begin{proof}[Solution]
The standard basis for $\sL_2(\F)$ consists of $\{h=\left(
\begin{array}{ccc} 1 & 0 \\ 0 & -1 \end{array} \right), x=\left(
\begin{array}{ccc} 0 & 1 \\ 0 & 0 \end{array} \right), y=\left(
\begin{array}{ccc} 0 & 0 \\ 1 & 0 \end{array} \right)\}$.  With
$[h,x]=2x$, $[h,y]=-2y$, $[x,y]=h$.  Setting $a=\frac{h}{2}$,
$b=\frac{x}{\sqrt{2}}$, and $c=\frac{y}{\sqrt{2}}$, gives the
relations for the third algebra in case 3.

That all the algebras from exercise 8.3 are not isomorphic follows
from the proof of exercise 8.3 since every step in the classification,
except when otherwise mentioned and dealt with, i.e., that for the
second algebra $\lambda$ and $\frac{1}{\lambda}$ give the same
algebra, every step was uniquely determined from the algebra.
\end{proof}

\end{document}

% LocalWords:  subalgebra Cartan subalgebras eigenspace Zariski nilpotency
% LocalWords:  semisimple
