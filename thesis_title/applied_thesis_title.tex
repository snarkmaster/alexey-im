%% LyX 1.6.2 created this file.  For more info, see http://www.lyx.org/.
%% Do not edit unless you really know what you are doing.
\documentclass[11pt,english]{amsbook}
\usepackage[T1]{fontenc}
\usepackage[utf8]{inputenc}
\usepackage{textcomp}
\usepackage{amsthm}
\usepackage{setspace}

%%%%%%%%%%%%%%%%%%%%%%%%%%%%%% Textclass specific LaTeX commands.
\numberwithin{section}{chapter}
\numberwithin{equation}{section} %% Comment out for sequentially-numbered
\numberwithin{figure}{section} %% Comment out for sequentially-numbered
\theoremstyle{plain}

%%%%%%%%%%%%%%%%%%%%%%%%%%%%%% User specified LaTeX commands.
\usepackage{changepage}
\usepackage{fancyhdr}
% Uncomment this if you need to have a \title{} entry, but don't want 
% LaTeX to generate its own title page.
% \let\maketitle\relax

\usepackage{babel}

\begin{document}
\thispagestyle{empty}
\begin{adjustwidth}{-0.5in}{-0.5in}
\newcommand{\TitleGap}{\vspace{0.12in}}

\begin{onehalfspace}
\begin{center}
\textbf{\LARGE \vspace{-.75in}}
\par\end{center}{\LARGE \par}
\end{onehalfspace}

\begin{center}
\textbf{\LARGE Pattern-avoidance in Binary Fillings of Grid Shapes}{\Large \TitleGap }
\par\end{center}{\Large \par}

\begin{onehalfspace}
\begin{center}
{\Large by\TitleGap }
\par\end{center}{\Large \par}

\begin{center}
{\LARGE Alexey Spiridonov}{\Large \TitleGap }
\par\end{center}{\Large \par}

\begin{center}
{\Large A.B., Princeton University, 2004\TitleGap }
\par\end{center}{\Large \par}

\begin{center}
{\Large Submitted to the Department of Mathematics}
\par\end{center}{\Large \par}

\begin{center}
{\Large in partial fulfillment of the requirements for the degree
of\TitleGap }
\par\end{center}{\Large \par}

\begin{center}
{\Large Doctor of Philosophy\TitleGap }
\par\end{center}{\Large \par}

\begin{center}
{\Large at the\TitleGap }
\par\end{center}{\Large \par}

\begin{center}
{\Large MASSACHUSETTS INSTITUTE OF TECHNOLOGY\TitleGap }
\par\end{center}{\Large \par}

\begin{center}
{\Large June 2009\TitleGap }
\par\end{center}{\Large \par}

\begin{center}
{\Large © Alexey Spiridonov, 2009. All rights reserved.\TitleGap }
\par\end{center}{\Large \par}

\begin{center}
{\Large The author hereby grants to MIT permission to reproduce and
to distribute publicly paper and electronic copies of this thesis
document in whole or in part in any medium now known or hereafter
created.\vspace{0.7in}}
\par\end{center}{\Large \par}
\end{onehalfspace}

\begin{onehalfspace}
\noindent \begin{flushleft}
{\Large Author . . . . . . . . . . . . . . . . . . . . . . . . . .
. . . . . . . . . . . . .}
\par\end{flushleft}{\Large \par}
\end{onehalfspace}

\begin{onehalfspace}
\noindent \begin{flushright}
{\Large Department of Mathematics}\\
{\Large May 8, 2009\TitleGap }
\par\end{flushright}{\Large \par}
\end{onehalfspace}

\begin{onehalfspace}
\noindent \begin{flushleft}
{\Large Certified by . . . . . . . . . . . . . . . . . . . . . . .
. . . . . . . . . . . . .}
\par\end{flushleft}{\Large \par}
\end{onehalfspace}

\begin{onehalfspace}
\noindent \begin{flushright}
{\Large Alexander Postnikov}\\
{\Large Associate Professor of Applied Mathematics}\\
{\Large Thesis Supervisor\TitleGap }
\par\end{flushright}{\Large \par}
\end{onehalfspace}

\begin{onehalfspace}
\noindent \begin{flushleft}
{\Large Accepted by . . . . . . . . . . . . . . . . . . . . . . .
. . . . . . . . . . . . .}
\par\end{flushleft}{\Large \par}
\end{onehalfspace}

\begin{onehalfspace}
\noindent \begin{flushright}
{\Large Michel X. Goemans}\\
{\Large Chairman, Applied Mathematics Committee\TitleGap }
\par\end{flushright}{\Large \par}
\end{onehalfspace}

\begin{onehalfspace}
\noindent \begin{flushleft}
{\Large Accepted by . . . . . . . . . . . . . . . . . . . . . . .
. . . . . . . . . . . . .}
\par\end{flushleft}{\Large \par}
\end{onehalfspace}

\begin{onehalfspace}
\noindent \begin{flushright}
{\Large David S. Jerison}\\
{\Large Chairman, Department Committee on Graduate Students}
\par\end{flushright}{\Large \par}
\end{onehalfspace}

\newpage{}

\begin{center}
\textbf{\Large \thispagestyle{empty}Pattern-avoidance in Binary Fillings
of Grid Shapes}{\Large \TitleGap }
\par\end{center}{\Large \par}

\begin{onehalfspace}
\begin{center}
{\Large by\TitleGap }
\par\end{center}{\Large \par}

\begin{center}
{\Large Alexey Spiridonov}
\par\end{center}{\Large \par}

\begin{center}
{\large \vspace{0.35in}Submitted to the Department of Mathematics}\\
{\large on May 8, 2009 in partial fulfillment of the}\\
{\large requirements for the degree of}\\
{\large Doctor of Philosophy}{\Large \TitleGap }
\par\end{center}{\Large \par}
\end{onehalfspace}

\begin{onehalfspace}
\begin{flushleft}
\textbf{\Large Abstract}{\Large \TitleGap }
\par\end{flushleft}{\Large \par}
\end{onehalfspace}

\noindent {\large A }\emph{\large grid shape}{\large{} is a set of
boxes chosen from a square grid; any Young diagram is an example.
We consider a notion of pattern-avoidance for $0$-$1$ fillings of
grid shapes, which generalizes permutation pattern-avoidance. A filling
avoids a set of patterns if none of its sub-shapes, obtained by removing
some rows and columns, equal any of the patterns. We focus on patterns
that are }\emph{\large pairs}{\large{} of $2\times2$ fillings. }{\large \par}

{\large Totally nonnegative Grassmann cells are in bijection with
Young shape fillings that avoid particular $2\times2$ pair, which
are, in turn, equinumerous with fillings avoiding another $2\times2$
pair. The latter ones correspond to acyclic orientations of the shape's
bipartite graph. Motivated by this result, due to Postnikov and Williams,
we prove a number of such analogs of Wilf-equivalence for these objects
--- that is, we show that, }\emph{\large in certain classes of shapes,
some pattern-avoiding fillings are equinumerous with others}{\large . }{\large \par}

{\large The equivalences in this paper follow from two very different
bijections, and from a family of recurrences generalizing results
of Postnikov and Williams. We used a computer to test each of the
described equivalences on a diverse set of shapes. All our results
are }\emph{\large nearly }{\large tight, in the sense that we found
no natural families of shapes, in which the equivalences hold, but
the results' hypotheses do not. }{\large \par}

{\large One of these bijections gives rise to some new combinatorics
on tilings of skew Young shapes with rectangles, which we name }\emph{\large Popeye
diagrams}{\large . In a special case, they are exactly Hugh Thomas's
}\emph{\large snug partitions}{\large{} for $d=2$. We show that Popeye
diagrams are a lattice, and, moreover, each diagram is a sublattice
of the Tamari lattice. We also give a simple enumerative result.\TitleGap}{\large \par}

\noindent {\large Thesis Supervisor: Alexander Postnikov}{\large \par}

\noindent {\large Title: Associate Professor of Applied Mathematics}{\large \par}

\end{adjustwidth}
\end{document}
